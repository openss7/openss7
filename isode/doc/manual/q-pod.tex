%%% Damy @ Brunel

\chapter {POD}
\label{pod}

POD is a directory user agent developed for the \xwindows/\index{X Windows}
version 11 release 4, using the Athena widget set.
It is intended to be an interface for the ``naive'' user.
If you want to use the full power of X.500 you should use the DISH interface
described in Chapter~\ref{dish} of this manual.

\section {Types of Widget}

POD makes use of various simple on-screen devices called widgets.
The most important of these are described in the following sections.

\subsection {Buttons}

A button is a rectangular screen area which when selected will activate some
specific function.
A button is selected by pointing at it with the cursor and then
clicking with the first mouse button.

Most buttons contain a label and sometimes graphics
which denote the button's function.
In addition POD buttons are denoted by a change to a pointing hand cursor or by
a border becoming highlighted.

A button box is simply a collection of buttons.

\subsection {Dialogue Boxes}

A dialogue box is a means of supplying textual information to an application,
and can be thought of as a miniature text editor.
The dialogue boxes used in POD follow the command structure of the EMACS
editor.

Dialogue boxes,
like most others widgets,
only become active (or have the input focus) when pointed at by the cursor.

\subsection {Menus}

A menu is a vertically arranged collection of buttons that appears
on selection of a menu button.

\section {Using POD}

\pgm{pod} is invoked with the following command:

\begin{quote}\begin{verbatim}
pod [-t <tailor file>] [-T <oidtable>] 
    [-c <dsa address>] [-u <quoted username>] 
    [-p <password>]
\end{verbatim}\end{quote}

\tt -t\rm \ is used to tell POD which \file{tailor} file to use in place of the
default system dsaptailor file.

\tt -T\rm \ is used to force POD to use an alternative oidtable.

\tt -c\rm \ is used to bind to a DSA other than the local default.

\tt -u\rm \ is used to bind as a specific user.

\tt -p\rm \ is used to bind against the given password.

POD is made up of a number of separate windows,
described in the following sections.

\subsection {The Main Window}

POD's main window is comprised of three sections:
a current position display,
a search input box and a button box.
Each of these is described below.

\subsubsection {Current Directory Position}

All operations are performed relative to a base directory entry.
The name of this entry is displayed in the main window under the title 
``Current Directory Position''.

Parts of the displayed entry name can be selected in order to move to
other positions in the directory.

\subsubsection {Searching for Entries}

The search input area contains a dialogue box and a menu button.

The dialogue box is used to enter a value describing an entry to be searched
for;
for example if searching for the entry for ``Damanjit Mahl'' a suitable value
would be ``D Mahl''.

The menu button specifies the type of entry being searched for (the search
type);
in the above example this would be ``Person''.
The search type can be changed by pulling down the menu attached to the type
button and selecting from the contained list of types.

Searches are activated by clicking on the ``Search'' button or by pressing the 
Return key.

\subsubsection {Main Functions}

The buttons displayed in POD's main window are:
\begin{describe}
\item [\verb+Quit+:] Quit from POD.
\item [\verb+Help+:] Invoke interactive help.
\item [\verb+Search+:] Search for specified entry.
\item [\verb+List+:] List entries under current position.
\item [\verb+History+:] Display history of visited entries.
\end{describe}
	
\subsection {The List Window}

A {\em list} window displays a list of named directory entries,
which can be returned as a result of list,
search or history operations.
It comprises a pair of buttons,
a list display and two message bars.

The upper message bar contains the information regarding the source of the
displayed list, e.g., ``Result of List under Brunel University''.
The lower message bar contains errors and constraint messages.

\subsubsection {List Display}

To move to or read a listed entry simply click on that entry's name.
Non-leaf entries cause a move and read,
whereas leaf entries only cause a read.

\subsubsection {Close and Keep Buttons}

The {\em close} button can be used to close the {\em list} window.
Note that this does not iconify the window but rather deletes it and all
contained information.

The {\em keep} button can be used to make the information contained in
the {\em list}
window semi-permanent.
In the default case,
results from ensuing list/search operations overwrite any unkept list 
windows,
thereby limiting the number of windows containing temporary information to one.
If the {\em keep} button is selected,
displayed information becomes semi-permanent,
and is only removed if the {\em close} button is selected.
The text in a {\em keep} button changes from Keep to Kept upon selection,
in order to distinguish from list windows containing transient data.

The {\em list} window that displays the session history does not have
a {\em keep} button.

\subsection {The Read Window}

The {\em read} window displays selected parts of a directory entry.
It is comprised of a message bar, 
a button box and a {\em text} window.

The message bar displays the entry's name.
If selected this bar places the complete distinguished name of an entry
into the primary X cut buffer.
Ordinarily this is of little use,
though it can help when modifying attributes that require a distinguished name as
a value.

The {\em text} window displays the body of the entry,
which currently contains textual information and may contain
a (single) fax image stored in the photo attribute.

It is possible to cut text from the {\em text} window.

The {\em close} and {\em keep} buttons perform a similar function to those described above
for the {\em list} window.

The {\em modify} button activates a window containing facilities to modify the
entry being read.

\subsection {The Modify Window}

The {\em modify} window allows the user to modify the attributes of an entry.
Each attribute value displayed in the {\em text} window is contained in its
own dialogue box.
Pointing to a dialogue box will allow editing of the contained value.

Each value has an associated menu button (denoted by the menu
icon) which allows various operations on a particular value,
e.g., delete value,
undo changes, etc.
In addition each attribute label, e.g., 
``commonName'',
is itself a menu button and allows various operations on all values
associated with that attribute,
e.g., delete all values,
undo all changes to this attribute,
add a new value field.
Undo operations only undo all changes since the last successful modify
operation or else return all affected values to their original states.

The {\em close} and {\em keep} buttons behave consistently with those described for
the {\em read} and {\em list} windows.

The {\em modify} button attempts to make the modification entered into
the {\em text} window.
A message reporting the success (or otherwise) of a modification request
is displayed in the lower right {\em message} window.

\subsection {Error and Message Popups}

From time to time,
during or after directory operations,
POD will provide error or status reports in a popup that appears in
the top left corner of the screen.
Errors sometime require the user to click on the {\em error} window before
normal operation can resume.

\section {Configuration of POD}

POD can be configured on a system-wide or per-user basis.
POD installs system default files into a directory called \file{xd/duaconfig/}
under ISODE's ETCDIR.

Per-user configuration files other than \file{.podrc},
must be held under a directory \file{.duaconfig/} 
contained in a user's home directory.

The POD configuration files are
\begin{quote}\begin{verbatim}
<CONFIGDIR>/readTypes,
<CONFIGDIR>/duaconfig/friendlyNames,
<CONFIGDIR>/duaconfig/filterTypes/Type_*,
<CONFIGDIR>/duaconfig/typeDefaults.
\end{verbatim}\end{quote}

\subsection {The .podrc File}

The \file{.podrc} file is analogous to the \file{.quipurc} file for DISH
(see Section~\ref{quipurc}),
though less extensive in the number and flexibility of options provided.
It has the following format:

\begin{quote}\begin{verbatim}
flag: value
\end{verbatim}\end{quote}

The following flags are currently recognised:

\begin{describe}

\item [\verb+username+:] The name of the user to bind as.

\item [\verb+password+:] The password to use when binding. For this reason care
should be taken to ensure your \file{.podrc} file is not publicly readable.

\item [\verb+service+:] A list of default service control flags
(as defined in Section~\ref{dish_serv}).

\item [\verb+dsap+:] This can be followed by one of the dsaptailor options
described in Section~\ref{dua:tailor}.\index{dsaptailor}

\item [\verb+isode+:]
This can be followed by one of the ISODE tailor options
described in \volone/ of this manual.

\item [\verb+history+:]
This can be followed by a number that specifies the number of entry names
to be maintained in the history buffer.

\item [\verb+prefergreybook+:]
For those who prefer grey book mail address format.
This takes values of TRUE or FALSE.

\item [\verb+readnonleaf+:]
This tells POD whether or not to read non-leaf entries.
Values are TRUE or FALSE.

\end{describe}

\subsection {The readTypes File}

The file \file{readTypes} contains a list of attributes that are to be read for
entry display.
The format of this file is;
\begin{quote}\small\begin{verbatim}
"quotedAttributeName" numericOid
\end{verbatim}\end{quote}
an example line then being
\begin{quote}\small\begin{verbatim}
"photo"               0.9.2342.19200300.100.1.7
\end{verbatim}\end{quote}

This format was chosen as it is the same format as the output from the
{\em oiddump}
utility provided with ISODE.

\subsection {The friendlyNames File}

This file maps attribute names onto user-friendly names,
for use in displaying the ``Current Directory Position''.
The current defaults map the names onto empty strings so that the
bare values are shown,
e.g.:

\begin{quote}\begin{verbatim}
The World
GB
Brunel University
\end{verbatim}\end{quote}

The format of the file is

\begin{quote}\begin{verbatim}
attribute list : friendly name
\end{verbatim}\end{quote}
where an attribute list can be one or more comma-separated attribute names.

\subsection {The filterTypes Files}
	
POD searches are based upon a complex filter;
e.g., the default filter for the type {\em Person} is
\begin{quote}\begin{verbatim}
objectClass=person AND (cn~=* OR sn~=* OR title~=*)
\end{verbatim}\end{quote}
Where ``*'' represents a value supplied at search
time.

The directory \file{filterTypes/} contains a set of files,
each with a prefix \file{Type\_},
describing the set of such abstract types used in POD.
The set of contained files may be edited or added to.

The syntax used in these files is shown in the example below:

\begin{quote}\small\begin{verbatim}
<filter_type> ::= <filter_name> <filter>
<name> ::= "name:" <asciistring>
<filter> ::= <filter_item> | <assertion>
<assertion> ::= "(" <filt_type> <filter> <filter>
                <filter_list> ")"
<filter_list> ::= <filter> <filter_list> | <filter> | NULL
<filter> ::= <filter_item> | <assertion>
<filter_item> ::= "(" NUMERIC_OID <match_type> <value> ")"
<filt_type> ::= "&" | "|"
<match_type> ::= "=" | "~=" | "%="
<value> ::= "*" | STRING
<comment> ::= "#" STRING
\end{verbatim}\end{quote}

Where the symbol ``$~$='' represents approximate match,
``\%='' represents substring match and ``='' represents exact match.
The ``*'' character for value represents a value supplied at search time.

As an example,
the file representing the default type ``Person'' is:

\begin{quote}\begin{verbatim}
#Composition of type Person
name:"Person"
( & (2.5.4.0="2.5.6.6") # objectClass = person
    ( | (2.5.4.3~=*)	# cn (common name)
	(2.5.4.4~=*)	# sn (surname)
	(2.5.4.12~=*)))	# title
\end{verbatim}\end{quote}

\subsection {The typeDefaults File}

The \file{typeDefaults} file defines the relationships between each of the
type defined in the \file{filterTypes} directory.
This is best described by the following example line from the provided
\file{typeDefaults} file:
\begin{quote}\begin{verbatim}
2.5.4.10:Person, Place, Department: Person
\end{verbatim}\end{quote}
The first field defines the type of entry to which this line applies,
the OID shown here is the one for {\em organizationName}.
The second field lists the POD search types which are available when visiting 
an entry of the specified type. 
So this line specifies that the types {\em Person},
{\em Place} and {\em Department} are available when visiting an entry of
type {\em organizationName}.
The third field defines the default search type to be used when visiting
an entry of the specified type.
