% run this through LaTeX with the appropriate wrapper

\chapter {SD}
\label{sd}

SD is a simple directory user agent for X.500 directory services.

SD supports the following directory operations:
\begin{itemize}
\item read
\item list
\item search
\end{itemize}

SD is limited in it's flexibility and does not allow modification of entries.
If you want to use the full power of X.500 you should use the DISH interface 
described in Chapter~\ref{dish}.

SD is curses based, and splits the users screen up into a number of
small areas called ``widgets''.  Each widget is used to enter information.

\section {Types of Widget}

There are various types of ``widget'' used by the SD program, these are
described in the following sections.

Each widget (except label widgets) have an identification
``letter'' associated with them,
this is indicated by the left most character in the widget, for example,
the widget below is identified by the letter ``\tt b\rm ''.

\begin{tabular}{|l|}
\hline
\tt h help\ \ \\
\hline
\end{tabular}

\subsection {Label Widgets}
Label widgets are used to simply write information to the screen, they do not 
accept any form of input.

\subsection {Command Widgets}
A command widget is used to make the program take some kind of action, such as
send a {\em list} operation to the directory,
or to change the set of widgets displayed.
  
To issue the command type the letter indicated in
the widget. 

\subsection {Dialog Widgets}

Dialog widgets are used to get information from the user.
To use a dialog widget, press the letter associated with the widget.
Any further input from the keyboard will be entered into the widget,
a ``return'' character is used to end the input.

\subsection {Toggle Widgets}

Toggle widgets are used to select one of a set of pre-determined values.
Simply enter the widget letter, and the value will change the next value in
the circular list.

\section {Using SD}

The SD interface is invoked with the \pgm{sd} command.  
Help is given at any stage by typing ``\tt ?\rm '', type ``\tt q\rm '' to quit.

\subsection {Main Options}
The ``main options'' set of widgets,
presents the users with {\em quit, help, list, widen, history, search for} and
{\em goto} commands.

The {\em search area} label widget informs the user of the currently visited
directory entry.
All directory operations are performed relative to the current {\em search
area}.

\subsection {List}

The {\em list} operation causes SD to list a limited number of entries below
the {\em search area}.
The results are displayed in an enumerated list.

\subsection {Search For}

The {\em search for} command requires the user to enter a {\em search value}
after invocation.
The {\em search value} can be a single filter item or a value based on the 
current search type, 
displayed in the {\em type} label widget.
In general a value based on the current type is used.
For example if the current type is ``Person'' a suitable value might be
``findlay'' or ``nahajski'',
or if the current type is ``Department'' then a suitable value might be
``Computer Centre''.

The attribute type is selected with the {\em type} toggle widget.

Any entries that match the given information are displayed in an enumerated
list.
If a single match is made then the contents of that entry are displayed
and it also becomes the current {\em search area},
assuming that it is a non-leaf entry.

\subsection {Go To Number}

The {\em go to number} operation allows the user to visit and read a listed
entry (lists can be displayed as a result of {\em list} or {\em search}
operations).
To visit and view a listed entry,
simply type in the number of the entry directly.

\subsection {Scrolling Lists}

In order to simplify viewing of long lists of entry names a simple scrollbar
feature is provided.

The scrollbar is displayed on the left side of the viewing area.
The entries currently being viewed are represented by the block of ``*''
characters, 
the length of the complete scrollbar represents the length of the entire
list.

To scroll downwards press the ``['' key,
to scroll upwards press the ``]'' key.

\subsection {Help}

The {\em help} command invokes a set of command widgets.
Each command will bring up help text concerning the specified topic,
e.g., the {\em search for} operation.

\section {Configuration of SD}

SD can be configured on a system-wide or per-user basis.
SD installs system default files into a directory called \file{sd/duaconfig/}
under \verb+ISODE+'s ETCDIR.

Per-user configuration files other than \file{.duarc},
must be held under a directory \file{.duaconfig/} 
contained in a user's home directory.

The SD configuration files are: 
\begin{verbatim}
<CONFIGDIR>/readTypes,
<CONFIGDIR>/duaconfig/filterTypes/Type_*,
<CONFIGDIR>/duaconfig/typeDefaults.
\end{verbatim}

\subsection {The .duarc file}

The \file{.duarc} file is analogous to the \file{.quipurc} file for DISH
(see Section~\ref{quipurc}),
though less extensive in the number and flexibility of options provided.
It has the following format:

\begin{verbatim}
flag: value
\end{verbatim}

The following flags are currently recognised:

\begin{describe}

\item [\verb+username+:] The name of the user to bind as.

\item [\verb+password+:] The password to use when binding. For this reason care
should be taken to ensure your \file{.duarc} file is not publicly readable.

\item [\verb+service+:] A list of default service control flags
(as defined in Section~\ref{dish_serv}).

\item [\verb+dsap+:] This can be followed by one of the dsaptailor options
described in Section~\ref{dua:tailor}.\index{dsaptailor}

\item [\verb+isode+:]
This can be followed by one of the ISODE tailor options
described in \volone/ of this manual.

\item [\verb+history+:]
This can be followed by a number that specifies the number of entry names
to be maintained in the history buffer.

\item [\verb+prefergreybook+:]
For those who prefer grey book mail address format.
This takes values of TRUE or FALSE.

\end{describe}

\subsection {The readTypes file}

The file \file{readTypes} contains a list of attributes that are to be read for
entry display.
The format of this file is;
\begin{quote}\small\begin{verbatim}
"quotedAttributeName" numericOid
\end{verbatim}\end{quote}
an example line then being
\begin{quote}\small\begin{verbatim}
"photo"         0.9.2342.19200300.100.1.7
\end{verbatim}\end{quote}

This format was chosen as it is the same format as the output from the OIDDUMP
utility provided with ISODE.

\subsection {The filterTypes files}
	
SD searches are based upon a complex filter,
e.g., the default filter for the type {\em Person} is:
\begin{verbatim}
objectClass=person AND (cn~=* OR sn~=* OR title~=*)
\end{verbatim}
Where \* represents a value supplied at search
time.

The directory \file{filterTypes/} contains a set of files,
each with a prefix \file{Type\_},
describing the set of such abstract types used in SD.
The set of contained files may be edited or added to.

The syntax used in these files is shown in the example below:

\begin{verbatim}
<filter_type> ::= <filter_name> <filter>
<name> ::= "name:" <asciistring>
<filter> ::= <filter_item> | <assertion>
<assertion> ::= "(" <filt_type> <filter> <filter> 
                                         <filter_list> ")"
<filter_list> ::= <filter> <filter_list> | <filter> | NULL
<filter> ::= <filter_item> | <assertion>
<filter_item> ::= "(" NUMERIC_OID <match_type> <value> ")"
<filt_type> ::= "&" | "|"
<match_type> ::= "=" | "~=" | "%="
<value> ::= "*" | STRING
<comment> ::= "#" STRING
\end{verbatim}

Where the symbol ``~='' represents approximate match,
``\%='' represents substring match and ``='' represents exact match.
The ``*'' character for value represents a value supplied at search time.

As an example,
the file representing the default type {\em Person} is:

\begin{verbatim}
#Composition of type Person
name:"Person"
( & (2.5.4.0="2.5.6.6") # objectClass = person
    ( | (2.5.4.3~=*)	# cn (common name)
	(2.5.4.4~=*)	# sn (surname)
	(2.5.4.12~=*)))	# title
\end{verbatim}

\subsection {The typeDefaults file}

The \file{typeDefaults} file defines the relationships between each of the
type defined in the \file{filterTypes} directory.
This is best described by the following example line from the provided
\file{typeDefaults} file.
\begin{verbatim}
2.5.4.10:Person, Place, Department: Person
\end{verbatim}
The first field defines the type of entry to which this line applies,
the OID shown here is the one for {\em organizationName}.
The second field lists the SD search types which are available when visiting 
an entry of the specified type. 
So this line specifies that the types {\em Person},
{\em Place} and {\em Department} are available when visiting an entry of
type {\em organizationName}.
The third field defines the default search type to be used when visiting
an entry of the specified type.
