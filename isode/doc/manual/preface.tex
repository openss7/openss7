% run this through LaTeX with the appropriate wrapper

\preface
The software described herein has been developed as a research tool and
represents an effort to promote the use of the International 
Organisation for Standardisation (ISO) interpretation of open systems interconnection (OSI),
particularly in the Internet and RARE research communities.

\newpage\section*	{Notice, Disclaimer, and Conditions of Use}\label{license}
The ISODE is openly available but is {\bf NOT\/} in the public domain.
You are allowed and encouraged to take this software and build commercial
products.
However, as a condition of use, you are required to ``hold harmless'' all
contributors.

\noindent
Permission to use, copy, modify, and distribute this software and its
documentation for any purpose and without fee is hereby granted, provided
that this notice and the reference to this notice
appearing in each software module be retained unaltered, 
and that the name of any contributors shall not be used in advertising
or publicity pertaining to distribution of the software without specific
written prior permission.
No contributor makes any
representations about the suitability of this software for any purpose.
It is provided ``as is'' without express or implied warranty.

\vskip 0.15in
\noindent
\begin{small}
{\bf All contributors disclaim all warranties with regard to this
software, including all implied warranties of me\-chan\-ti\-bil\-ity
and fitness. In no event shall any contributor be liable for any
special, indirect or consequential damages or any damages whatsoever
resulting from loss of use, data or profits, whether in action of
contract, ne\-gli\-gence or other tortuous action, arising out of or
in connection with, the use or performance of this software.}
\end{small}

\vskip 0.15in
\noindent
As used above,
``contributor'' includes, but is not limited to:
\begin{quote}
\begin{tabular}{l}
The MITRE Corporation\\
The Northrop Corporation\\
NYSERNet, Inc.\\
Performance Systems International, Inc.\\
University College London\\
The University of Nottingham\\
X-Tel Services Ltd\\
The Wollongong Group, Inc.\\
Marshall T. Rose
\end{tabular}
\end{quote}
In particular,
the Northrop Corporation provided the initial sponsorship for the ISODE
and the Wollongong Group, Inc., also supported this effort.
The ISODE received partial support from the U.S.~Defense Advanced Research
Projects Agency and the Rome Air Development Center of the U.S.~Air Force
Systems Command under contract number F30602--88--C--0016 to NYSERNet Inc.

\newpage\section*	{Revision Information}
This document (version \versiontag/) and its companion volumes are believed
to accurately reflect release v~\isodevrsn/ of \today{}.

\newpage\section*	{Release Information}
If you would like a copy of the release described in this document,
there are several avenues available:
\begin{itemize}
\item	NORTH AMERICA\\
For mailings in NORTH AMERICA,
send a check for 375 U.S. Dollars to:
\[\begin{tabular}{ll}
Postal address:&University of Pennsylvania\\
&		\small Department of Computer and Information Science\\
&		Moore School\\
&		Attn: David J. Farber (ISODE Distribution)\\
&		200 South 33rd Street\\
&		Philadelphia, PA 19104-6314\\
&		U.S.A.\\[0.1in]
Telephone:&	+1 215 898 8560
\end{tabular}\]

\pagebreak[3]
Specify one:
\begin{enumerate}
\item	1600bpi 1/2--inch tape, or

\item	Sun 1/4--inch cartridge tape.
\end{enumerate}
The tape will be written in \pgm{tar} format and returned with
a documentation set.
Do not send tapes or envelopes.
Documentation only is the same price.  

\item	EUROPE\\
For mailings in EUROPE, send a cheque or bankers draft and a purchase order
for 200 Pounds Sterling to:  
\[\begin{tabular}{ll}
Postal address:&	Department of Computer Science\\
&			Attn: Natalie May/Dawn Bailey\\
&			University College London\\
&			Gower Street\\
&			London, WC1E 6BT\\
&			UK
\end{tabular}\]
For information only:
\[\begin{tabular}{ll}
\ Telephone:&		+44 71 380 7214\\
\ Fax:&			+44 71 387 1397\\
\ Telex:&		28722\\
\ Internet:&		\verb"natalie@cs.ucl.ac.uk"\\
&			\verb"dawn@cs.ucl.ac.uk"
\end{tabular}\]
Specify one:
\begin{enumerate}
\item	1600bpi 1/2--inch tape, or

\item	Sun 1/4--inch cartridge tape.
\end{enumerate}
The tape will be written in \pgm{tar} format and returned with
a documentation set.
Do not send tapes or envelopes.
Documentation only is the same price.  

\item	EUROPE (tape only)\\
Tapes without hardcopy documentation can be obtained via the European
Forum for Open Systems (EurOpen, formerly known as EUUG).
The ISODE~\isodevrsn/ distribution is called EurOpenD14.
\[\begin{tabular}{ll}
Postal address:&	EurOpen Software Distributions\\
&			c/o Frank Kuiper\\
&			Centrum voor Wiskunde en Informatica\\
&			Kruislann 413\\
&			1098 SJ  Amsterdam\\
&			The Netherlands\\[0.1in]
For information only:&\\
\ Telephone:&		+31 20 5924056\\
	&		(or +31 20 5929333)\\
\ Telex:&		12571 mactr nl\\
\ Telefax:&		+31 20 5924199\\
\ Internet:&		\verb"euug-tapes@cwi.nl"
\end{tabular}\]
Specify one:
\begin{enumerate}
\item	1600bpi 1/2--inch tape: 140 Dutch Guilders.

\item	Sun 1/4--inch cartridge tape (QIC-24 format): 200 Dutch Guilders.

\end{enumerate}
If you require DHL this is possible and will be billed through.
Note that if you are not a member of EurOpen
then there is an additional handling fee of 300 Dutch Guilders
(please enclose a copy of your membership or contribution payment form when
ordering). 
Do not send money, cheques, tapes or envelopes,
you will be invoiced.

\item	PACIFIC RIM\\
For mailings in the Pacific Rim,
send a cheque for 300 dollars Australian to:  
\[\begin{tabular}{ll}
Postal address:&	Isode Distribution\\
&			(Attn: Andrew Waugh)\\
&			723 Swanston St\\
&			Carlton, VIC 3053\\
&			Australia
\end{tabular}\]
For information only:
\[\begin{tabular}{ll}
\ Telephone:&		+61 3 282 2615\\
\ Fax:&			+61 3 282 2600\\
\ Internet:&		\verb"ajw@mel.dit.csiro.au"
\end{tabular}\]
Specify one:
\begin{enumerate}
\item	1600/3200/6250bpi 1/2-inch tape, or

\item	Sun 1/4-inch cartridge tape in either QIC-11, QIC-24 or QIC-150 format.

\item	Exabyte 2.3 Gigabyte or 5 Gigabyte format

\end{enumerate}
The tape will be written in \pgm{tar} format and returned with a documentation set.
Do not send tapes or envelopes.
Documentation only is the same price.

\item	Internet\\
If you can FTP to the Internet,
you can use anonymous FTP to the host \verb"uu.psi.com"
\verb"[136.161.128.3]"
to retrieve \compressfile/ in BINARY mode from the \tarplace/ directory.
This file is the \pgm{tar} image after being run through the compress program
and is approximately \compressize/ in size.

\item	NIFTP\\
If you run NIFTP over the public X.25 or over JANET, and are
registered in the NRS at Salford, you can use NIFTP with username
``guest'' and your own name as password, to access \verb"UK.AC.UCL.CS" to
retrieve the file \ukcompressfile/.
This file is the \pgm{tar} image after being run through the compress program
and is approximately \compressize/ in size.

\item	FTAM on the JANET, IXI or PSS\\
The source code is available by FTAM at the University College London
over X.25 using 
\begin{describe}
\item JANET 
(DTE \verb"00000511160013")
\item IXI 
(DTE \verb+20433450420113+)
\item PSS 
(DTE \verb"23421920030013") 
\end{describe}
Use TSEL~\verb"259" (ASCII encoding).
Use the ``anon'' user-identity and retrieve the file 
\compressfile/ from the \uktarplace/ directory.
This file is the \pgm{tar} image after being run through the compress program
and is approximately \compressize/ in size.

The file service is provided by the FTAM implementation in ISODE~6.0 or later
(IS FTAM) and is registered in the pilot OSI Directory below
\begin{quote}\footnotesize\begin{verbatim}
bells, Computer Science, University College London, GB
\end{verbatim}\end{quote}


%%%\item	FTAM on the Internet\\
%%%The source code is available by FTAM over the
%%%Internet at host 
%%%\linebreak
%%%\verb"osi.nyser.net" \verb"[192.33.4.10]"
%%%(TCP port~102 selects the OSI transport service)
%%%with TSEL~\verb"259" (numeric encoding).
%%%Use the ``anon'' user-identity, supply any password,
%%%and retrieve \compressfile/ from the \tarplace/ directory.
%%%This file is the \pgm{tar} image after being run through the compress program
%%%and is approximately \compressize/ in size.

\end{itemize}

For distributions via either FTAM or FTP, there are two additional files
available for retrieval:

\begin{itemize}

\item  \Docfile/\\
This is the \LaTeX source for the entire documentation set.
It is a compressed \pgm{tar} image (\DocSize/).

\item  \PSfile/\\
This contains the five volume manual in PostScript format.
It is a compressed \pgm{tar} image (\PSsize/).

\end{itemize}



\newpage\section*	{Discussion Groups}
The Internet discussion group {\tt ISODE@NIC.DDN.MIL\/} is
used as a forum to discuss ISODE.
Contact the Internet mailbox {\tt ISODE-Request@NIC.DDN.MIL\/}
to be asked to be added to this list.

\subsection*{Support}
Although the ISODE is not ``supported'' per se, it does have a problem
reporting address, {\tt Bug-ISODE@XTEL.CO.UK\/}.  
Bug reports (and fixes) are
welcome by the way. 

\subsection*{Commercial Support}
A UK company has been set up to provide commercial support for the ISODE and
associated packages --- X-Tel~Services Ltd.  This company provides an update
service, general assistance and site specific support.  Although inclusion
of this information should not be considered an endorsement, it should be
noted that two of the primary ISODE developers now work at X-Tel~Services
Ltd.

\[\begin{tabular}{ll}
Postal address:&	X-Tel Services Ltd.\\
&	University Park,\\
&	Nottingham, NG7 2RD\\
&        UK\\[2em]
Telephone&	+44 602 412648\\
Fax&		+44 602 790278\\
Internet&	support@xtel.co.uk\\
\end{tabular}\]

If other organisations offering formal support for the ISODE wish to be
included in future announcements, a suitable index will be organized.



\newpage\section*	{Acknowledgements}
Many people have made comments about and contributions to the ISODE which have
been most helpful.
The following list is by no means complete:

The first three releases of the ISODE were developed at the Northrop
Research and Technology Center,
and the first version of this manual is referenced as NRTC Technical
Paper \#702.
The initial work was supported in part by Northrop's Independent
Research and Development program.

The Wollongong Group supported ISODE for its 4.0 and 5.0 release.
They deserve much credit for that.
Furthermore,
they contributed an implementation of RFC1085,
a lightweight presentation protocol for TCP/IP-based internets.

The ISODE development up until version 7.0 was 
supported by Performance Systems International,
Inc.~and \mbox{NYSERNet}, Inc.
This was a 
substantial commitment to which the project in indebted.
Before their support, ISODE was always an after-hours activity.
The \mbox{NYSERNet} effort was partially supported by the U.S.~Defense Advanced Research
Projects Agency and the Rome Air Development Center of the U.S.~Air Force
Systems Command under contract number F30602--88--C--0016 to NYSERNet Inc.

Christopher W.~Moore\index{Moore, Christopher W.}
of the Wollongong Group has provided much help with ISODE in terms of both
policy and implementational matters.
He also performed Directory interoperability testing against a different
implementation of the OSI Directory.

Dwight E.~Cass\index{Cass, Dwight E.}
of the Northrop Research and Technology Center was one of the original
architects of {\em The ISO Development Environment}.
His work was critical for the original proof of concept and should not be
forgotten.
John L.~Romine\index{Romine, John L.}
also of the Northrop Research and Technology Center provided many
fine comments concerning the presentation of the material herein.
This resulted in a much more readable manuscript.
Stephen H.~Willson\index{Willson, Stephen H.},
also of the Northrop Research and Technology Center,
provided some help in verifying the operation of the
software on a system running the AT\&T variant of \unix/.

The \man librosap(3n) library was heavily influenced by an earlier native-TCP
version written by George Michaelson\index{Michaelson, George},
formerly of University College London, in
the United Kingdom.
Stephen E.~Kille\index{Kille, Stephen E.},
of University College London,
provided valuable feedback on the \man pepy(1) utility.
In addition,
both Steve and George provided us with some good comments concerning the
\man libpsap(3) library.
Steve is also the conceptual architect for the addressing scheme used in
the software,
and he modified the \man librosap(3n) library to support half-duplex mode when
providing ECMA ROS service.
George contributed the CAMTEC and \ultrix/ X.25 interfaces.
Simon Walton\index{Walton, Simon},
also of University College London,
has been very helpful in providing constant feedback on the ISODE during
beta-testing.

The INCA project donated the QUIPU Directory implementation to the ISODE.
Stephen E.~Kille\index{Kille, Stephen E.},
Colin J.~Robbins\index{Robbins, Colin J.},
and Alan Turland\index{Turland, Alan},
at the time all of University College London,
are the three principals who developed the 6.0 version of the
directory software.
In addition,
Steve Titcombe\index{Titcombe, Steve},
also of UCL
spent considerable time on the DIrectory SHell (DISH), and 
Mike Roe\index{Roe, Mike} formerly of UCL,
put a large amount of effort into the security requirements of QUIPU.
Development of the current version of QUIPU has been coordinated by 
Colin J.~Robbins now of X-Tel Services Ltd, and designed by
Stephen E.~Kille.

The UCL work has been partially supported by the commission of the
EEC\index{EEC} under its ESPRIT\index{ESPRIT} program,
as a stage in the promotion of OSI standards.
Their support has been vital to the UCL activity.
In addition,
QUIPU is also funded by the UK Joint Network Team (JNT).

Julian P.~Onions\index{Onions, Julian},
of X-Tel Services Ltd
is the current \man pepy(1) guru,
having brainstormed and implemented the encoding functionality along with
Stephen E.~Easterbrook\index{Easterbrook, Stephen}
formerly of University College London.
Julian also contributed the UBC X.25 interface 
along with the TCP/X.25 TP0 bridge,
and has also contributed greatly to \man posy(1).
Julian's latest contribution has been a {\em transport service bridge}.
This is used to masterfully solve interworking problems between different OSI
stacks (TP0/X.25, TP4/CLNP, RFC1006/TCP, and so on).

John Pavel\index{Pavel, John}
and Godfrey Cowin\index{Cowin, Godfrey}
of the Department of Trade and Industry/ National Physical Laboratory in the
United Kingdom
both contributed significant comments during beta-testing.
In particular, John gave us a lot of feedback on \man pepy(1)
and on the early FTAM DIS implementation.
John also contributed the SunLink X.25 interface.

Keith Ruttle\index{Ruttle, Keith} of CAMTEC Electronics Limited in the United
Kingdom contributed both the driver for the new CAMTEC X.25 interface and
the CAMTEC CONS interface (X.25 over 802 networks).
This latter driver was later removed from the distribution for lack of use.

In addition,
Andrew Worsley\index{Worsley, Andrew}
of the Department of Computer Science at the University of Melbourne in
Australia
pointed out several problems with the FTAM DIS implementation.
He also developed a replacement for \pgm{pepy} and \pgm{posy} called
\pgm{pepsy}.
After moving to University College London,
he improved this system and integrated into the ISODE.

Olivier Dubous\index{Dubous, Olivier}
of BIM sa in Belgium contributed some fixes to concurrency control in the FTAM
initiator to allow better interworking with the \vms/ implementation of the
filestore.
He also suggested some changes to allow interworking with the FTAM T1 and
A/111 profiles.

Olli Finni\index{Finni, Olli}
of Nokia Telecommunications
provided several fixes found when testing interoperability with the TOSI
implementation of FTAM.

Mark R.~Horton\index{Horton, Mark R.}
of AT\&T Bell Laboratories
also provided some help in verifying the operation of the
software on a 3B2~system running \unix/ System~V release~2.
In addition,
Greg Lavender\index{Lavender, Greg} of NetWorks One,
under contract to the U.S.~Navy Regional Data Automation Center (NARDAC),
provided modifications to allow the software to run on a generic port of
\unix/ System~V release~3.

Steve D.~Miller\index{Miller, Steve D.}
of the University of Maryland
provided several fixes to make the software run better on the \ultrix/ variant
of \unix/.

Jem Taylor\index{Taylor, Jem}
of the Computer Science Department at the University of Glasgow
provided some comments on the documentation.

Hans-Werner Braun\index{Braun, Hans-Werner}
of the University of Michigan provided the inspiration for the initial part of
Section~\ref{name}.

A previous release of the software contained an ISO TP4/CLNP package
derived from a public-domain implementation developed by the National
Institute of Standards and Technology\index{NIST}
(then called the National Bureau of Standards\index{NBS}).
The purpose of including the NIST package (and associated support)
was to give an example of how one would interface the code to a ``generic'' TP4
implementation.
As the software has now been interfaced to various native TP4 implementations,
the NIST package is no longer present in the distribution.

John A.~Scott\index{Scott, John A.}
of the MITRE Corporation contributed the SunLink OSI interface for TP4.
He also wrote the FTAM/FTP gateway which the MITRE Corporation has generously
donated to this package. 

Philip B.~Jelfs\index{Jelfs, Philip B.}
of the Wollongong Group upgraded the FTAM/FTP gateway to the
``IS-level'' (International Standard) FTAM.

Rick Wilder\index{Wilder, Rick}
and
Don Chirieleison\index{Chirieleison, Don}
of the MITRE Corporation contributed the VT implementation which the MITRE
Corporation has generously donated to this package.

Jacob Rekhter\index{Rekhter, Jacob}
of the T.~J.~Watson Research Center, IBM Corporation,
provided some suggestions as to how the system should be ported
to the IBM RT/PC running either AIX or 4.3\bsd/.
He also fixed the incompatibilities of the FTAM/FTP gateway when running on
4.3\bsd/ systems.

Ashar Aziz\index{Aziz, Ashar}
and
Peter Vanderbilt\index{Vanderbilt, Peter},
both of of Sun Microsystems Inc.,
provided some very useful information on modifying the SunLink OSI interface
for TP4.

Later on,
elements of the SunNet 7.0 Development Team
(Hemma Prafullchandra, Raj Srinivasan, Daniel Weller, and Erik Nordmark)
\index{Prafullchandra, Hemma}
\index{Srinivasan, Raj}
\index{Weller, Daniel}
\index{Nordmark, Erik}
made numerous enhancements and fixes to the system.

John Brezak\index{Brezak, John}
of Apollo Computer, Incorporated 
ported the ISODE to the Apollo workstation.
Don Preuss\index{Preuss, Don},
also of Apollo,
contributed several enhancements and minor fixes.

Ole-Jorgen Jacobsen\index{Jacobsen, Ole-Jorgen} of Advanced Computing
Environments provided some suggestions on the presentation of the material
herein.

Nandor Horvath\index{Horvath, Nandor}
of the Computer and Automation Institute of the Hungarian Academy of Sciences
while a guest researcher at the DFN/GMD in Darmstadt, FRG,
provided several fixes to the FTAM implementation and documentation.

George Pavlou\index{Pavlou, George}
and Graham Knight\index{Knight, Graham}
of University College London contributed some management instrumentation to
the \man libtsap(3n) library.

Juha Hein\"{a}nen\index{Hein\"{a}nen, Juha}
of Tampere University of Technology
provided many valuable comments and fixes on the ISODE.

Paul Keogh\index{Keogh, Paul}
of the Nixdorf Research and Development Center, in Dublin, Ireland,
provided some fixes to the FTAM implementation.

Oliver Wenzel\index{Wenzel, Oliver}
of GMD Berlin contributed the RFA system.

L.~McLoughlin\index{McLoughlin, L.}
of Imperial College contributed Kerberos support for the FTAM responder.

Kevin E.~Jordan\index{Jordan, Kevin E.}
of CDC provided many enhancements to the G3FAX library.

John A.~Reinart\index{Reinart, John A.}
of Cray Research contributed many performance enhancements.

Ed Pring\index{Pring, Ed}
of the T.~J.~Watson Research Center, IBM Corporation,
provided several fixes to the SMUX implementation in ISODE's SNMP
agent.

Uli Betzler\index{Betzler, Uli}
of University of Karlsruhe
donated the HP X.25 interface and other HP-UX modifications.

Alan Young\index{Young, Alan}
of Concurrent Computer Corporation contributed many helpful changes to
the transport code.

James Gosling,\index{Gosling, James}
author of the superb Emacs screen editor for \unix/,
and
Leslie Lamport,\index{Lamport, Leslie}
author of the excellent \LaTeX{} document preparation system
both deserve much praise for such winning software.
Of course,
the whole crew at U.C.~Berkeley\index{U.C.~Berkeley}
also deserves tremendous praise for their wonderful work on their variant of
\unix/.\\[1.5ex]

Finally, full credit for this release must be given to 
Marshall T. Rose\index{Rose, Marshall T.}.
Marshall has for so long master-minded the whole ISODE effort.
Although he has not been involved in the last month or so leading upto
this release, the vast majority of the effort getting ISODE-7.0 into 
shape came from Marshall.

\vspace{0.25in}
%%%{\raggedleft /mtr\par}
%%%{\raggedright Mountain View, California\par}
{\raggedleft jpo \& cjr \par}
{\raggedright Nottingham, England\\
\ifcase\month
    \number\month\or
    January\or February\or March\or April\or May\or June\or
    July\or August\or September\or October\or November\or December\else
\number\month\fi,
{\oldstyle\number\year}\par}
