% run this through LaTeX with the appropriate wrapper

\chapter	{The ISO Macros Database}\label{isomacros}
The database \file{isomacros} in the ISODE \verb"ETCDIR" directory
(usually \file{/usr/etc/})
contains a simple mapping between
user-friendly strings and network addresses.
This database is used by when trying to resolve textual respresentations of
network addresses
(as desccribed in Section~\ref{addr:encodings}) for use with the network.

The database itself is an ordinary ASCII text file containing an entry for
each locally defined macro.
Each entry contains
\begin{itemize}
\item	the macro, a simple string; and,

\item	the prefix of the corresponding network address.
\end{itemize}
Blanks and/or tab characters are used to seperate items.
However, double-quotes may be used to prevent separation for items containing
embedded whitspace.
The sharp character (`\verb"#"') at the beginning of a line indicates a
commentary line.

Unlike the other databases in the ISODE,
the user may not directly access this file.
The routine \verb"str2paddr" and \verb"paddr2str" use this automatically.

\section	{User-Specific Macros}
By default a user-specific macros database is consulted before the
system-wide macros file.
The user-specific file is called \file{\$HOME/.isode\_macros} in the user's
home directory.
