% run this through LaTeX with the appropriate wrapper
% adt@castle.ed.ac.uk

\chapter {The Async DAP Procedural Interface}

Originally, the procedural interface providing a representation of
the X.500 DAP operations was only required to provide synchronous
behaviour.
It is likely that many useful applications will be written which only
require synchronous representation of operations; so that having
bound to a DSA, operations are constructed in some way and then
the routine representing the relevant operation is called and will
block until an appropriate response has been received from the DSA.

However, in more sophisticated DUAs it will be desirable to have
asyn\-chron\-ous or non-blocking routines representing the various
DAP operations, so that the DUA can construct and send operations,
perform useful work whilst waiting for the operations to complete,
and collect and use the responses to operations as they become
available.

For the above reasons, a new set of procedure calls has been written
as part of the \man libdsap(3n) library. These calls provide a representation
of the X.500 operations (including the bind operation) which can be
used to program asynchronous performance of operations within a DUA.
This chapter describes those routines and their intended use.

To use the async DAP interface described here it is necessary to include
the file \file{quipu/dap2.h}.

\section {Procedure Model}

\subsection {Styles of Behaviour}

The routines which it is desirable to have to represent X.500 DAP
operations depends on the behaviour required within a DUA.
A simple DUA may be more easily developed by expecting a synchronous
behaviour from routines representing operations; DAP operations
are invoked as routines which block until a response to the
operation arrives and the DUA can then continue according to
that response.

If, for whatever reasons, it is undesirable to block every time a
DAP operation is requested, then a different behaviour is
required from the routines representing operations; the operation
request should be issued and the routine returned. Subsequently
a routine should be available to accept a response to the request
issued and process that response.

An additional twist is that there are actually three styles of
behaviour which it is desirable to make available to DUA programs:
a simple synchronous interface, an interruptible synchronous interface,
and an asyn\-chron\-ous interface.

The interruptible synchronous interface is the style of behaviour
described in the previous section. If an operation is called
synchronously and then interrupted (a Control-C to the user
interface for instance) during performance of the operation, then
an abandon DAP operation should automatically be invoked on the
outstanding operation and appropriate responses gathered from the
DSA before returning to the user.
The interruptible synchronous style of behaviour is only available
for the remote operations of the directory access context (read,
add entry, \ldots) and not for the bind and unbind operations.

Mixing of styles of behaviour is deprecated, although there are no
known problems. An asynchronous operation request followed by
a synchronous operation request may well get the response to either
operation, which means checking the responses to all synchronous
requests in the same ways as the responses to asynchronous requests
must be checked, thus losing most of the advantages that make the
synchronous style worthwhile in the first place.

In order to enable the incorporation of all three styles, a generalised
interface has been written over which the interruptible synchronous
interface described in Section~\ref{proc_model} has been reimplemented.
Thus, the routines in that previous section can be thought of as a
specialised restriction of the interface described here.

The generalised interface also differs, in that it does not provide
automatic decoding of results and errors into particular parameters,
instead returning an indication parameter which may include decoded
structures within it.
This means that the implementation of the old style interface using
the generalised routines is not as straightforward as the above
comments might suggest.

Future revisions should take on board the separation of the provision
of routines which perform DAP operations from the provision of routines
which provide DUA services in the construction of DAP operations (e.g.,
getting passwords for a bind request is a high-level DUA task;
parameterisation which allows the use of pre-allocated result and 
error return parameters is a lower-level DUA task). All that the DAP
operations should provide is the encoding of arguments, sending of
operations and appropriate subsequent behaviour.
Hopefully, future documentation will be produced which is structured
so as to enable a clearer relationship between the different interfaces
and behaviour styles provided for DAP operations.

Where each routine of the synchronous interface described in the
previous chapter had a prefix of ``\verb"ds_"'' (when using an implicit
association identifier) or of ``\verb"dap_"'' (when using an explicit
association identifier), all the routines in the asynchronous interface
have a prefix of ``\verb"Dap"''.
All operations in the asynchronous interface take an explicit
association identifier.

For each of the DAP bind operations, the DAP unbind operations and the
DAP remote operations (read, list, search, addentry, removeentry,
modifyentry, modifyrdn and abandon) a request routine is provided.
The request routine takes a parameter indicating the style of behaviour
to be used (synchronous, or asynchronous for the bind and unbind
operations requests; synchronous, interruptible or asynchronous for
the DAP remote operations).

If the style indicated is synchronous, then the request routine will
attempt to construct and send the appropriate operation and wait for
a response.
If the operation completes successfully (a result or an error is
returned) then the routine returns \verb"OK". Otherwise, on failure,
the routine returns \verb"NOTOK".

If the style indicated is interruptible, then the operation is sent and
an interruptible wait initiated for a response from the DSA.
If an interruption occurs (see a description of the routine
\verb"RoIntrRequest" in the rosap module for details
of what constitutes an interruption and how it is indicated), then an
abandon operation is sent for the outstanding operation and responses
from the DSA are awaited for both the abandon operation and for the
outstanding operation.
Except in the case of an interruption the style of behaviour is exactly
the same as the synchronous style.

If the style indicated is asynchronous, then the operation is sent
and the routine returns.
For the DAP bind and DAP unbind operations a retry routine is provided
for each operation which when called will either complete the
outstanding operation or indicate that it is not yet complete.
For the DAP remote operations a single wait routine is available
which will complete an outstanding operation, if there is one which
can be completed, or indicate that there is no completable operation.

The value used to indicate the style of behaviour is borrowed from the
``\verb"rosap"'' module:
\begin{describe}
\item [\verb"ROS\_SYNC":]
Simple synchronous behaviour: block until DSA responds.
\item [\verb"ROS\_INTR":]
Interruptible synchronous behaviour: block until DSA responds or an
interruption occurs. (Untidily, the definition of the value
\verb"ROS_INTR" is currently contained in \file{quipu/dap2.h}.)
\item [\verb"ROS\_ASYNC"]:
Asynchronous behaviour: send operation, do not block.
\end{describe}

\subsection {Arguments}

Arguments taken by async DAP request operations are, in general, a
binding (or association) identifier which identifies the association
on which to invoke the operation; an operation identifier which it is
the responsibility of the DUA to generate, maintain and use as a
reference (e.g., for abandoning a previously requested operation);
a representation of the argument for the operation; an indication
return parameter through which information on the outcome of the
operation can be returned to the invoker of the routine; and of course
the behaviour style indicator.

\subsection {Indications}

All of the routines representing X.500 operations in this procedural
interface take an indication parameter, the type of which is a
pointer to a pre-allocated \verb"DAPindication" structure.

\label{DAPindication}
\begin{quote}\index{DAPindication}\small\begin{verbatim}
struct DAPindication {
    int     di_type;
#define DI_RESULT       2
#define DI_ERROR        3
#define DI_PREJECT      4
#define DI_ABORT        6
    union {
        struct DAPresult        di_un_result;
        struct DAPerror         di_un_error;
        struct DAPpreject       di_un_preject;
        struct DAPabort         di_un_abort;
    } di_un;
#define di_result di_un.di_un_result
#define di_error di_un.di_un_error
#define di_preject di_un.di_un_preject
#define di_abort di_un.di_un_abort
};
\end{verbatim}\end{quote}

This structure is a discriminated union (tag element followed by
a union).
Depending on the routine called, style of behaviour requested, and
the value returned by the routine the contents of the indication
should be readily retrievable.

For the bind and unbind routines the \verb"DAPindication" structure
is only updated on failure and will contain an abort indication.

For the DAP remote operations the \verb"DAPindication" structure is
used to indicate failure of the association, failure of the request,
an error response to a request, or a result response to a request.

When the \verb"DAPindication" structure is used to indicate a
failure of the association it will contain a \verb"DAPabort"
structure.

\label{DAPabort}
\begin{quote}\index{DAPabort}\small\begin{verbatim}
struct DAPabort {
    int     da_source;

    int     da_reason;

#define DA_SIZE 512
    int     da_cc;
    char    da_data[DA_SIZE];
}
\end{verbatim}\end{quote}

\begin{describe}
\item [\verb"da\_source":]
the source of the abort, one of:
\[\begin{tabular}{|l|l|}
\hline
    \multicolumn{1}{|c|}{\bf Value}&
                \multicolumn{1}{c|}{\bf Source}\\
\hline
    \tt DA\_USER&              service-user (peer)\\
    \tt DA\_PROVIDER&          service-provider\\
    \tt DA\_LOCAL&             local DAPM\\
\hline
\end{tabular}\]


\item [\verb"da\_reason":]
The reason for aborting the association (if known). 

\item [\verb"da\_data"/\verb"da\_cc":] a diagnostic string from the
provider.
\end{describe}

When the \verb"DAPindication" structure is used to indicate a rejection
it will contain a \verb"DAPpreject" structure.

\label{DAPpreject}
\begin{quote}\index{DAPpreject}\small\begin{verbatim}
struct DAPpreject {
    int           dp_id;

    int           dp_source;

    int           dp_reason;

#define DP_SIZE 512
    int     dp_cc;
    char    dp_data[DP_SIZE];
}
\end{verbatim}\end{quote}

\begin{describe}
\item [\verb"dp\_id":]
the operation identifier of the operation rejected.

\item [\verb"dp\_source":]
the source of the abort, taking the same values as \verb"da_source"
in the \verb"DAPabort" structure described above.

\item [\verb"dp\_reason":]
The reason for rejecting the operation (if known). 

\item [\verb"dp\_data"/\verb"dp\_cc":] a diagnostic string from the
provider.
\end{describe}

When the \verb"DAPindication" structure is used to indicate a result
in response to a previously issued remote operation request
it will contain a \verb"DAPresult" structure.

\label{DAPresult}
\begin{quote}\index{DAPresult}\small\begin{verbatim}
struct DAPresult {
    int                   dr_id;
    struct DSResult       dr_res;
}
\end{verbatim}\end{quote}

\begin{describe}
\item [\verb"dr\_id":]
the operation identifier of the operation for which this is the result.

\item [\verb"dr\_res":]
the decoded result structure.
\end{describe}

The operation identifier should be used on the receipt of an operation
result to determine the operation the result applies to, and the
result structure handled appropriately.

When the \verb"DAPindication" structure is used to indicate a error
in response to a previously issued remote operation request
it will contain a \verb"DAPerror" structure.

\label{DAPerror}
\begin{quote}\index{DAPerror}\small\begin{verbatim}
struct DAPerror {
    int                   de_id;
    struct DSError        de_err;
}
\end{verbatim}\end{quote}

\begin{describe}
\item [\verb"de\_id":]
the operation identifier of the operation for which this is the error
response.

\item [\verb"de\_err":]
the decoded error structure.
\end{describe}

The operation identifier should be used on the receipt of an operation
error to determine the operation the error applies to, and the
error structure handled appropriately.

\subsection {Return Values}

The values returned by the various routines in this interface and
their meanings are more complicated than may be entirely reasonable.
However, a quick overview goes as follows:
\begin {itemize}
\item
for bind and unbind operations synchronous style and DAP remote
operations in synchronous or interruptible style, the request routines
may return \verb"NOTOK" on failure and \verb"OK" on success.

\item
for bind and unbind operations asynchronous style, the request routines
may return \verb"NOTOK" on failure, \verb"DONE" on completion, and
\verb"CONNECTING_1" or 
\verb"CONNECTING_2" when incomplete.

\item
for DAP remote operations asynchronous style, the request routines
may return \verb"NOTOK" on failure, \verb"OK" on successful invocation.
\end{itemize}

\section {Binding and Unbinding}

\subsection {Binding}

To be able to send DAP remote operations to a DSA it is necessary
to establish an association with the DSA using a DAP bind operation.
This can be accomplished either synchronously or asynchronously
using the routine \verb"DapAsynBindReqAux".

\begin{quote}\index{DapAsynBindReqAux}\small\begin{verbatim}
int       DapAsynBindReqAux (callingtitle, calledtitle,
                callingaddr, calledaddr, prequirements,
                srequirements, isn, settings, sf,
                bindarg, qos, dc, di, async)
AEI                       callingtitle;
AEI                       calledtitle;
struct PSAPaddr         * callingaddr;
struct PSAPaddr         * calledaddr;
int                       prequirements;
int                       srequirements;
long                      isn;
int                       settings;
struct SSAPref          * sf;
struct ds_bind_arg      * bindarg;
struct QOStype          * qos;
struct DAPconnect       * dc;
struct DAPindication    * di;
int                       async;
\end{verbatim}\end{quote}

This routine will attempt to call a DSA at the called address and/or
with the called title, to establish a directory access association
using the directory bind argument provided.

Often, all that is required for many of these parameters is that they
are initialised to appropriate values.
The routine \verb"DapAsynBindRequest" is provided which sets up
sensible defaults for most of the parameters before calling
the routine \verb"DapAsynBindReqAux".

\begin{quote}\index{DapAsynBindReqAux}\small\begin{verbatim}
int DapAsynBindRequest (calledaddr, bindarg, dc, di, async)
struct PSAPaddr         * calledaddr;
struct ds_bind_arg      * bindarg;
struct DAPconnect       * dc;
struct DAPindication    * di;
int                       async;
\end{verbatim}\end{quote}

\begin{describe}
\item [\verb"calledaddr":]
the address of the DSA to send the bind request to.
The \verb"PSAPaddr" structure is described in
\voltwo/, and more general discussion of addresses is given in
\volone/.

\item [\verb"bindarg":]
is a representation of the argument of the DAP bind operation.
The \verb"ds_bind_arg" structure is described in Section~\ref{proc_model}.

\item [\verb"dc":]
is used to return bind response information when a response is received.
The \verb"DAPconnect" structure comprises connection information from
the underlying association and a structure containing a representation
of the bind response (bind result or bind error) if any.

\item [\verb"di":]
is the indication parameter described above in Section~\ref{DAPindication}.

\item [\verb"async":]
can take one of two values: \verb"ROS_ASYNC" or \verb"ROS_SYNC".
If the value is \verb"ROS_SYNC", then the routine will await a response
to the association request and attempt to decode the value returned in
that response.
If the value is \verb"ROS_ASYNC", then the routine may return without
receiving an association response, which may be checked for later
using the \verb"DapAsynBindRetry" routine.
\end{describe}

The \verb"DAPconnect" structure is used to return the response to a DAP
bind operation:

\begin{quote}\index{DAPconnect}\small\begin{verbatim}
struct DAPconnect {
    int     dc_sd;
    int     dc_pctx_id;
    struct AcSAPconnect dc_connect;

    int     dc_result;
#define DC_RESULT       1
#define DC_ERROR        2
#define DC_REJECT       3

    union {
        struct ds_bind_arg        dc_bind_res;
        struct ds_bind_error      dc_bind_err;
    } dc_un;
};
\end{verbatim}\end{quote}

\begin{describe}

\item [\verb"dc\_sd":]
the association identifier assigned to the bound association.

\item [\verb"dc\_pctx\_id":]
the identifier of the directory access context in the negotiated list of
presentation contexts.

\item [\verb"dc\_connect":]
the connect information for the underlying association.

\item [\verb"dc\_result":]
the type of response received.

\item [\verb"dc\_bind\_res":]
the decoded bind result if the response is a result.

\item [\verb"dc\_bind\_err":]
the decoded bind error if the response is an error.
\end{describe}

For explanations of the other parameters to \verb"DapAsynBindReqAux"
see the description of the \verb"AcAssocRequest" routine on
page~\pageref{AcAssocRequest} of \volone/.

When the asynchronous style of behaviour is selected for a bind request,
the \verb"DapAsynBindRetry" can be used to attempt to complete the
bind and process any response.
In order to determine when to call \verb"DapAsynBindRetry", the
methods for asynchronously establishing connections described in
Section~\ref{tsap:async} on page~\pageref{tsap:async} in \voltwo/
should be applied.
Essentially, the values \verb"CONNECTING_1" and \verb"CONNECTING_2",
are used to determine whether the association establishment is
currently blocked on reading or writing, and is used in constructing
a call to \verb"xselect" to determine if further progress can be
made. If the call to \verb"xselect" indicates that further work
can be done then \verb"DapAsynBindRetry" should be called.

\begin{quote}\index{DapAsynBindRetry}\small\begin{verbatim}
int       DapAsynBindRetry (sd, do_next_nsap, dc, di)
int                       sd;
int                       do_next_nsap;
struct DAPconnect       * dc;
struct DAPindication    * di;
\end{verbatim}\end{quote}

\begin{describe}
\item [\verb"sd":]
the association descriptor returned in the \verb"DAPconnect"
structure after an asynchronous call to \verb"DapAsynBindReqAux".

\item [\verb"do\_next\_nsap":]
is used to specify whether the association should be retried on the
same NSAP of the called address originally passed to
\verb"DapAsynBindReqAux" or whether to give up on the current NSAP
and go on to the next (if any).
A value of zero will specify retrying on the same NSAP, a non-zero
value specifies retrying on the next NSAP.
This is somewhat messy as it forces the calling code to keep track
of how long a particular NSAP has been tried for and when to try
another.
If there are no more NSAPs when the next NSAP is requested, then the
association attempt is deemed to have failed and an appropriate
indication is generated.

\item [\verb"dc":]
if the routine returns \verb"DONE" then this structure will contain
the connect information as described for \verb"DapAsynBindReqAux"
above.

\item [\verb"di":]
if the routine returns \verb"NOTOK" then this structure will contain
the indication information.
\end{describe}

\subsection {Unbinding}

When a DUA no longer needs to be bound to a DSA it can issue a
DAP unbind operation to unbind from the DSA and end the association.
This is achieved by using the \verb"DapUnBindRequest" routine.

\begin{quote}\index{DapAsynBindRequest}\small\begin{verbatim}
int       DapUnBindRequest (sd, secs, dr, di)
int                       sd;
int                       secs;
struct DAPrelease       * dr;
struct DAPindication    * di;
\end{verbatim}\end{quote}

\begin {describe}

\item [\verb"sd":]
the association identifier for the association from which the
DUA wishes to unbind.

\item [\verb"secs":]
the number of seconds to spend trying to unbind.

\item [\verb"dr":]
the result of unbinding when complete.

\item [\verb"di":]
an indication.
\end{describe}

If the unbind operation fails then the routine will return \verb"NOTOK"
and an appropriate indication.

If the unbind operation completes successfully within \verb"secs"
seconds then the routine will return \verb"OK" and the \verb"DAPrelease"
structure will be filled out.

If no unbind response is received within \verb"secs" seconds the routine
will return a value of \verb"DONE" and the unbind should be completed
using the routine \verb"DapUnBindRetry".

\begin{quote}\index{DapUnBindRetry}\small\begin{verbatim}
int       DapUnBindRetry (sd, secs, dr, di)
int                       sd;
int                       secs;
struct DAPrelease       * dr;
struct DAPindication    * di;
\end{verbatim}\end{quote}

Which takes exactly the same parameters and has similar behaviour to the
routine \verb"DapUnBindRequest" above.

\section {DAP Remote Operations}

For each DAP remote operation (read, abandon, add entry, \ldots) there
is a routine in the asynchronous interface to request that operation.
The \verb"DapRead" routine will be described in detail, which taken in
conjunction with the descriptions of arguments given in
Section~\ref{DUA:proc} should
also provide adequate description of the routines \verb"DapCompare",
\verb"DapAbandon", \verb"DapList", \verb"DapSearch", \verb"DapAddEntry",
\verb"DapRemoveEntry", \verb"DapModifyEntry", and \verb"DapModifyRDN".

All these routines can be invoked with synchronous, interruptible or
asynchronous styles of behaviour. In the synchronous and interruptible
cases the call will not return until a response is received (in the
interruptible style an interruption will cause an automatic sending
of an abandon request for the operation requested by the routine and
will await and return the response to that operation).
In the asynchronous case the routine will return after the request
has been sent and specific procedures for receiving responses must be
undertaken using the \verb"DapInitWaitRequest" routine.

\subsection {Invoking Requests}

\begin{quote}\index{DapRead}\small\begin{verbatim}
int       DapRead (ad, id, arg, di, asyn)
int                       ad;
int                       id;
struct ds_read_arg      * arg;
struct DAPindication    * di;
int                       asyn;
\end{verbatim}\end{quote}

\begin{describe}

\item [\verb"ad":]
the association identifier for the bound association over which to send
the read operation.

\item [\verb"id":]
the operation identifier to assign the the read operation.

\item [\verb"arg":]
the read argument.

\item [\verb"di":]
an indication parameter, only used if the request fails.

\item [\verb"asyn":]
the behaviour style selector.
\end{describe}

If \verb"DapRead" is invoked with synchronous behaviour, then it
will block until a response arrives, and hope that the response
is for the operation just requested, although no guarantees are
made that this is so.

If the interruptible style is selected then the call may be interrupted
by some signals and an abandon operation for the operation requested
is generated when this occurs, with the routine then awaiting both
the response to the abandon operation and to the original operation
before returning.

If the asynchronous style is selected, then the call issues the request
and returns, the response should be expected in a subsequent call to
the routine
\verb"DapInitWaitRequest".

\subsection {Receiving Responses}

\begin{quote}\index{DapInitWaitRequest}\small\begin{verbatim}
int       DapInitWaitRequest (sd, secs, di)
int                       sd;
int                       secs;
struct DAPindication    * di;
\end{verbatim}\end{quote}

\begin{describe}

\item [\verb"sd":]
the association identifier.

\item [\verb"secs":]
the number of seconds to spend waiting for a response; \verb"OK" will
produce a polling effect and \verb"NOTOK" will produce a blocking
effect.

\item [\verb"di":]
the indication structure.
\end{describe}

If the \verb"DapInitWaitRequest" routine returns \verb"OK" then there
is a response contained in the indication parameter.
If it returns \verb"DONE" then there was no response available within
the specified time.

\section {Programming Comments}

This interface has been written with expectations of the following
sort of program structure in mind.

Through interaction with a user or from arguments or tailoring, a
DUA will undertake to attempt one or more DAP bind operations to
one or more DSAs.

An asynchronous call to \verb"DapAsynBindReqAux" should be made using
an appropriate bind argument for each DSA that the DUA wishes to bind to.
If the association completes then operations can be constructed and
invoked on the bound association identified by the value returned in
the \verb"dc_sd" fields of the \verb"DAPconnect" structure \verb"dc_sd".

If the value returned by the \verb"DapAsynBindRequest" procedure call is
either 
\verb"CONNECTING_1" or \verb"CONNECTING_2" then this value should
be recorded along with the value returned in the \verb"dc_sd" field
of the \verb"dc" parameter, which identifies the association even
though it is not yet complete.
Operations {\em must not} be invoked using this identifier until the
bound association is completely established.

At a later stage, for each outstanding association request, depending
on whether the last value returned for the association attempt on that
association identifier is \verb"CONNECTING_1" or \verb"CONNECTING_2",
the routine \verb"xselect" should be used to check for writing or
reading on the value returned by \verb"PSelectMask" for the given
association identifier.

If the call to \verb"xselect" indicates that writing (or reading) is
available then the routine \verb"DapAsynBindRetry" should be called
for that association identifier.

This should be repeated until the association attempt fails or is
completed, at which point operations can be constructed and invoked over
the bound association.
This is done using the appropriate routine for the required operation.

If operations are invoked asynchronously, then a call to
the procedure \verb"DapInitWaitRequest" will need to check for responses
to outstanding operations and to return such a response as an
indication if there is one available.
The DUA can then process responses to the operations it invoked.

When the DUA no longer needs to be bound to a DSA, the
procedure \verb"DapUnBindRequest" can be used to initiate the termination
of a bound association, and if necessary, the \verb"DapUnBindRetry"
routine used to complete the binding.
