% run this through LaTeX with the appropriate wrapper

\chapter	{Installation and Configuration}\label{installation}
This chapter describes how to install, configure, and test the pilot project
software.
Upon completion,
your Level-1 DSA will have successfully joined the pilot project DMD!

\section	{Software}
The software supporting the pilot project is the QUIPU implementation
of the OSI Directory,
which is a part of the ISO Development Environment (ISODE).

QUIPU was originally developed as a part of the INCA project
(under the auspices of the ESPRIT initiative of the EEC).
The Inca of Peru did not have writing.
Instead,
they stored information on strings,
carefully knotted in a specific manner with colored thread,
and attached to a larger rope.
Such a device was known as a {\em Quipu\/}
(pronounced {\em kwip-ooo}).
The encoding was obscure,
and could only be read by selected trained people:
the {\em Quipucamayocs}.
The Quipu was a key component of Inca society,
as it contained information about property and locations throughout
the extensive Inca empire.

The pilot sponsors and the administrators of the participating organizations
form the {\em camayocs\/} for the pilot project.
There is a special mailing list
\begin{quote}\small\begin{verbatim}
wpp-camayocs@psi.com
\end{verbatim}\end{quote}
which each \camayoc/ belongs to.

\[\fbox{\begin{tabular}{lp{0.8\textwidth}}
\bf NOTE:&	This is the official mailing list for administrators of the
		pilot project.
		If you have a problem, send mail!
\end{tabular}}\]

\subsection	{Selecting a Machine}
The machine you run the software on should support the Berkeley variant of
\unix/.
Although the software is known to run on other variants of \unix/, such as
those from AT\&T and HP,
these configurations are not supported by the pilot sponsors.

You should plan on the DSA consuming 2Kbytes of memory {\em per\/} entry kept
in your DMD
This is called the QUIPU {\em 2K rule}.
Make sure that your machine has available memory for the DSA.
A memory-constrained (or worse yet, paging DSA) performs extremely poorly.
Find an unloaded system to run your DSA.
This investment might seem large,
just think of it as a sunk cost.

\subsection	{Files}
Regardless of how you install QUIPU and the ISODE,
the number of files needed for the pilot project are quite small.

In ISODE's \verb"BINDIR" directory,
typically \file{/usr/local/bin/},
there are a few programs of interest:
\begin{describe}
\item[dish:]	The DIrectory SHell\\
		This is a interface to the OSI Directory which can be used to
		access the full functionality of the Directory service.
		This program is not directly run by the ``users'' of the pilot
		project, rather various front-ends access \pgm{dish}.
		In addition,
		the administrators employ this program directly to perform
		routine maintenance and diagnostics.

\item[bind:]	Shell interface to \pgm{dish}\\
		There are actually several links to a program called
		\pgm{bind}.
		These act to export the \pgm{dish} interface to the \unix/
		shell.
		As such,
		you can issue commands to \pgm{dish} from the shell,
		rather than running \pgm{dish} directly.
\begin{quote}
\begin{tabular}{l}
add\\
compare\\
delete\\
dsacontrol\\
list\\
modify\\
modifyrdn\\
moveto\\
search\\
showentry\\
showname\\
squid
\end{tabular}
\end{quote}

\item[fred:]	A white pages user interface\\
		This is the interface for the users of the pilot project.

\item[editentry:]	Edit a Directory entry\\
		This is a simple shell script that \pgm{dish} invokes when you
		ask \pgm{dish} to edit an entry in the Directory.

\item[unbind:]	Unbind from \pgm{dish}\\
		This command is used to terminate \pgm{dish}.
\end{describe}
In ISODE's \verb"SBINDIR" directory,
typically \file{/usr/etc/},
the DSA resides:
\begin{describe}
\item[ros.quipu:]	The QUIPU DSA\\
			This program will be started once, for each DSA you
			are running, from rc.local.
			A script is provided to invoke this program,
			in case you need to restart it.

\item[dsaconfig:]	a configurator for Level-1 DSAs.
\end{describe}
In ISODE's \verb"ETCDIR" directory,
also typically \file{/usr/etc/},
there are a few programs and files of interest:
\begin{describe}\sloppy
\item[oidtable.at, oidtable.gen, oidtable.oc:]
			These define the attribute types,
			generic object identifiers,
			and object classes known to the system.
			(An object identifier is a method used to unambiguously
			encode, among other things,
			the names of attributes and object classes.)
			These files you never deal with unless they are
			accidently corrupted.
			
\item[dsaptailor:]	This is the run-time tailor file for the DUAs.
			You will configure this file initially and then
			probably leave it alone.


\item[isoaliases, isobjects, isoentities, isomacros, isoservices:]
			These are \\ %%%
			various databases used by the ISODE.
			These files you never deal with unless they are
			accidently corrupted.

\item[isologs:]		This script runs nightly under \man cron(8) to
			trim the ISODE log files, kept in ISODE's \verb"LOGDIR"
			directory,
			typically \file{/usr/tmp/}.
			This file you never deal with unless it is
			accidently corrupted.

\item[isotailor:]	This is the run-time tailor file for the ISODE.
			You will configure this file initially and then
			probably leave it alone.
\end{describe}
There are a few other things kept in a directory called \file{quipu/}
which resides in ISODE's \verb"ETCDIR" directory.
The \file{scripts/} directory contains various scripts which may be used for
maintenance purposes.
The \file{templates/} directory contains templates of entries.

\subsection	{Source Release}
You should retrieve a copy of the pilot software.
It is available via anonymous FTP from from host
\verb"uu.psi.com" (\verb"[136.161.128.3]"):
\begin{quote}\small\begin{verbatim}
% cd /usr/src/local/
% ftp uu.psi.com
Connected to uu.psi.com.
220 uu.psi.com FTP server ready
Name (uu.psi.com:user): anonymous
Password (uu.psi.com:anonymous): guest
331 Guest login ok, send ident as password.
230 Guest login ok, access restrictions apply.
ftp> binary
200 Type set to I.
ftp> cd pilot/src
ftp> get isode-pilot.tar.Z
ftp> quit
% uncompress < isode-pilot.tar.Z | tar xfp -
% rm isode-pilot.tar.Z
\end{verbatim}\end{quote}
Note that this is a interim release of the ISODE.
All of the changes are upwards compatible,
but you must run this version rather than the standard ISODE~5.0 release.

Now change to the top-level directory.
\begin{quote}\small\begin{verbatim}
% cd isode-5.9/
\end{verbatim}\end{quote}
The file \file{READ-ME} in this directory contains instructions on how to
configure, generate, and install the ISODE.

\[\fbox{\begin{tabular}{lp{0.8\textwidth}}
\bf NOTE:&	The source installation instructions are intended for
		sites which are not already running QUIPU.
		If you are,
		care should be taken to see that your existing Directory
		environment is not compromised.
		For example,
		the DSAs run under the pilot do not provide the
		``higher performance'' name service,
		nor is it intended that they support application entities
		other than DSAs.
\end{tabular}}\]

\subsubsection	{Configuration of the ISODE}
As described in the \file{READ-ME} file,
configuration is straight-forward.
Here's an example of how it's done on a Sun workstation running SunOS~3.5:
\begin{quote}\small\begin{verbatim}
% cp config/sunos3.h h/config.h
% cp config/sunos3.make config/CONFIG.make
% cp config/*.local support/
\end{verbatim}\end{quote}
For other systems,
consult the \file{READ-ME} file to see which templates to use.
For all systems,
take a look at
\[\begin{tabular}{l}
\file{h/config.h}\\
\file{config/CONFIG.make}\\
\end{tabular}\]
to see if things like \verb"BINDIR",
\verb"SBINDIR", \verb"ETCDIR", and \verb"LOGDIR" are set to your liking.

\subsubsection	{Generation of the ISODE}
Once properly configured,
generate both the ISODE and QUIPU using this command from the top-level
directory:
\begin{quote}\small\begin{verbatim}
% ./make once-only all all-quipu
\end{verbatim}\end{quote}
This takes a while,
so its best to redirect the output to a file and then go do something else for
a few hours:
\begin{quote}\small\begin{verbatim}
% ./make once-only all all-quipu >& make.out &
\end{verbatim}\end{quote}

If the make command finishes with a non-zero exit status,
or if you see a line with:
\begin{quote}\small\begin{verbatim}
*** Error code 1
\end{verbatim}\end{quote}
or perhaps
\begin{quote}\small\begin{verbatim}
Exit
\end{verbatim}\end{quote}
Then something has gone wrong.
Try and determine what the problem is.
If you are unable to find the problem,
drop a note to the Internet Mailbox
\begin{quote}\begin{verbatim}
Bug-ISODE@NISC.PSI.NET
\end{verbatim}\end{quote}
Your message should contain your two configuration files,
the output of the make command,
and a description of the hardware/software platform you are using.

Keep in mind that the ISODE runs ``flawlessly'' worldwide on many thousands of
machines running numerous variants of \unix/.
Failure to complete generation usually indicates a broken configuration file,
or a non-standard system
(i.e., one with a broken compiler or a different \pgm{make} command).

Now generate the additional commands needed for the pilot project:
\begin{quote}\small\begin{verbatim}
% (cd others/quipu; ./make pilot)
\end{verbatim}\end{quote}

\subsubsection	{Installation of the ISODE}\label{isode:install}
Once you are satisfied that the generation was successful,
become the super-user and install both the ISODE and QUIPU.
First make sure that the directories \verb"BINDIR",
\verb"SBINDIR",
\verb"ETCDIR",
\verb"LOGDIR",
 as defined in \file{config/CONFIG.make} exist.
If not, create them using \pgm{mkdir}.
The protection on the \verb"LOGDIR" directory should be 0777.

Next, install the programs, databases, and libraries:
\begin{quote}\small\begin{verbatim}
# ./make inst-all inst-quipu
# (cd others/quipu; ./make inst-pilot)
\end{verbatim}\end{quote}

Do {\bf not\/} perform any other installation instructions for QUIPU as
given in the \file{READ-ME} file.
The remainder of this chapter supercede those instructions.

If you are not planning on doing program development (probably a safe bet),
you can remove the various libraries and program development tools by using
\begin{quote}\small\begin{verbatim}
# ./make zap
\end{verbatim}\end{quote}
from the \file{isode-5.9/} directory as the super-user.

\section	{Starting the ISODE}
In order to aid debugging,
you should keep the ISODE system around in an operational state.
So,
you need to do a few more things:
\begin{enumerate}
\item	Verify that the file \file{/etc/rc.local} has an entry like this:
\begin{quote}\small\begin{verbatim}
if [ -f $(SBINDIR)tsapd ]; then
   $(SBINDIR)tsapd -t & (echo -n ' tsap') > /dev/console
fi
\end{verbatim}\end{quote}
in the section where the network servers are started
(e.g., where \man sendmail(8) is started).

\item	Verify that the file \file{/etc/services} has an entry like this:
\begin{quote}\small\begin{verbatim}
tsap    102/tcp
\end{verbatim}\end{quote}

\item	Verify that the file \file{/usr/lib/crontab} has an entry similar 
to this:
\begin{quote}\small\begin{verbatim}
0 4 * * * su daemon < $(SBINDIR)isologs
\end{verbatim}\end{quote}
or this
\begin{quote}\small\begin{verbatim}
0 4 * * * su daemon -c "/bin/sh $(SBINDIR)isologs"
\end{verbatim}\end{quote}
This runs a script to clean-up the logfiles at 4:00am each morning.

\item	Start the \man tsapd(8c) daemon by hand:
\begin{quote}\small\begin{verbatim}
# $(SBINDIR)tsapd -t >& /dev/null
\end{verbatim}\end{quote}
The daemon will automatically detach.
Instead,
if its diagnostic output is associated with a terminal,
then the daemon will not detach and will log informative messages to
the terminal.

\item	Now verify that the ISODE is installed and tailored correctly.
See Sections~\ref{isode:tailoring} and~\ref{isode:testing}.
\end{enumerate}

\section	{ISODE Tailoring}\label{isode:tailoring}
Before any DSAs or DUAs are installed,
the file \file{isotailor} in the ISODE \verb"ETCDIR" directory should be
properly configured. 

Typically source installations have ISODE's \verb"ETCDIR" and \verb"LOGDIR"
directories hard-wired in during configuration of the ISODE.
As such,
the source installation procedures do not install a \file{isotailor} file.
If a future need arises,
the \file{isotailor} file may be modified accordingly.
Initially it is recommended that you use a tailoring file which contains
no commands;
copy it from the \file{quipu/scripts/} area:
\begin{quote}\small\begin{verbatim}
# cp $(ETCDIR)quipu/scripts/isotailor.empty $(ETCDIR)isotailor
\end{verbatim}\end{quote}

\section	{Testing of the ISODE}\label{isode:testing}
The easiest way to test the ISODE system is to run
the \file{isode-test} script in the directory \file{quipu/scripts/}
under ISODE's \verb"ETCDIR" directory.
\begin{quote}\small\begin{verbatim}
% isode-test
\end{verbatim}\end{quote}
Prior to termination,
the script will indicate the number of errors encountered.
If an error occurs,
try and determine what the problem is.
If you are unable to find the problem,
drop a note to the Internet Mailbox \verb"Bug-ISODE@NISC.PSI.NET".
Your message should contain your two configuration files,
the output of \file{isode-test},
and a description of the hardware/software platform you are using.

\section	{Configuration of DUAs}\label{dua:configure}
The first step will be to have the \man dish(1c) program
try to connect to one of the Level-0 DSAs in the pilot project:
\begin{quote}\small\begin{verbatim}
% dish -c "Alpaca"
Welcome to Dish (DIrectory SHell)
Dish ->
\end{verbatim}\end{quote}
indicates that you are able to connect.
You might now try some simple commands.
Figure~\ref{dish-simple} shows a small interaction.
Chapter~4 of \volfive/ describes \pgm{dish} in considerable detail.
\tagfigure[tp]{CNF-1}{A Small DISH Interaction}{dish-simple}

\subsection	{DUA Failure}\label{dua:failure}
There are a few reasons why the DUA might fail.
First verify that the ISODE is operating correctly
(see Section~\ref{isode:testing} on page~\pageref{isode:testing}.)
If the ISODE is working,
then here is the most common error encountered:

If instead of the \verb"Dish ->" prompt,
you see:
\begin{quote}\small\begin{verbatim}
*** Service error : Unable to contact DSA ***
\end{verbatim}\end{quote}
then it is likely that there is some problem with the network,
or you have mistyped the name of the DSA.
If you see any other diagnostic,
contact a \camayoc/ for help.

Otherwise,
first check the spelling of the name you are providing.
If this is correct,
then look at the file \file{dsaptailor} in the ISODE \verb"ETCDIR" directory
for the line which starts with:
\begin{quote}\small\begin{verbatim}
dsa_address "Alpaca"
\end{verbatim}\end{quote}
You will find an Internet address and TCP port number on the rest of
the line.
\begin{enumerate}
\item	Use the \man ping(8c) utility to see if you can reach the IP address.

\item	If not,
	then try contacting one of the other DSAs,
	e.g., \verb"Fruit Bat",
	listed in the \file{dsaptailor} file.
	
	Repeat this process until \pgm{dish} successfully contacts a DSA.

\item	If \pgm{ping} can reach the IP address,
	then use \pgm{telnet} to try to connect to the TCP port.

\item	If you can connect,
	immediately close the connection.
	Something is drastically wrong since \pgm{telnet} can reach the TCP
	address of the DSA,
	but \pgm{dish} can't.
	Contact a \camayoc/ for assistance.

\item	If \pgm{telnet} reports an error,
	usually
\begin{quote}\small\begin{verbatim}
Connection refused
\end{verbatim}\end{quote}
	then this indicates that the DSA isn't listening at that TCP
	address.
	Either it is temporarily unavailable or it has moved to a new
	address.
	In either event,
	notify a \camayoc/ of the situation.
	Then go back to Step~2 above.
\end{enumerate}
If you exhaust all DSAs listed in the \file{dsaptailor} file,
contact a \camayoc/ and give up.

\section	{An Important Note}
The configuration instructions contained herein are for sites in the United
States.
Depending on the policies of the administrators for other national DMDs,
these instructions may, or may not, be useful. 
If your DSA is joining a DMD other than that of c=US,
then you should contact your national administrator to see what turn-key
procedures exist for joining your national DMD.

Note that the \man dsaconfig(8) program can be used to configure Level-1 DSAs
residing outside of the c=US DMD.
At present,
the current list of supported countries is:
\[\begin{tabular}{l}
AU\\ DE\\ ES\\ FI\\ GB\\ IS\\ NO\\ SE
\end{tabular}\]
In most cases,
for the countries listed above,
you can simply follow the directions below substituting your country code for
any occurrence of \verb"c=US".

In all cases,
for DSAs residing outside of the c=US DMD,
you should always contact your national administrator before proceeding.

\section	{Configuring your Level-1 DSA}
The steps towards configuring a Level-1 DSA are straight-forward.
For the text that follows,
any example that includes \verb"O_i" means that the name of your organization
should be substituted.
Similarly,
any occurrance of \verb"U_j" refers to the name of your organizational unit.

\subsection	{Choosing a DSA Name}
First, you must choose a name for your DSA.
It is a QUIPU convention that DSAs should be named after endangered
South American wildlife,
and that the entry for the DSA should contain a description of the animal
or plant in question.
Table~\ref{endangered-wildlife},
taken from \volfive/ of the ISODE User's Manual,
contains a partial list of these species.
You might also want to consult IUCN's {\em Red Book\/} \cite{IUCN.Mammal}.
\tagtable[tp]{CNF-1}{Endangered South American Wildlife}{endangered-wildlife}

For the purposes of the pilot project,
the name of each Level-1 DSA takes the form:
\begin{quote}\small\begin{verbatim}
c=US@cn=wildlife name
\end{verbatim}\end{quote}
So, go to Table~\ref{endangered-wildlife} and pick an unused name.
Alternately,
find a name which sounds plausible enough to fool a \camayoc/.
Prepend \verb"c=US@cn=",
and this is the Distinguished name of your Level-1 DSA.

To find out what names are already taken,
do the following:
\begin{quote}\small\begin{verbatim}
% dish -c "Alpaca"
Welcome to Dish (DIrectory SHell)
Dish -> search @c=US -filter objectClass=dsa -nosizelimit
\end{verbatim}\end{quote}
This will print out the list of names already in use.

Since some of the names in Table~\ref{endangered-wildlife} are long,
it is acceptable to use only part of the wildlife name as the common name
of your Level-1 DSA.
For example,
if the wildlife name is something like \verb"Southern Bearded Saki",
then it's probably okay to use \verb"Saki" as the name of the DSA,
e.g., \verb"c=US@cn=Saki".

Now decide upon a ``sanitized'' name that will be used for the
\unix/ directory which will contain the database for your Level-1 DSA.
Using all lower case letters and substituting hyphens for spaces is a good
rule.
So, if the name of your DSA is \verb"c=US@cn=Spectacled Bear",
then \file{spectacled-bear/} or just \file{bear/} would be fine.
For the purpose of discussion,
let's call this directory \file{wildlife/}.

The owner of the directory should be some maintenance account
(e.g., \verb"daemon") or \verb"root".
When the DSA executes,
it will set its user- and group-IDs based on the owner and group of this
directory.

\subsection	{Editing the DSA configuration file}
Go to the directory \file{quipu/} under ISODE's \verb"ETCDIR" directory:
\begin{quote}\small\begin{verbatim}
# cd $(ETCDIR)quipu/
\end{verbatim}\end{quote}
Now copy the configuration tempate file, e.g.,
\begin{quote}\small\begin{verbatim}
# cp wildlife.dsa spectacled-bear.dsa
# chmod 0600 spectacled-bear.dsa
\end{verbatim}\end{quote}
Now edit the new file to define the initial configuration for your Level-1
DSA.
Figure~\ref{dsaconfig} shows an example DSA configuration file.
\tagfigure[tp]{CNF-2}{Example DSA configuration file}{dsaconfig}

Please be sure to spell out the full name of your organization in the 
configuration file,
do not use an abbreviation (e.g., \verb"PSI").

Note that when defining information such as \verb"streetaddress",
\verb"telephone",
and \verb"description", you should use the values for your organization,
not for your own department.
If you like,
{\em after\/} your DSA is running,
you can edit \file{c=US/o=O\_i/EDB} to insert the corresponding
information for your own department.

Note that if the name of your organization is long than 30 characters,
then you will have to manually define a postal address in the configuration
file for \pgm{dsaconfig}.
The format is:
\begin{quote}\small\begin{verbatim}
postalAddress    "item1 $ item2 $ ... $ itemN"
\end{verbatim}\end{quote}
where each item is less-than or equal-to 30~characters in length and there are
no more than six items.

\subsection	{Running the DSA configuration program}
Once the file is edited,
run the DSA configuration program, \man dsaconfig(8):
\begin{quote}\small\begin{verbatim}
# $(SBINDIR)dsaconfig wildlife
\end{verbatim}\end{quote}
e.g.,
\begin{quote}\small\begin{verbatim}
# $(SBINDIR)dsaconfig spectacled-bear
\end{verbatim}\end{quote}
This will create the database directory and all requisite files.
Finally,
it will set the owner and group of the files appropriately.

It is necessary to explain a little bit about what \pgm{dsaconfig} does.
Way back in Section~\ref{edb:format} on page~\pageref{edb:format},
the ``EDB format'' of the database was described.
All that is now missing is an explanation how different nodes in the Directory
tree are structured.
This is done using the \unix/ directory hierarchy.

Starting at the ROOT,
at every level in the Directory tree for which your Level-1 DSA holds data,
there is a single file called \file{EDB}.
If an entry defined in an \file{EDB} file has children,
the \file{EDB} file for the children will be found in a subdirectory
whose name is the string encoded Relative Distinguished Name (RDN) of
the entry.

For example,
if a \file{EDB} file has an entry whose RDN is \verb"o=Computer Science",
and if the entry has children,
and if your Level-1 DSA stores this information locally,
then the information can be found in the file \file{o=Computer Science/EDB}
relative to the current directory.
Note that:
\begin{itemize}
\item	QUIPU is case sensitive when it looks for the subordinate \file{EDB}
	file~---~the case of the attribute type
	(e.g., \verb"o") is taken from the file \file{oidtable.at} in the
	ISODE \verb"ETCDIR" directory, 
	whilst the case of the attribute value is taken from the superior
	\file{EDB} file;
	and,	

\item	blanks are {\em actually\/} present in the sub-directory name.%
\footnote{An interesting example of the maxim {\em can implies shall\/}~---~because
the Berkeley \unix/ file system {\em can\/} let us have long directory names
with embedded spaces, we {\em shall\/} make use of that feature.}
\end{itemize}
Thus, for a Level-1 DSA, the initial database hierarchy looks like:
\begin{quote}\begin{tabular}{l}
\file{EDB}\\
\file{c=US/EDB}\\
\file{c=US/o=O\_i/EDB}\\
\file{c=US/o=O\_i/ou=U\_j/EDB}\\
\end{tabular}\end{quote}
So,
\pgm{dsaconfig} built these files for you.

\subsection	{Testing the stand-alone DSA}
The first thing to do is to start the DSA interactively so as to see
that it can successfully load its local EDB files.
So your Level-1 DSA is started thusly:
\begin{quote}\small\begin{verbatim}
# cd wildlife/
# $(SBINDIR)ros.quipu -t ./quiputailor &
\end{verbatim}\end{quote}
If your DSA is configured properly,
it will print out something like:
\begin{quote}\smaller\begin{verbatim}
DSA wildlife has started on '0101'H/Internet=130.117.128.3+17003
\end{verbatim}\end{quote}

\subsection	{DSA Failure}\label{dsa:failure}
Instead,
if the DSA prints out something like:
\begin{quote}\smaller\begin{verbatim}
FATAL ERROR: <<reason>>
\end{verbatim}\end{quote}
then something has gone wrong.

There are a few reasons why the DSA might fail.
First verify that the ISODE is operating correctly
(see Section~\ref{isode:testing} on page~\pageref{isode:testing}.)
Next, verify that the DUA is operating correctly
(see Section~\ref{dua:configure} on page~\pageref{dua:configure}.)

There are generally three causes of failure:
\begin{itemize}
\item	there is an attribute-related schema problem;

\item	there is a structure-related schema problem;
or,

\item	something else is running at the OSI presentation address of the DSA.
\end{itemize}
Since you are using files which have been automatically generated,
the first two errors are unlikely.
However,
they might occur due to spelling errors or catastrophic editing
on your part.
In order to diagnose the problem,
you will need to look at the tail of the file \file{dsap.log} in the
\file{wildlife/} directory.

\subsubsection	{Attribute Errors}
If you misspell the name of an attribute,
then the diagnostic will look something like this:
\begin{quote}\small\begin{verbatim}
line 14: unknown attribute type '<<bad-name>>'
File ...wildlife/c=US/o=O_i/EDB not loaded
\end{verbatim}\end{quote}
Alternately,
if you specify an illegal attribute for an object
(a violation of the object's schema),
then the diagnostic will look something like this:
\begin{quote}\small\begin{verbatim}
attribute '<<illegal-name>>' not allowed in the specified objectclass
Schema error in entry ending line 17...
File ...wildlife/c=US/o=O_i/EDB not loaded
\end{verbatim}\end{quote}
Conversely,
if you leave out an attribute which is required by an object:
\begin{quote}\small\begin{verbatim}
'Must' attribute missing '<<mandatory-name>>'
Schema error in entry ending line 16...
File ...wildlife/c=US/o=O_i/EDB not loaded
\end{verbatim}\end{quote}

\subsubsection	{Structure Errors}
If a subordinate object belongs to a class which is not permitted by
the schema of its immediate parent
(as determined by the parent's \verb"treeStructure" attribute, if any),
then the diagnostic looks something like this:
\begin{quote}\small\begin{verbatim}
Specified object class is not in the tree structure (schema) list
Schema error in entry ending line 16...
File ...wildlife/c=US/o=O_i/EDB not loaded
\end{verbatim}\end{quote}
Note that this means you have to look in the file \file{wildlife/c=US/EDB}
to find the \verb"treeStructure" attribute of the parent entry
and then compare that to the \verb"objectClass" attribute of the entry
in error.

\subsubsection	{Network Errors}
If the local EDB files are consistent,
but another process is already listening on the IP address and TCP
port for the DSA,
then your DSA will print:
\begin{quote}\small\begin{verbatim}
TNetListen: [Congestion at TSAP] start_tcp_server failed:
        Address already in use
Address: '0101'H/Internet=130.117.128.3+17003

FATAL ERROR: Couldn't start the DSA!!
\end{verbatim}\end{quote}
Perhaps another copy of this DSA is already running
or the TCP port in the entry for the DSA
(found in the file \file{wildlife/c=US/EDB/})
is wrong.
It is also possible that an early invocation of the DSA was ungracefully
terminated and left the TCP port in a \verb"FIN_WAIT_x" state.
Use the \pgm{netstat} program to find out.
If so,
reboot \unix/.

Alternately, you might see:
\begin{quote}\small\begin{verbatim}
TNetListen: [Congestion at TSAP] start_tcp_server failed:
        Can't assign requested address
Address: '0101'H/Internet=130.117.128.3+17003

FATAL ERROR: Couldn't start the DSA!!
\end{verbatim}\end{quote}
This indicates that you mis-identified the IP address of your Level-1 DSA in
your \pgm{dsaconfig} configuration file.

\subsection	{Editing the DUA tailoring file}
The file \file{dsaptailor} in the ISODE \verb"ETCDIR" directory contains
run-time tailoring information for the DUAs provided by the pilot software.
You will need to be the super-user to edit this file.

You should now edit the \file{dsaptailor} file so that your DUAs
will contact that DSA by default.
This is done by looking for these two lines in the \file{dsaptailor} file:
\begin{quote}\smaller\begin{verbatim}
# the level-1 DSA
#dsa_address "wildlife name"      '0101'H/Internet=aaa.bbb.ccc.ddd+port
\end{verbatim}\end{quote}
and then removing the \verb"`#'"-sign at the beginning of the second line.

Next,
you should also have your DUAs start, by default, in your portion of the
Directory tree.
This is done by looking for these two lines in the \file{dsaptailor} file:
\begin{quote}\small\begin{verbatim}
# local DIT position (where a DUA starts in the DIT)
#local_DIT   "c=US@O_i"
\end{verbatim}\end{quote}
and then removing the \verb"`#'"-sign at the beginning of the second line.

Section~12.2 of \volfive/ discusses the options available for run-time
tailoring.
Typically,
you will have no need of further editing this file.

Now run the \man dish(1c) program again,
telling it to connect to your Level-1 DSA.
\begin{quote}\small\begin{verbatim}
% dish -c "wildlife name"
Welcome to Dish (DIrectory SHell)
Dish ->
\end{verbatim}\end{quote}
indicates that the DUA connected to your Level-1 DSA.
Otherwise consult Section~\ref{dua:failure} on
page~\pageref{dua:failure} and try to debug the problem.

Now look around the Directory tree using \pgm{dish}:
\begin{quote}\small\begin{verbatim}
Dish -> list
\end{verbatim}\end{quote}
A good test to run is to try and bind to your own entry,
but to do so by dereferencing the alias for the Manager of your DMD:
\begin{quote}\small\begin{verbatim}
Dish -> bind -user "@c=US@o=O_i@cn=Manager"
Enter password for "@c=US@o=O_i@cn=Manager": 
Dish -> 
\end{verbatim}\end{quote}
Indicates that you are now bound to the directory as that DN.
Instead, if you see:
\begin{quote}\small\begin{verbatim}
Dish -> bind -user "@c=US@o=O_i@cn=Manager"
Enter password for "@c=US@o=O_i@cn=Manager": 
Security Error - check name and password
\end{verbatim}\end{quote}
then either you may have entered the DN or password wrong.
Try again.
If not,
or if you encounter some other problem,
contact a \camayoc/ for assistance.

\subsection	{Joining the Pilot DMD}
Once your DUA is configured to use your Level-1 DSA and your Level-1 DSA
appears to be functioning properly in stand-alone mode,
it is time for it to join the Pilot DMD.

First,
backup your EDB files:
\begin{quote}\small\begin{verbatim}
# cp EDB EDB.stand-alone
# cp c=US/EDB c=US/EDB.stand-alone
\end{verbatim}\end{quote}

\subsubsection	{Contact the c=US Manager}
Next,
contact the Manager of the \verb"c=US" portion of the DMD.
Use the \pgm{dish} program to get the necessary contact information, e.g.,
\begin{quote}\small\begin{verbatim}
% dish -c "Alpaca"
Welcome to Dish (DIrectory SHell)
Dish -> show @c=US@cn=Manager -fred
\end{verbatim}\end{quote}
If this fails,
contact a \camayoc/ for assistance.
Otherwise, e-mail the file \file{c=US/EDB} along with your phone number to the
Manager and await a phone call or message in return.
Leave your DSA running until you hear from the Manager.

The Manager will enter these into the MASTER \verb"c=US" EDB and then
try to connect to your DSA to test it out.
The Manager will also give you a ``time slot'' when your \file{nightly.sh}
script should run.
(This is a feeble attempt to prevent multiple systems from
mailing statistical information simultaneously each night.)

If everything looks fine,
the Manager will direct you to proceed.

\subsubsection	{Connecting In and Checking Out}
Restart your DSA and use \pgm{dish} to bind as the manager.
Next,
direct the DSA to update its slave EDBs:
\begin{quote}\small\begin{verbatim}
% dish -user "@c=US@o=O_i@cn=Manager" -simple
Welcome to Dish (DIrectory SHell)
Enter password for "@c=US@o=O_i@cn=Manager": 
Dish -> dsacontrol -slave
Scheduled
Dish -> quit
\end{verbatim}\end{quote}
The \verb"Scheduled" message means that the EDB updates have been scheduled
for execution.
The DSA will actually perform these in the background,
so it may be a few minutes (or longer, depending on DSA/network availability)
before the EDBs are updated.

To verify that your Level-1 DSA is updating its ROOT and \verb"c=US" EDB
files,
look for the files \file{EDB.bak} and \file{c=US/EDB.bak}
in the \file{quipu/wildlife/} directory.
If present,
these indicate that your Level-1 DSA was able to successfully update the EDBs.
If not,
look at the \file{dsap.log} log file and contact a \camayoc/ for
assistance.

Finally,
check that the global Directory tree knows about your DSA
(the \verb"c=US" Manager verified this, but you should check anyway).
Use the \pgm{dish} but connect to a Level-0 DSA:
\begin{quote}\small\begin{verbatim}
% dish -c "Alpaca"
Welcome to Dish (DIrectory SHell)
Dish -> moveto "@c=US@o=O_i"
Dish -> list
   ...
Dish -> list "ou=U_j"
   ...
\end{verbatim}\end{quote}
If this succeeds, the Level-0 DSAs know how to talk to your Level-1 DSA.

The other Level-1 DSAs won't know about your organization and Level-1 DSA
for another day or so.
(The Manager of the \verb"c=US" may cause a few Level-1 DSAs to immediately
reload the \verb"c=US" in order to facilitate things.)

\subsubsection	{More System Administration}
Once everything checks out,
its time to restart the DSA in the background.
First, edit the \file{quiputailor} file for your DSA by changing the
line
\begin{quote}\small\begin{verbatim}
update  off
\end{verbatim}\end{quote}
to
\begin{quote}\small\begin{verbatim}
update  on
\end{verbatim}\end{quote}
This tells your DSA to automatically update its SLAVE EDB files on a regular
basis.

Next, use \pgm{dish} to abort the DSA:
\begin{quote}\small\begin{verbatim}
% dish -user "@c=US@o=O_i@cn=Manager" -simple
Welcome to Dish (DIrectory SHell)
Enter password for "@c=US@o=O_i@cn=Manager": 
Dish -> dsacontrol -abort
*** Problem with DSA ***
Dish -> quit
\end{verbatim}\end{quote}
Now run the \file{startup.sh} script:
\begin{quote}\small\begin{verbatim}
% $(ETCDIR)quipu/wildlife/startup.sh
\end{verbatim}\end{quote}
Take a look at the log files it creates and once you're satisfied
that it is operational,
use \pgm{dish} one last time before considering things up and running.

Finally,
it's time for the last bit of system administration:
\begin{enumerate}
\item	Add an entry to the file \file{/etc/rc.local}:
\begin{quote}\smaller\begin{verbatim}
if [ -d $(ETCDIR)quipu/wildlife ]; then
   $(ETCDIR)quipu/wildlife/startup.sh & \
                                (echo -n ' wildlife') > /dev/console
fi
\end{verbatim}\end{quote}
in the section where the network servers are started.
If your \file{rc.local} file starts \man tsapd(8c),
then place this entry after the one which starts \pgm{tsapd}.

The next time your system reboots,
check to make sure that your DSA started.
On {\em some\/} systems running Sun's ``Yellow Pages'',
it has been observed that YP prevents the \pgm{ros.quipu} from starting.
No cause or cure is known for this.
The problem just seems to ``go away'' after a while.

\item	Based on the time that the \verb"c=US" manager gave you,
modify the \file{crontab} file according; e.g.,
\begin{quote}\small\begin{verbatim}
0 4 * * * $(ETCDIR)quipu/wildlife/nightly.sh
\end{verbatim}\end{quote}
If the directory database for the Level-1 DSA is owned by a user-ID other
than \verb"root" (e.g., \verb"daemon"),
then instead the line should look something like this:
\begin{quote}\smaller\begin{verbatim}
0 4 * * * su daemon -c "/bin/sh $(ETCDIR)quipu/wildlife/nightly.sh"
\end{verbatim}\end{quote}
or perhaps
this
\begin{quote}\small\begin{verbatim}
0 4 * * * su daemon -c "sh $(ETCDIR)quipu/wildlife/nightly.sh"
\end{verbatim}\end{quote}
Be sure that the ownership of the \file{nightly.sh} file matches that of the
userid that will be running under cron.
Verify that the user's shell, paths, and so on, are all correct as well.
\end{enumerate}

Congratulations!
Your Level-1 DSA has now joined the pilot DMD.
Further,
the Manager of \verb"c=US" has elevated you to the role of
junior \camayoc/ by adding your electronic mail address to the 
\verb"wpp-camayocs" mailing list.
Each \camayoc/ is expected to share experiences with others so as to
expand and advance understanding of running a white pages service using the
OSI Directory.

\section	{Configuring the White Pages}
The very last step you need do is to configure the white pages service which
resides on top of the OSI Directory.

You should now determine if a substantive portion of your user community does
not have easy access to a machine running the pilot project software.
If this is the case,
then you will probably want to make the white pages service available via the
network (either via the DA-service, or via TELNET, or by emulating WHOIS)
or via electronic mail.

\subsection	{White Pages via the DA-service}
Choose a machine in your local environment which is running the pilot project
software.
This machine will offer the white pages service via the DA-service.

On this machine,
add an entry to the file \file{/etc/rc.local}
\begin{quote}\smaller\begin{verbatim}
if [ -f $(SBINDIR)dad ]; then
   $(SBINDIR)dad & (echo -n ' da') > /dev/console
fi
\end{verbatim}\end{quote}
in the section where the network servers are started.
If your \file{rc.local} file starts a DSA,
then place this entry after the one which starts the DSA.

When this server starts,
it will export the DA-service.

If you want to enable picture support with the DA-service,
you will also need to install the \pgm{pbmtofax} program.
The sources to this program are in the directory \file{others/quipu/photo/}.
You will need the PBMPLUS package to compile faxtopbm.
See the \file{READ-ME} file in the \file{others/quipu/photo/} directory for
further information.

\subsubsection	{Fred-only hosts}
After bringing up the software for both the Directory and the White Pages,
you might wish to copy the executables to other machines,
just to make \pgm{fred} available.
To do this, you need the following files:
\begin{itemize}
\item	In the ISODE \verb"BINDIR" directory:
\begin{quote}\begin{tabular}{l}
\pgm{editentry}\\
\pgm{fred}
\end{tabular}\end{quote}

\item	In the ISODE \verb"SBINDIR" directory:
\begin{quote}\begin{tabular}{l}
\file{fredrc}\\
\file{isotailor}
\end{tabular}\end{quote}
Note that the \file{isotailor} file need not be present,
depending on the dynamic changes to your system configuration.
\end{itemize}
Now simply add the line
\begin{quote}\small\begin{verbatim}
set server hostname
\end{verbatim}\end{quote}
to the \file{fredrc} file,
where \verb"hostname" is the domain-name or IP-address of the machine running
\pgm{dad}.
The \pgm{fred} program will automatically connect to the server instead of
trying to run \pgm{dish} locally.

\subsection	{White Pages via TELNET}
Choose a machine in your local environment which is running the pilot project
software.
This machine will offer the white pages service via TELNET.

On this machine,
add a line to the \file{/etc/passwd} file for a user called \verb"fred"
\begin{quote}\smaller\begin{verbatim}
fred::30:30:Anonyous White Pages User:/:$(SBINDIR)fredsh
\end{verbatim}\end{quote}
where group-id \verb"30" is the \verb"guest".

\subsubsection	{Fred-only hosts}
After bringing up the software for both the Directory and the White Pages,
you might wish to copy the executables to other machines,
just to make \pgm{fred} available.
To do this, you need the following files:
\begin{itemize}
\item	In the ISODE \verb"BINDIR" directory:
\begin{quote}\begin{tabular}{l}
\pgm{dish}\\
\pgm{editentry}\\
\pgm{fred}
\end{tabular}\end{quote}

\item	In the ISODE \verb"SBINDIR" directory:
\begin{quote}\begin{tabular}{l}
\file{fredrc}\\
\file{dsaptailor}\\
\file{oidtable.*}\\
\file{isobjects}\\
\file{isologs}\\
\file{isomacros}\\
\file{isotailor}
\end{tabular}\end{quote}
Note that the \file{isotailor} file need not be present,
depending on the dynamic changes to your system configuration.

\item	In the ISODE \verb"SBINDIR" directory:
\begin{quote}\begin{tabular}{l}
\pgm{fredsh}\\
\pgm{in.whitepages}
\end{tabular}\end{quote}
Note that you need \pgm{in.whitepages} only if you plan on using this host to
offer the white pages via WHOIS emulation.
\end{itemize}

\subsection	{White Pages via WHOIS}\label{wp:whois}
Choose a machine in your local environment which is running the pilot project
software.
This machine will offer the white pages service via a network port offering an
emulation of the WHOIS service.

On this machine,
edit the file \file{/etc/services} so that it has an entry like this:
\begin{quote}\small\begin{verbatim}
whitepages 17005/tcp
\end{verbatim}\end{quote}

Next,
edit the file \file{/etc/servers} so that it has an entry like this:
\begin{quote}\small\begin{verbatim}
whitepages tcp  $(SBINDIR)in.whitepages
\end{verbatim}\end{quote}
Because most user interfaces to WHOIS,
e.g., \man whois(1c),
do not allow the user to specify a special port,
you should probably also add this line as well:
\begin{quote}\small\begin{verbatim}
whois   tcp     $(SBINDIR)in.whitepages
\end{verbatim}\end{quote}
If you already have a line for \verb"whois" in the \file{servers} file,
then you are already running a WHOIS service,
and you should {\bf not\/} add a second \verb"whois" line.
This machine is not a good choice for running the white pages via WHOIS
emulation.

Note that on newer systems derived from Berkeley \unix/,
\file{/etc/servers} is called \file{/etc/inetd.conf}.

\subsubsection	{The whitepages Command}
On those systems which are to access the white pages via the network and not
locally
(i.e., those systems which are not running the pilot project software),
you should determine how a user invokes the WHOIS service via the network.
For \unix/ systems,
you should provide a shell script like this:
\begin{quote}\small\begin{verbatim}
: run this script through /bin/sh

exec /usr/ucb/whois -h wp.psi.net "$*"
\end{verbatim}\end{quote}
where the name of a host running the pilot project software is substituted for
\verb"whitepages", e.g., \verb"wp.psi.net".
This host must have the files \file{/etc/services} and \file{/etc/servers}
edited as described above.

\subsection	{White Pages via mail}\label{wp:mail}
In addition,
you might want to make the whitepages available via electronic mail.
To do this,
choose a machine in your local environment which is running the pilot project
software.

On this machine,
edit the file \file{/usr/lib/aliases} so that it has an entry like this:
\begin{quote}\small\begin{verbatim}
whitepages: "|/usr/local/bin/fred -m"
\end{verbatim}\end{quote}
After editing the \file{aliases} file,
you will need to run the \pgm{newaliases} command as the super-user:
\begin{quote}\small\begin{verbatim}
# newaliases
\end{verbatim}\end{quote}
Also make sure that the \file{sendmail.cf} file lists \pgm{sh} and not
\pgm{csh} as the \verb"prog" mailer.

Once these changes are in place,
Whenever someone sends to the address \verb"whitepages" on this machine,
the subject line or body of the message will be interrogated for WHOIS-like
commands and the appropriate reply generated.

Note that on systems not running \man sendmail(8c),
a similar procedure is followed.
However,
it's up to you to figure out how to define aliases that run commands on
receipt of mail.

You should tell the local PostMaster about the white pages service.
In this fashion,
when the PostMaster get queries from other sites,
they can the sender of the message
(and the Postmaster at the sender's site)
about the service.

\subsection	{An Undocumented Feature}
In some environments,
in which the IP-address of a host is reasonably trustworthy,
it may be desirable to have an automatic means of mapping an IP-address into a
Distinguished Name.
\pgm{fred} provides such a mechanism under the following conditions:
\begin{enumerate}
\item	Create a file called \file{fredmap} in the ISODE \verb"ETCDIR"
	directory.

	The syntax of this file is a series of entries, one to a line.
	Each line contains:
    \begin{itemize}
    \item	a IP-address mask
		(in the familiar ``dot'' notation, e.g., \verb"255.255.255.0");

    \item	a IP-network address
		(in the familiar ``dot'' notation, e.g., \verb"192.52.180");

    \item	a Distinguished Name
		(e.g., \verb|"c=US@o=DMD@cn=anon"|);
		and,

    \item	the password to use when binding to that DN
		(e.g, \verb|"secret"|).
    \end{itemize}
Blanks and/or tab characters are used to separate items.
However, double-quotes may be used to prevent separation for items containing
embedded whitspace.
The sharp character (`\verb"#"') at the beginning of a line indicates a
commentary line.

This file should be mode \verb"0640", owned by user \verb"root" and
some previously unused group, e.g., \verb"wp".

\item	After installing \pgm{fred} with the
\begin{quote}\small\begin{verbatim}
# (cd others/quipu; ./make inst-pilot)
\end{verbatim}\end{quote}
command
(this was done way back in Section~\ref{isode:install} on
page~\pageref{isode:install}),
make the group of the binary the same as the \file{fredmap} file,
and change the mode to \verb"02755".

\item	When a local user invokes \pgm{fred}
\pgm{fred} looks-up the local IP-address,
reads the \file{fredmap} file looking for the first entry satisfying:
$$\mbox{(local IP-address)} \land \mbox{(IP-address mask)}\ \equiv\ 
\mbox{(IP-network address)}$$
Upon finding such an entry,
\pgm{fred} will bind to the Directory using the DN and password for that entry.

\item	When \pgm{fred} is accessed via the network
(e.g., with the \verb"in.whitepages" mechanism described in
Section~\ref{wp:whois} on page~\pageref{wp:whois}),
\pgm{fred} will automatically read the \file{fredmap} file,
looking for the first entry satisfying:
$$\mbox{(remote IP-address)} \land \mbox{(IP-address mask)}\ \equiv\ 
\mbox{(IP-network address)}$$
Upon finding such an entry,
\pgm{fred} will bind to the Directory using the DN and password for that entry.
\end{enumerate}
Of course,
the ordering of entries in the \file{fredmap} file is important!
It is suggested that the file begin with host-specific entries
(those with an IP-address mask of \verb"255.255.255.255"),
and then entries follow in decreasing order by the number of bits on in the
IP-address mask.

Although use of a ``default'' entry,
one which will match any IP-address,
is strongly discouraged,
it is possible to define such an entry.
This should be the very last entry in the \file{fredmap} file.
The format of this entry is simply:
\begin{quote}\small\begin{verbatim}
0.0.0.0 0.0.0.0    "some DN"    "some password"
\end{verbatim}\end{quote}

It should be noted that this scheme provides a convenient mechanism for
allowing ``local'' users to automatically be authorized as specific entries in
the local DMD.
By doing so,
this eases the administrative burden of assigning passwords to users in an
environment which contains read-only but, nonetheless, sensistive information.
The information is protected via access control,
e.g.,
\begin{quote}\smaller\begin{verbatim}
acl= prefix # c=US@o=DMD # read # attributes # <attribute-list>
acl= self # write # attributes # <attribute-list>
acl= others # none # attributes # <attribute-list>
\end{verbatim}\end{quote}
Note that for a writeable environment,
whilst the access control scheme is still used,
this simplification of locally anonymous DNs usually can not be made.

Note that under the current QUIPU access scheme,
the access control list above will prevent any user,
regardless of the DN they are bound as (including local users),
from searching on the values of the attributes named in
\verb"<attribute-list>".
As such, a less restrictive set of rules might be:
\begin{quote}\smaller\begin{verbatim}
acl= prefix # c=US@o=DMD # read # attributes # <attribute-list>
acl= self # write # attributes # <attribute-list>
acl= others # compare # attributes # <attribute-list>
\end{verbatim}\end{quote}
which will enable searching for all users.
However,
it will also allow any user to compare specific values against the attributes
named in \verb"<attribute-list>".

Of course,
this simplification is severely limited in that it works only for
IP-addresses, and further assumes that such addresses are trustworthy.
The trustworthiness depends entire upon the network environment in which
\pgm{fred} runs.
For example,
since IP-addresses can be readily forged,
one might rely on routers on the edge of the local environment to simply drop
IP datagrams with local source IP-addresses that originate from the outside
world.
In this case,
entries in the \file{fredmap} file containing local IP-addresses can be
thought to be ``reasonably'' secure against outside attack.
Of course,
nothing prevents a local host from forging a local IP-address.
If the trust policy is simply to distinguish between local and non-local
users,
then the \file{fredmap} mechanism is adequite.

Given all of these concerns,
this ``undocumented feature'' is a useful, albeit non-OSI, local mechanism.

\section	{Other Services}
There are some additional programs that you might wish to install for your
user community.

\subsection	{Faces}
When \pgm{fred} and \pgm{dish} display the entry for someone,
they check to see if there is a photograph associated with the user.
This is stored in facsimile format in the \verb"photo" attribute for the entry.
If a photograph is present,
then the \file{dsaptailor} file is consulted for directives
indicating how the picture should be displayed.
The most common entry is:
\begin{quote}\small\begin{verbatim}
photo    xterm Xphoto
\end{verbatim}\end{quote}
which says that if the user's terminal type is \verb"xterm" then the program
\pgm{Xphoto} should be run.

Although there are many different programs that can be used to display a
photograph,
only the \pgm{Xphoto} is supported by the sponsors of the pilot.

\subsubsection	{Xphoto}
To generate the \pgm{Xphoto} program,
run the following command:
\begin{quote}\small\begin{verbatim}
% (cd others/quipu/photo; ./make all Xphoto)
\end{verbatim}\end{quote}
To install the program,
become the super-user and do:
\begin{quote}\small\begin{verbatim}
# (cd others/quipu/photo; ./make inst-all)
\end{verbatim}\end{quote}
Next,
add this line at the end of the \file{dsaptailor} file:
\begin{quote}\small\begin{verbatim}
photo    xterm Xphoto
\end{verbatim}\end{quote}

\subsubsection	{xwho and xface}
There are two other programs which consult the Directory for people's faces.
These are \pgm{xwho}, who for X windows;
and,
\pgm{xface}, face agent for X windows.

To generate these programs,
first generate \pgm{Xphoto} as described above.
next:
\begin{quote}\small\begin{verbatim}
% (cd others/image; ./make all)
\end{verbatim}\end{quote}
To install the program,
become the super-user and do:
\begin{quote}\small\begin{verbatim}
# (cd others/image; ./make inst-all)
\end{verbatim}\end{quote}

The \pgm{xwho} program is free-standing,
but requires that the local system run the \man rwhod(8c) daemon.

The \pgm{xface} program runs, for each user, in the background.
When a mail reading program displays a message,
it sends a wake-up call to \pgm{xface} which must then try to find the picture
associated with the sender of the message.

At present,
only one mail reading system has been modified to communicate with
\pgm{xface},
the \MH/ system.
The file \file{others/image/mhlsbr.c} is a replacement for the
\file{uip/mhlsbr.c} in the \MH/ distribution.
Once \MH/ has been modified,
the manual entry for \man xface(1c) describes how the user tells \MH/ to
display faces.

Finally,
the file \pgm{others/image/READ-ME} describes how images may be collected and
placed in the Directory.

\subsection	{Mail Composition}
In addition to using the Directory when messages are being displayed, 
the White Pages may also be consulted when a message is being composed.

At present,
only one mail composition system has been modified to use the White Pages,
the \MH/ system.

The file \file{others/quipu/uips/fred/mh-patches} contains a patch set to \MH/
to use this facility along with instructions for apply the patch set.
\[\fbox{\begin{tabular}{lp{0.8\textwidth}}
\bf NOTE:&	These are very different from the patches supplied earlier
		in the file \file{MH-patches}.
		You should revert your \MH/ sources back to their
		original condition and then apply the new patches.
\end{tabular}}\]

Once installed,
users can specify a name by bracketing a White Pages query between
``\verb"<<"'' and ``\verb">>"'' using the user-friendly naming syntax,
e.g.,
\begin{quote}\small\begin{verbatim}
To: << rose, psi, us >>
\end{verbatim}\end{quote}
At the \whatnow/ prompt,
the user can say \verb"whom" to have the names expanded into addresses.
Alternately, the \verb"send" option can be used as well.
For each query appearing between ``\verb"<<"'' and ``\verb">>"'',
the DA-server will be asked to perform a White Pages resolution.
All matches are printed and the user is asked to select one.
If one is not selected,
the user returns to \whatnow/ level.

Note that expansion can occur only if \verb"whom" or \verb"send" is invoked
interactively. 
If the \verb"push" option is used instead,
then the expansion will fail because MH will be unable to query 
the user to select/confirm the right entry to use for the substitution.

Finally,
to enable this feature,
the user must add a line
\begin{quote}\small\begin{verbatim}
da-server: hostname
\end{verbatim}\end{quote}
to the \profile/,
where \verb"hostname" is the domain-name or IP-address of the machine running
\pgm{dad}.

\section	{Tell The Users!}
OK, everything is installed and running,
so it's time for you to tell your users about the system!
Print out copies of \thebook/ and the accompanying
{\em White Pages Quick Reference Sheet},
and start distributing them.
Also, encourage your users to subscribe to the 
\begin{quote}\begin{verbatim}
wpp-users@psi.com
\end{verbatim}\end{quote}
discussion group.
This is done by sending a message to the
\begin{quote}\begin{verbatim}
wpp-users-request@psi.com
\end{verbatim}\end{quote}
mailbox and asking to be added.

\subsection	{More New and Exciting Stuff}
New and exciting white pages applications are written from time to time.
You should contact \verb"wpp-manager" to see if there's something fun that you
can try out.
