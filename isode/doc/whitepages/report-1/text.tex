% -*- LaTeX -*-

\newpage
\section	{Introduction}
The need for a comprehensive white pages service increases in relation to the
size of the user community.
The early Internet was served well by a relatively simple facility.
Today's rapidly expanding Internet has outstripped the capabilities of the
existing system.
In order to meet new requirements,
NYSERNet, Inc.~is sponsoring a pilot project to provide white pages
service based on the OSI Directory.

A natural function of computer networks is to form the {\em infrastructure\/}
between the users they interconnect.
For example,
the electronic mail service offered by computer networks provides a means for
users to collaborate towards some common goal.
In the simplest cases,
this collaboration may be solely for the dissemination of information.
In other cases,
two users may work on joint research project,
using electronic mail as their primary means of communication.

Most network services are based on the implicit assumption that each user can
supply {\em infrastructural information} to 
facilitate information transfers through the network.
For example,
electronic mail services expect that an originator can supply 
addressing information 
for all the intended recipients.
It is not necessarily the task of electronic mail, per se,
to provide this infrastructural information to the user.

This model works fine in small environments,
particularly those where infrastructural information is not difficult to 
obtain and remember.
However,
the model does not scale well.
Consider the case when the membership of a network consists of hundreds of
thousands of users belonging to thousands of organizations.
It is no longer reasonable for a single user to provide this information,
except in very limited circumstances.
Further,
it is likely that some of the information changes frequently,
due to personnel and other resource movement.
The goal of a {\em white pages\/} service is to 
provide the necessary information, and to mask the complexity of the
infrastructural information.

\newpage
\section	{Approach}
The approach taken by the NYSERNet White Pages Pilot Project is
straight-forward, though somewhat controversial:
use the emerging OSI Directory standard as the basis for a white pages
service, and realize that technology on top of the Internet's TCP/IP-based
infrastructure.

The choice of OSI Directory as the cornerstone technology was not made
lightly: 
the richness of the service was evident,
and early prototype work had demonstrated that the underlying technology could
be realized.
Further,
it has often been noted that:
\begin{quote}\em
if one is going to crash and burn,
then it's probably best to be at the front of the airplane.
\end{quote}
Given the magnitude of the white pages problem in the Internet,
this analogy seems quite apt!

Hence,
under the approach taken by the NYSERNet White Pages Pilot,
to implement a white pages service,
three things are needed:
\begin{itemize}
\item	an OSI infrastructure;

\item	an implementation of the OSI Directory;
	and,

\item	a white pages abstraction,
	provided by an administrative discipline along with at least one user
	interface through which the service is accessed.
\end{itemize}
It is important to distinguish between the white pages {\em service\/} and the
OSI Directory {\em technology} as defined by the International Standards.
The white pages abstraction is provided both by a focused use of the
underlying OSI Directory technology (the administrative discipline)
and by special user interfaces.

\subsection	{OSI Infrastructure}
The OSI infrastructure is provided by the ISODE (pronounced {\em I-SO-D-E\/}),
or {\em ISO Development Environment}, a collection of library
routines and programs that implements an extensive set of OSI upper-layer
services.%
\footnote{It is an unfortunate historical coincidence that the first three
letters of the ISODE are ``ISO''.
This is not an acronym for the International Organization for Standardization,
but rather three letters which,
when pronounced in English,
produce a pleasing sound.}
In brief,
the ISODE implementation of the upper-layers of OSI is interesting in four
respects:
it provides extensive automatic tools for the development of OSI applications;
it supports OSI applications on top of both OSI and TCP/IP-based networks;
it provides a novel approach to the problems of OSI coexistence
and transition;
and,
it is openly available (non-proprietary).

The ISODE has been the subject of much joy and grief in both the Internet and
OSI communities.
It would be counter-productive to provide a more exacting description.
Interested readers might refer to either
the five-volume
{\em The ISO Development Environment: User's Manual},
for a detailed description,
or the professional reference entitled {\em The Open Book} by Rose and
published by Prentice-hall.

The one detail worth mentioning is that the ISODE implements
``the RFC1006 method'',
which is a {\em transport service convergence protocol} providing a 
near-perfect emulation of the OSI transport service on top of TCP/IP-based
networks.
Thus,
given an TCP/IP-based internet infrastucture,
the RFC1006 method makes that infrastructure appear as a ``native'' OSI
environment,
in which OSI applications require no modifications in order to execute.

\subsection	{OSI Directory}
The ISODE implementation of the OSI Directory,
QUIPU
was implemented and is currently maintained by the University College London.
QUIPU is a complete implementation of the OSI Directory,
based on the {\oldstyle 1988\/} ISO standards and CCITT recommendations.

The QUIPU Directory User Agent (DUA) can be used directly at the programmatic
level,
or exported from a interface process called \pgm{dish}~---~the DIrectory SHell.
The \pgm{dish} process is available available either as a monolithic \unix/
program containing an input-interpreter with several commands,
or as several \unix/ programs, each implementing one of those commands.

The QUIPU Directory System Agent (DSA) is memory-based:
it uses the native \unix/ file-system to provide stable-storage between
reboots, but otherwise maintains all data in program memory.
As might be expected,
providing that the DSA avoids paging,
execution of the lookup and search facilities of the Directory can be realized
in a timely fashion.
Naturally, when an update operation occurs,
the copy on disk is updated and a journal entry written before the update is
acknowledged.
The disk copy is stored in a textual format to facilitate examination.
As this copy is read only once~---~when the DSA starts,
typically when \unix/ goes multi-user.
The cost of such a strategy is believed to be relatively small if properly
implemented and tuned.

The DSA supports both Directory distributed operations (DSA-DSA) and the
Directory Abstract Service (DUA-DSA),
along with the full range of standard Directory attributes.
Further,
a large number of other attributes have been defined,
both to facilitate experimentation and support the white pages service.
Most notably these attributes were necessary to support automatic use of
replication of information in the Directory.

\subsection	{White Pages Abstraction}
The white pages abstraction is the layer of conceptualization which resides
above the OSI Directory service.
Its purpose is to hide the complexity of the underlying technology and provide
a user-friendly service.
There are two parts to the service:
an {\em administrative discipline}, and a {\em user interface}.

\subsubsection	{Administrative Discipline}
The OSI Directory leaves several issues to be decided by the implementor and
service provider.
These form the the administrative discipline.

At the highest level, the key issue is that each entry in the white pages
corresponds to an information object in the OSI Directory.
Since the OSI Directory uses {\em Distinguished Names\/} to uniquely identify
information objects,
the straight-forward approach is to use Directory Distinguished Names as names
in the white pages.
This mapping considerly simplifies the complexity of providing the white pages
service with the OSI Directory.
However,
since Distinguished Names are length and unwieldy,
the user interface must provide effective mechanisms for managing these names
for the user.

A second issue is to focus the scope of the project on persons and
organizations.
Although the underlying Directory implementation is capable of managing the
entire range of permissible information objects,
the user interface focuses on:
\begin{quote}\small\begin{tabular}{l}
organizations\\ organizational units\\ organizational roles (e.g., ``Chair of
the Department'')\\ localities (for those individuals not belonging to an
organization)\\ persons
\end{tabular}\end{quote}
However,
some modest work has been done as a part of the pilot in supporting other
objects,
such as networks, hosts, application processes, and document lookup.
Nonetheless,
the pilot project is emphasizing providing excellent support for persons and
organizations,
e.g., by defining additional attributes which are useful.
For example, users can store a
facsimile image of themselves as their \verb"photo" attribute.

The administrative discipline also deals with issues such as caching, and
replication.
As these mechanisms were severely enhanced after initial deployment,
they are discussed later on.

The most significant work in terms of the administrative discipline was the
writing of a one-hundred page {\em Administration Guide} explaining OSI
Directory concepts,
along with installation and maintenance procedures,
to \unix/ system administrators.
This document was well-received by the community of pilot administrators.
In addition,
a turn-key configuration program,
\pgm{dsaconfig},
was written to automatically generate the initial Directory configuration for
a participating organization.
This, too, was well-received.

\subsubsection	{User Interface}
Building the user interface consists of two tasks:
selecting the appropriate interface paradigm,
and mapping that paradigm to the Directory service.

The paradigm selected is based on an earlier Internet nameservice called WHOIS.
This style of interaction was chosen for two reasons:
\begin{itemize}
\item	experience with WHOIS since {\oldstyle 1982\/} has shown the syntax to
be well-liked by the user community;%
\footnote{It is beyond the scope of this report to speculate if the success of
the WHOIS syntax is due to a lack of competing nameservices in the Internet.}
and,

\item	by using a similar syntax in the interface,
the problem of training the user community is greatly reduced.
\end{itemize}
The user interface to the white pages service is called \pgm{fred}.%
\footnote{In tried-and-true \unix/ style,
\pgm{fred} stands for FRont-End to Dish.
The \verb"dish" program was mentioned earlier as the primary means for
exporting the DUA interface to the \unix/ shell.}

Although the program has several commands,
only one is used for finding things in the white pages:
the \verb"whois" command,
which has syntax analogous to the WHOIS command.
For each \verb"whois" command,
\pgm{fred} constructs one or more Directory operations and then has \pgm{dish}
execute those operations.

The \verb"whois" syntax in \pgm{fred} has been extended from that of the
earlier WHOIS service,
in order to provide focused searching at the organizational level.
For example, the command:
\begin{quote}\small\begin{verbatim}
fred> whois rose
\end{verbatim}\end{quote}
directs \pgm{fred} to find information about something called \verb"rose" in
the default searching area.
Initially,
this results in a single Directory operation,
textually described as:
\begin{quote}\small\begin{verbatim}
search
    -nosizelimit -timelimit 300
    -subtree -filter "o=*rose* | ou=*rose* | l=*rose* | cn=*rose*"
\end{verbatim}\end{quote}
which performs an entire subtree search of the default area,
looking for any entry matching the filter.
For each match,
a Directory read operation is performed and the resulting information
displayed accordingly.
As might be imagined,
more efficient searches can be performed if the user tells \pgm{fred} that a
person is being searched for.

For a second example, the command
\begin{quote}\small\begin{verbatim}
fred> whois rose -org nyser
\end{verbatim}\end{quote}
directs \pgm{fred} to find information about something called \verb"rose"
associated with some organization called \verb"nyser".
Initially this results in this Directory operation:
\begin{quote}\small\begin{verbatim}
search
    -dontdereferencealias
    -singlelevel -filter "o=*nyser* & objectClass=organization"
    "@c=US"
\end{verbatim}\end{quote}
which performs a single-level search in the United States portion of the
Directory,
looking for any entry matching the filter.
For each matching organization,
another Directory search operation takes place,
similar to that of the first example,
but anchored in that organization.

The \pgm{fred} program also supports a mailbox specification for searching,
and performs a yellow pages-style search accordingly.%
\footnote{In general,
the distinction between white pages and yellow pages is poorly understood.
A white pages search implies that searching occurs on some part of an object's
name,
whilst a yellow pages search implies that searching occurs on any of an
object's attributes.
Since, in the OSI Directory,
an object's name is one of its attributes,
these definitions are problematic at best.}

In addition to \pgm{fred},
two simple X~windows programs were written to interface to the white pages and
the Directory:
\begin{itemize}
\item	\pgm{xwho}, an X~windows version of the \pgm{rwho} program that
	displays the faces of people logged in on the local network,
	by using the Directory to retrieve their \verb"photo" attribute;
	and,

\item	\pgm{xface}, an X~windows program that displays the face of the person
	who sent the mail message being read, by first using the Directory to
	perform an inverse-mapping on the originator's mailbox address, and
	then retrieving the \verb"photo" attribute.
\end{itemize}
Finally,
the popular \MH/ message handling system was modified to invoke \pgm{fred} to
provide name resolution when sending mail messages.

\newpage
\section	{First Milestone: Numbers}
Internally,
work began on the NYSERNet White Pages Pilot Project in
mid-May, {\oldstyle 1989}.
After defining and implementing the white pages abstraction,
the pilot began offering service in July, {\oldstyle 1989}.
The software supporting the pilot was based on ISODE~5.2(beta),
a stable, but bug-rich, version of the software.
By the end of the three-month mark,
28~Internet sites were participating
(half of which where NYSERNet members),
and there were approximately 98K entries in the Directory.

\subsection	{Interop '89}
At the Interop$^{\mbox{\tiny TM}}$ trade-show and exhibition in
October, {\oldstyle 1989},
in the NYSERNet booth,
the white pages made its first public debut on the show floor.
This was particularly exciting as the floor network had Internet connectivity.

\subsection	{International Participation}
Although NYSERNet is running the US~Directory Management Domain (DMD) as a
part of the pilot,
one Canadian site wished to participate as well.
Since a volunteer for running the CA~DMD was not forthcoming,
this site was temporarily placed under the US~DMD.

\newpage
\section	{Second Milestone: Software}
The next three months (October through December, {\oldstyle 1989\/}),
were spent on two tasks: fixing implementation problems and hardening the code.
This resulted in ISODE~5.9(frozen).

\subsection	{Reliability}
Experience with the ISODE~5.2(beta) version of the software led to the motto:
\begin{quote}\em
it's nearly as good as bind, but not nearly as fast$\ldots$
\end{quote}
which was a fairly accurate assessment.
The software,
when running,
would act correctly.
However,
it would frequently crash.

Thus, one major activity was to simply track down the myriad of bugs exercised
by placing the software in operational use.
It should be noted that this phenomenon is true of any complex system when
first fielded.
In the process of tracking down problems,
several performance improvements were made.
For example,
the program memory storage requirement for each entry was reduced by
approximately half.

However,
two major logic problems existed:
DSAs would occasionally lock-up during synchronous operations,
and DSAs would not use any intelligence when distributing
operations to other DSAs.

\subsection	{Asynchrony}
The asynchrony problem was traced to two areas:
\begin{itemize}
\item	the method used by QUIPU for replication of portions of the Directory
	Information Tree (DIT) was synchronous,
	resulting in the DSA blocking for potentially long (or infinite)
	periods of time;
	and,

\item	the underlying ISODE operations were only partially asynchronous;
	in particular, whilst connection establishment and data reception were
	non-blocking, connection release and data transmission were
	synchronous.%
	\footnote{Actually,
	connection establishment would actually lock-up if two DSAs
	simultaneously tried to associate with each other.}
\end{itemize}
Both problems were relative straight-forward to fix,
once the critical areas of code were identified.

\subsection	{Distribution}
When an operation cannot be satisfied locally,
a QUIPU DSA will generate,
based on knowledge information in the Directory,
a list of DSAs which either master or shadow the desired information.
The distribution problem was simple in that the QUIPU DSA did not
keep track of its previous associations with DSAs in order to judge the
``responsiveness'' or ``reliability''.
Its choice was random,
which almost always led to problematic interactions
(e.g., ``broken'' DSAs frequently appeared at the front of the list).

The new software now sorts the list based on several heuristics:
whether an association is currently open to that DSA;
how long ago the DSA was known to be operating responsively;
how long ago the DSA was known to be operating reliably;
and,
how ``close'' the DSA is.
Closeness is determined from a tailorable list of preferred DSAs defined by
the DSA's administrator,
and by the ``distance'' between the Distinguished Names of the two DSAs in the
DIT.

In addition,
all of the key parameters dealing with replication and caching are now
tailorable.
At present,
cached entries are removed after 6~hours,
and shadow copies of information are checked for refresh every 24~hours.

\subsection	{User Interface}
The \pgm{fred} program proved spectacularly uncontroversial,
primarily due to its ease of use.
In addition to the usual lot of random bug-fixes,
two small changes were made during this period:
\begin{itemize}
\item	In all cases,
	the old software would issue a Directory read operation for all of an
	entry's attributes when displaying an entry.
	Of course,
	the attributes would be cached in the DUA so that subsequent re-display
	would not require another Directory read operation.

	However,
	when searching results in multiple matches,
	the default action is to display a one-line summary of each matching
	entry.
	This one-line summary contains the value of only two or three
	attributes of the entry.
	In this circumstance,
	it is wasteful and slow to issue a Directory read operation for all
	the attributes,
	since the user will probably never display all of the entries matched.
	Hence, when issuing the Directory search operation,
	\pgm{fred} asks that those three key attributes be returned for each
	matching entry.
	Thus,
	when multiple matches occur,
	no further Directory operations need be issued.

\item	In any event,
	the old software would display the one-line summaries in the same
	order as the entries returned by the Directory,
	which was essentially unordered.
	At present,
	\pgm{fred} name-collates the entries to provide a sorted output.
\end{itemize}

\subsection	{An Update and Interpolation}
After three months of intensive work,
ISODE~5.9(frozen) was released,
and our motto revised:
\begin{quote}\em
it's as good as bind, and nearly as fast$\ldots$
\end{quote}

On 19 December, {\oldstyle 1989},
there were at least 84~DSAs in the global Directory pilot,
79~of those were running the QUIPU software,
and 31~of those were running something very close to ISODE~5.9(frozen).
Those 31~DSAs mastered 64462~entries,
for an average of 2079~entries/DSA.
If this number is indicative,
then, on that day,
the size of the global DIT was on the order of 175K~entries.

In the US portion of the DIT,
the allocation of sites and DSAs looked like this:
\[\begin{tabular}{|l|c|c|c|}
\hline
\multicolumn{1}{|c|}{\null}&
		\multicolumn{3}{c|}{\bf Type of DSA}\\
\cline{2-4}
\bf Type of Institution&
		local&	remote&	non-NYSER\\
\hline
University&	8&	3&	7\\
Corporate&	1&	1&	6\\
Non-profit&	1&	0&	1\\
Government&	0&	0&	3\\
\cline{2-4}
\multicolumn{1}{|r|}{total:}&
		10&	4&	17\\
\hline
\end{tabular}\]
where
\[\begin{tabular}{rp{3.0in}}
local&	refers to a NYSERNet site running its own DSA\\
remote&	refers to a NYSERNet site with a DSA being run remotely\\
non-NYSER&
	refers to an Internet site outside of NYSERNet running its own DSA
\end{tabular}\]

\newpage
\section	{Direction}
Given the initial success of the pilot and the stability and robustness of the
new software,
the current direction for the NYSERNet White Pages Pilot is to emphasize
growth.
The value of a comprehensive white pages service increases in relation to the
size of the user community.

\subsection	{Software}
Work on the software is most likely going to lapse into maintenance mode,
as no problems remain in the ISODE~5.9(frozen) distribution.
As new problems arise,
they will be dealt with.

\subsubsection	{Platforms}
At present,
the ISODE and QUIPU are supported only on \unix/-based hosts.
Although retargeting the entire package for other platforms may be
prohibitively expensive,
the interprocess communications mechanism between \pgm{fred} and \pgm{dish} is
sufficiently general to permit these programs to be distributed across a
local area network
(e.g., a TCP connection is used as the IPC,
even when both processes are resident on the same host).

\subsubsection	{Interfaces}
The most interesting interface that might be developed is probably one based on
X~windows which provides a hyper-textual interface to the white pages.
This paradigm appears to have much promise.

\subsection	{(Teutonic) Discipline}
In terms of the administrative discipline,
there are two areas to be addressed for the next milestone.

\subsubsection	{Replication}
Although individual QUIPU DSAs are now robust and reliable,
and distributed operations are now sensible,
transient network outages may still prevent information from being available.
The solution, of course, is to use more replication in the system.
As such,
white pages administrators will be encouraged to team with each other in order
to provide shadow replication of their organizational DMDs.

\subsubsection	{International Participation}
In the next milestone,
NYSERNet will be working with the University of Toronto to provide a tight
interaction between the US~and CA~DMDs.
In particular,
a new locality for North America will be jointly administered.
This locality contains Directory aliases to organizations in both the US~and
CA~DMDs.
With the introduction of this locality,
users of the white pages will be able to automatically search both DMDs when
looking for information about organizations.

\subsection	{On the Far Horizon}
The NYSERNet White Pages Pilot Project is scheduled to finish at the end of
May, {\oldstyle 1990}.
At this time,
a determination must be made as to the viability of the service.
If the service is judged useful and maintainable,
then two issues must be addressed:
the level of supported required to offer the service,
along with additional hierarchy in the US~DMD.

\subsubsection	{The Need for Support}
At present,
implementation, maintenance, and support for the white pages abstraction along
with the operational pilot is done with 0.75FTE.
As the number of sites joining the white pages increases,
even with tools to semi-automate administration procedures,
the load will be too great.
A larger infrastructure will be needed.

\subsubsection	{Three-level DMD Scheme}
The two-level DMD scheme used by the pilot,
in which all organizations are placed directly under the~US,
is unscalable.
A three-level scheme,
in which organizations operating with a single state,
are placed under the DMD for that state,
must be put into effect.

\newpage
\section	{Documents}
As of this writing,
five documents have been produced:
\begin{itemize}
\item	{\em An Introduction to a NYSERNet White Pages Pilot Project},
	by Rose and Schoffstall,
	12~pages.

This introduces the basic notion of a white pages service and outlines
the goals and milestones of the project.

\item	{\em NYSERNet White Pages Pilot: Administrator's Guide},
	by Rose,
	104~pages.

This is the authoritative tome which introduces the white pages service and
OSI Directory,
describes how to configure and install the service,
and finally discusses maintenance issues.

This is a ``living'' document.

\item	{\em NYSERNet White Pages Pilot: User's Handbook},
	by Rose,
	38~pages.

This describes the pilot project from a user's perspective
and provides operational reference for the use of the white pages service.
In particular,
the white pages user interface, \pgm{fred}, is fully described.

There is also a two-page {\em White Pages Quick Reference Sheet},
which summarizes the basic commands given to \pgm{fred}.

\item	{\em An Implementation of a White Pages Service},
	by Rose,
	28~pages.

This describes the technology underlying the NYSERNet White Pages Pilot
Project.

Submitted to the IEEE {\em Journal on Selected Areas in Communications}.

\item	{\em NYSERNet White Pages Pilot Project},
	by Rose,
	39~images.

An accompanying presentation to the JSAC paper.
\end{itemize}
In addition,
independent of the work on the NYSERNet White Pages Pilot Project,
four other documents have been produced by the ISODE/QUIPU effort which
are germane to white pages:
\begin{itemize}
\item	{\em The ISO Development Environment: User's Manual, Volume 5: QUIPU},
	by Kille, Robbins, Rose, and Turland,
	290~pages.

\item	{\em Directory Navigation in the QUIPU X.500 System},
	by Barker and Robbins,
	15~pages.

\item	{\em An interim approach to use of Network Addresses},
	by Kille,
	10~pages.

\item	{\em A string encoding of Presentation Address},
	by Kille,
	5~pages.
\end{itemize}

\newpage
\section	{Acknowledgements}
The NYSERNet White Pages Pilot Project builds on the work of many others:
\begin{itemize}
\item	At the core,
	the ISODE provides the programmatic infrastructure for OSI
	applications.
	So many people have contributed to the ISODE that it is presumptuous
	to attempt to list them.

\item	The QUIPU directory from University College London,
	designed by Stephen E.~Kille and programmed primarily by
	Colin J.~Robbins, form the heart of the pilot's functionality.

\item	Although NYSERNet has devoted considerable resources to the hardening
	of QUIPU,
	it should be noted that UCL has been responsible for the vast majority
	of improvements and fixes.

\item	The white pages abstraction was designed primarily at NYSERNet with
	the help of several experts: Kille, Robbins, and Einar Stefferud.

\item	Geoffrey S.~Goodfellow was kind enough to be the alpha tester for the
	white pages abstraction.

\item	The white pages administrators have been patient as the software
	twitches.

\item	Finally,
	it should be noted that none of this would be possible if it
	weren't for the excellent end-to-end services provided by the Internet
	suite of protocols (namely the TCP and the IP).
\end{itemize}
