% run this through LaTeX with the appropriate wrapper

\section	{The OSI Directory}
The OSI Directory is designed to provide
for the management of information objects.
The Directory's representation of an information object,
typically called an {\em entry},
contains information about a person, a place, an organization, etc.
Each entry consists of one or more attributes.

Each attribute consists of a type,
indicating what kind of attribute it is,
and one or more values
(one of which is termed the {\em distinguished value\/}).
Attribute values are structured using a data definition language
called Abstract Syntax Notation One (ASN.1).
This structuring is important.
With structuring,
different programs using the Directory will interpret information in the same
way.
In addition,
the Directory will perform type-checking on the values in order
to keep things consistent.

\subsection	{Naming}
One of the attributes of an entry is particularly special:
it is referred to as the {\em Relative Distinguished Name\/} (RDN) of
the entry.
The RDN is formed by taking the name of the attribute and its
distinguished value.
For example,
if the attribute in question was called \verb"countryName" and it had
a distinguished value of \verb"US", then we might say that the RDN
for the entry was \verb"countryName=US".
Of course,
this is strictly a ``user-friendly'' notation:
the Directory uses a concise binary format for representing an RDN.
Fortunately,
the pilot project software allows simple textual strings to be used in their
place and converts back and forth accordingly.

In the OSI Directory,
information is primarily organized according to a hierarchical tree
structure.
The top of the tree is termed the {\em root},
and has no explicit name.
To find the name of an object,
termed its {\em Distinguished Name\/} (DN),
one concatenates the RDNs found when traversing the tree by starting
at the root and proceeding directly to the object's entry.

For purposes of discussion,
we write a Distinguished Name as an ordered series of RDNs separated by
an \verb"`@'"-sign with the most significant RDN appearing at the left;
e.g.,
\begin{quote}\small\begin{verbatim}
countryName=US@organizationName=Performance Systems International
\end{verbatim}\end{quote}
refers to an entry with an RDN of
\verb"organizationName=Performance Systems International"
whose parent has an RDN of \verb"countryName=US".
In turn,
this parent entry is an immediate child of the root.

To avoid any potential ambiguity when using an interface to the Directory
such as \man fred(1c) or \man dish(1c),
one prefixes a \verb"`@'"-sign to a string when referring to a fully
qualified Distinguished Name;
e.g.,
\begin{quote}\small\begin{verbatim}
@countryName=US@organizationName=Performance Systems International
\end{verbatim}\end{quote}
always refers to the same entry regardless of context.
Note that this is a convention only for interface programs such as these.

As a rule,
unless searching,
text before the \verb"`='"-sign is not case sensitive,
neither is text after the \verb"`='"-sign.

The Directory itself is distributed,
being composed of {\em Directory System Agents\/} (DSAs).
A group of DSAs under a common administration is responsible for a portion of
the tree,
termed a {\em Directory Management Domain\/} (DMD).
When a user wishes to access the Directory,
a {\em Directory User Agent\/} (DUA) is invoked.
This DUA contacts a DSA and issues requests.
The DSA may (or may not) have the information locally available.
If not,
a decision has to be made:
either the DSA can contact another DSA to get the information
(this is called {\em chaining\/}); or,
the DSA can tell the DUA to contact another DSA directly
(this is called {\em referral\/}).

In short,
the DSAs provide mechanisms for traversing the tree and manipulating the
information contained therein.

In the context of the pilot project,
each participating organization runs its own DMD for that organization.
This usually consists of a single DSA containing information on that
organization,
with some of this information being replicated on additional DSAs.

\section	{Ramifications on the White Pages Service}
In order to appreciate the ``feel'' of the white pages service,
it is instructive to compare the white pages to an existing facility.

You might be familiar with an older facility called WHOIS.
This uses a centralized database to keep track of information on various
people, networks, hosts, and so on.
This facility has proven useful for many years.
Only recently,
with the explosive growth of the Internet,
has the WHOIS mechanism become unworkable.

\subsection	{Unique Identification of Users}
Each entry in the WHOIS database is identified by a unique key,
called a {\em handle}.
This is (typically) a short string such as \verb"MTR".
For a community many orders of magnitude larger than the current entries in
the WHOIS database,
a handle must contain some structure.
This makes it possible to delegate naming authority to different organizations
and thus de-centralize management of the white pages service.

In the white pages service,
a Directory Distinguished Name is used to uniquely identify a person.
Thus,
while \verb"MTR" might be enough to identify someone named ``Marshall Rose''
in the WHOIS database,
the DN
\begin{quote}\small\begin{verbatim}
c=US
    @o=Performance Systems International
    @ou=Research and Development
    @ou=Mountain View
    @cn=Marshall Rose
\end{verbatim}\end{quote}
serves as the handle for the same person in the white pages service.
(That's progress for you!)

Of course,
you don't {\em really\/} have to type all that information in.
The user interfaces provided with the pilot project allow you to manage very
short strings to refer to these DNs.
These interfaces also provide a means for incrementally building up a DN from
scratch.

Actually,
the handle in the example above is probably a somewhat longer than the average.
In terms of the pilot project,
a handle probably looks closer to:
\begin{quote}\small\begin{verbatim}
c=US@o=Organization Name@ou=Unit Name@cn=FirstName LastName
\end{verbatim}\end{quote}
While this is still a far cry from a simple three or four letter acronym,
it is the price one pays for using a service designed to meet the needs of a
global (or galactic) population.

\subsection	{Searching the White Pages}
When the WHOIS database is searched,
{\em all\/} entries in the database are examined for a match.
Since the current size of the WHOIS database is estimated at roughly 70,000
entries,
this is an appropriate strategy.

Unfortunately,
the potential size of the white pages is many orders of magnitude larger than
that of the WHOIS database.
As such,
the information contained in the white pages is distributed.
This makes management of the information a shared responsibility,
and has the potential to address organization-specific privacy concerns.

Thus,
when the white pages service is invoked,
searches are performed relative to a particular {\em area}.
This is similar to the White Pages of the telephone system~---~there are
several white pages, one for each particular geographical area.
As such,
before you can find someone's entry in the white pages,
you have to already know the area in which they are listed.

The default area is the portion of the Directory corresponding to your own
organization.
Of course,
if you specify a user's handle (a fully-qualified Distinguished Name),
this bypasses the default area and goes directly to the portion of
the Directory containing the desired entry.

Usually,
when you are trying to find an entry,
you have only partial information.
For example,
you might know parts of the name of the organization and the person you're
looking for.
In this case,
it is natural to use an iterative process to find the information you desire.
You begin by finding the organization(s) likely to contain the entry,
you then initiate a search starting at that area.

Having said that,
I'll let you in on a little secret:
in addition to people,
organizations and organizational units also have entries in the Directory.
As such,
searching an {\em area\/} is nothing more than starting a search at a
particular node at the tree.
Thus,
you might look for the organization starting at the \verb"@c=US" node.
In order to make searching easy,
the pilot project requires that all organizations be listed directly under
this node.
How the subtree is structured beyond that is an organization-specific matter,
although the pilot project provides various guidelines.

Thus,
to find someone,
you look for the organization name in the \verb"@c=US" area.
This should give you back a single entry in the Directory,
perhaps two or three at the most.
You then look for that person in the area corresponding to that entry.
To make this easier,
the white pages user interface, \pgm{fred}, has a special command syntax
which directs it to find out the names of the likely organizations and then
search each one for the person you're looking for, automatically!

Of course,
if you have the cycles and network bandwidth to burn,
in theory there is nothing to stop you from simply going to the top of the
tree and searching for the person.
However,
this is {\em very\/} resource-expensive,
particularly in terms of time.
Since time is probably the most valuable resource you have,
it is worth it to issue two commands which complete quickly,
rather than one command which may take hours.

There are  two user interfaces provided with the pilot software.
With the ``simple'' one,
you follow this two-step process.
With the ``complicated'' one,
you can form {\em arbitrarily\/} complex queries to the Directory.
Thus,
if you  want to type just one command and don't mind typing a bit more,
you can still have an optimized search.
Both of these interfaces will be introduced in due course.

\subsection	{Structure of Information}
In addition to a handle,
an entry in the WHOIS database consists of a {\em type},
which indicates what kind of user is recorded by the entry
(e.g., a person);
and,
several {\em fields}, each containing a textual description.

For example,
an entry for a person might look like:
\begin{quote}\small\begin{verbatim}
Rose, Marshall T. (MTR)    mrose@psi.com
    PSI, Inc.
    PSI California Office
    POB 391776
    Mountain View, CA  94039
    (415) 961-3380
\end{verbatim}\end{quote}
The first line contains both the handle and all fields available for searching.
Here,
the handle is \verb"MTR",
and there are two fields available for searching:
a name and a mailbox.
The remainder of the entry is a textual annotation.

Because the Directory must accommodate many kinds of access from various users
and programs.
It is important that the information contained therein be highly structured.
As noted earlier,
this allows universal understanding of the information,
and hence consistent interpretation.
Fortunately,
most of the information is represented by textual strings.

It is important to remember however that all information associated with an
entry is contained in an attribute.
This attribute has a type,
describing both its syntax and semantics.
For example,
the \verb"surName" attribute of a person has a textual string syntax and
semantics corresponding to someone's last name.

How the information associated with an entry is displayed to you
is {\em strictly\/} a function of the interface you use when talking to the
Directory.
The Directory will enforce all of the syntactic constraints associated with
the attributes,
but only the users of the Directory can assign meaning to the attribute
semantics.

With this in mind,
here's an entry associated with a person,
as it might be displayed by a user interface:
\begin{quote}\small\begin{verbatim}
Marshall Rose (3)            mrose@psi.com
    aka: Marshall T. Rose

Principal Scientist
PSI, Inc.
    PSI California Office
    POB 391776
    Mountain View, CA  94039

Telephone:    +1 415-961-3380
FAX:          +1 415-961-3282

Mailbox information:
    Internet: mrose@psi.com
    UUCP:     uupsi!mrose

Principal Implementor of the ISO Development Environment

Name:          Marshall Rose, Mountain View, ...
\end{verbatim}\end{quote}
Of course,
there are dozens of possible ways that this information could have been
displayed.
Or {\em not\/} displayed~---~for example,
there are other attributes which the interface may not care (or be able) to
display,
such as access control information, passwords, and so on.

Appendix~\ref{person:attributes} on page~\pageref{person:attributes} lists all
of the attributes which may be present for a person participating in the pilot
project.
