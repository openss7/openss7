% -*- LaTeX -*-

\input lcustom

\documentstyle[10pt]{article}

\advance\textwidth by1in
\advance\oddsidemargin by-0.5in
\advance\evensidemargin by-0.5in

\pagestyle{empty}

\begin{document}

\begin{center}\LARGE\bf
WHITE PAGES\\ 
\large\bf QUICK REFERENCE SHEET
\end{center}

\section*	{Query Syntax}
\begin{verbatim}
whois input-field [record-type] [area-designator] [output-control]
    input-field is one of:
        name NAME               e.g., surname "smith", or fullname "john smith"
        handle HANDLE           e.g., handle @c=US@cn=Manager
        mailbox LOCAL@DOMAIN    e.g., mailbox postmaster@nisc.psi.net

    record-type is one of:
        person or -title NAME   e.g., -title scientist
        organization
        unit (a division under an organization)
        role (a role within an organization)
        locality
        dsa (white pages server)

    area-designator is one of:
        -org NAME       e.g., -org psi
        -unit NAME      e.g., -unit engineering
        -locality NAME  e.g., -locality rensselaer
        -area HANDLE    e.g., -area "@c=US@o=Performance Systems International"
            and may be followed by -geo HANDLE, e.g., -geo @c=GB

    output-control is any of:
        expand     - give a detailed listing, followed by children
        subdisplay - give a one-listing listing, followed by children
        full       - give a detailed listing, even on ambiguous matches
        summary    - give a one-line listing, even on unique matches
\end{verbatim}
\vfill\noindent
The most common usage:
\begin{quote}\small\begin{verbatim}
fred> whois schoffstall -org psi
Trying @c=US@o=Performance Systems International ...
3 matches found.
  2. Marvin Schoffstall                         marv@psi.com
  3. Martin Schoffstall                         schoff@psi.com
  4. Steve Schoffstall                          steve@psi.com

fred> whois !3
Martin Schoffstall (3)                          schoff@psi.com
    ...
\end{verbatim}\end{quote}

\newpage

\section*	{Examples of usage}
\begin{verbatim}
fred> whois "smith"
\end{verbatim}
\begin{quote}
looks for any entries with this name in the default area
(choice of matching on the entry's surname or fullname is based on the value
of the \verb"namesearch" variable)
\end{quote}

\begin{verbatim}
fred> whois surname "smith"
\end{verbatim}
\begin{quote}
looks for any entries with this name in the default area
\end{quote}

\begin{verbatim}
fred> whois fullname "john smith"
\end{verbatim}
\begin{quote}
looks for any entries with this name in the default area
\end{quote}

\begin{verbatim}
fred> whois "smith" -org psi
fred> whois smith@psi
\end{verbatim}
\begin{quote}
looks for any entries with this name in any organization containing ``psi''
\end{quote}

\begin{verbatim}
fred> whois "smith" -area "@c=US@o=Performance Systems International"
\end{verbatim}
\begin{quote}
could be used if you already know the "area" that the user resides in
\end{quote}

\begin{verbatim}
fred> whois "smith" -area 17
\end{verbatim}
\begin{quote}
could be used if an alias were already established for this area
\end{quote}

\begin{verbatim}
fred> whois @c=US@cn=Manager
\end{verbatim}
\begin{quote}
looks for the entry with this distinguished name (handle)
\end{quote}

\begin{verbatim}
fred> whois !7
\end{verbatim}
\begin{quote}
could be used if an alias were already established for this entry
\end{quote}

\begin{verbatim}
fred> whois smith@psi.com
\end{verbatim}
\begin{quote}
looks for any entries with the given mailbox
\end{quote}

\begin{verbatim}
fred> whois -title operator
\end{verbatim}
\begin{quote}
looks for any entries who are operators
\end{quote}

\begin{verbatim}
fred> whois -org *
\end{verbatim}
\begin{quote}
reports on all registered organizations
\end{quote}

\begin{verbatim}
fred> whois -org * -geo @c=GB
\end{verbatim}
\begin{quote}
reports on all registered organizations under @c=GB
\end{quote}

\end{document}
