% run this through SLiTeX with the appropriate wrapper

\dotopic	{HISTORY AND INTRODUCTION}

\begin{bwslide}
\part*	{OUTLINE}\bf

\begin{description}
\item[PART I:]		HISTORY OF THE ISODE

\item[PART II:]		DESIGN AND IMPLEMENTATION

\item[PART III:]	UNDERLYING ABSTRACTIONS
\end{description}
\end{bwslide}


\begin{bwslide}
\part	{HISTORY OF THE ISODE}\bf

\begin{nrtc}
\item	EXPERIMENT WITH OSI UPPER LAYERS

\item	EXPLORE PROTOCOL TRANSITION ISSUES

\item	WHAT'S HAPPENED SINCE THEN

\item	ISODE AVAILABILITY
\end{nrtc}
\end{bwslide}


\begin{bwslide}
\ctitle	{WHAT IS THE ISODE?}

\begin{nrtc}
\item	THE ISO DEVELOPMENT ENVIRONMENT

\item	AN OPENLY AVAILABLE IMPLEMENATION OF THE UPPER LAYERS OF OSI?

\item	A BASIS FOR THE TRANSITION TO OSI?

\item	AN EXERCISE IN MEGA-CODING?

\item	A PLAYGROUND FOR ``THE PIED-PIPER OF OSI''?
\end{nrtc}
\end{bwslide}


\begin{bwslide}
\ctitle	{NORTHROP RESEARCH AND\\ TECHNOLOGY CENTER:\\ JANUARY, 1986}

\begin{nrtc}
\item	THE AUTOMATION SCIENCES LABORATORY WAS INTERESTED IN SOLVING CERTAIN
	PROBLEMS IN THE FACTORY AUTOMATION AREA

\item	AN ``AFTER-HOURS'' PROJECT WAS STARTED TO LOOK INTO THE APPLICABILITY
	OF MIXING OSI AND TCP/IP TECHNOLOGIES
\end{nrtc}
\end{bwslide}


\begin{bwslide}
\ctitle	{MOTIVATION\\ (WHY ISODE?)}

\begin{nrtc}
\item	EXPERIMENT WITH OSI UPPER LAYERS

\item	EXPLORE PROTOCOL TRANSITION ISSUES
\end{nrtc}
\end{bwslide}


\begin{bwslide}
\part*	{EXPERIMENT WITH OSI UPPER LAYERS}\bf

\begin{nrtc}
\item	THE UPPER LAYERS OF OSI APPEARED TO BE A RICH PLAYGROUND

\item	WE WANTED TO SEE HOW USEFUL THE UPPER LAYERS REALLY WERE
\end{nrtc}
\end{bwslide}


\begin{bwslide}
\ctitle	{(OBLIGATORY SLIDE SHOWING)\\ THE 7--LAYER STACK}

\vskip.5in
\diagram[p]{figureH-1}
\end{bwslide}


\begin{bwslide}
\ctitle	{THE UPPER-LAYER ARCHITECTURE}

\begin{nrtc}
\item	PROVIDES A COMMON FRAMEWORK FOR APPLICATIONS \emph{DESIGNERS}

\item	TO MAKE BUILDING APPLICATIONS EASIER FOR \emph{IMPLEMENTORS}

\item	IN THEORY, THIS YIELDS BETTER APPLICATIONS FOR THE \emph{USERS}
\end{nrtc}
\end{bwslide}


\begin{bwslide}
\ctitle	{PAYOFF OF THE INFRASTRUCTURE}

\begin{nrtc}
\item	RAPID PROTOTYPING OF APPLICATIONS

\item	FOCUS DESIGN/DEVELOPMENT ON \emph{APPLICATION} ISSUES,
	RATHER THAN UNDERLYING ISSUES

\item	MINIMIZE LONG-TERM INVESTMENT OF SOFTWARE-RELATED INVESTMENT
    \begin{nrtc}
    \item	(AT EXPENSE OF START-UP COST)
    \end{nrtc}
\end{nrtc}
\end{bwslide}


\begin{bwslide}
\ctitle	{THE UPPER-LAYER ARCHITECTURE}

%\vskip.5in
\diagram[p]{figureH-2}
\end{bwslide}


\begin{bwslide}
\ctitle	{EXAMPLE OF APPLICATION LAYER USE}

\vskip.5in
\diagram[p]{figureH-3}
\end{bwslide}


\begin{bwslide}
\ctitle {INFRASTRUCTURE REVISITED}

\begin{nrtc}
\item	STRENGTHS
    \begin{nrtc}
    \item	COMMON INFRASTRUCTURE

    \item	MORE TECHNICALLY COMPREHENSIVE
    \end{nrtc}

\item	WEAKNESSES
    \begin{nrtc}
    \item	COMPLEXITY

    \item	POLITICAL POLARIZATION

    \item	LACK OF PRACTICAL EXPERIENCE
    \end{nrtc}
\end{nrtc}
\end{bwslide}

\begin{bwslide}
\ctitle	{ONLY ONE LITTLE PROBLEM$\ldots$}

\begin{nrtc}
\item	HOW TO RUN THE OSI UPPER-LAYERS IN A TCP/IP-BASED NETWORK?

\item	A SOLUTION IS OFFERED BY LAYERING
    \begin{nrtc}
    \item	THE OSI TRANSPORT \underline{SERVICE} IS VERY SIMPLE

    \item	CAN WE PROVIDE AN EMULATION OF THAT SERVICE USING TCP?
    \end{nrtc}
\end{nrtc}
\end{bwslide}


\begin{bwslide}
\part*	{EMULATION OF OSI END-TO-END SERVICES}\bf

\begin{nrtc}
\item	IS IT POSSIBLE TO PROVIDE OSI APPLICATIONS IN NON-OSI NETWORKS?

\item	A SOLUTION IS OFFERED BY LAYERING
    \begin{nrtc}
    \item	THE OSI TRANSPORT \underline{SERVICE} IS VERY SIMPLE
    \end{nrtc}

\item	CAN WE OFFER THE OSI TRANSPORT SERVICE USING NON-OSI PROTOCOLS?
\end{nrtc}
\end{bwslide}


\begin{bwslide}
\ctitle	{SERVICE EMULATOR AT TRANSPORT}

\vskip.5in
\diagram[p]{figureH-4}
\end{bwslide}


\begin{bwslide}
\ctitle	{APPROACH:\\ TRANSPORT SERVICE\\ CONVERGENCE PROTOCOL}

\begin{nrtc}
\item	USE THE CONNECTION-ORIENTED TRANSPORT SERVICE PROVIDED BY
	THE NON-OSI PROTOCOL SUITE

\item	DEFINE A ``TSCP'' WHICH SMOOTHS OVER THE DIFFERENCES IN THE SERVICES
	OFFERED
    \begin{nrtc}
    \item	IN PRACTICE, THESE ARE QUITE SMALL
    \end{nrtc}

\item	FOR EXAMPLE, THE RFC1006 METHOD DEFINES A TSCP FOR TCP/IP NETWORKS
\end{nrtc}
\end{bwslide}


\begin{bwslide}
\ctitle	{OSI TRANSPORT SERVICES\\ ON TOP OF THE TCP}

\vskip.25in
\diagram[p]{figureH-5}
\end{bwslide}


\begin{bwslide}
\ctitle	{EXPLORE PROTOCOL TRANSITION ISSUES}

\begin{nrtc}
\item	DOES THIS APPROACH MAKE TRANSITION OR COEXISTENCE EASIER?

\item	BEYOND THE SCOPE OF THIS TALK
    \begin{nrtc}
    \item	CHECK OUT ``PRACTICAL PERSPECTIVES ON OSI NETWORKING''
    \end{nrtc}
\end{nrtc}
\end{bwslide}


\begin{bwslide}
\part*	{WHAT'S HAPPENED\\ SINCE THEN}\bf

\begin{nrtc}
\item	BASIS OF OSI-POSIX PROJECT

\item	MULTIPLE VENDOR ADOPTION

\item	FOUNDATION OF SEVERAL OSI PILOTS
    \begin{nrtc}
    \item	e.g., PSI WHITE PAGES PILOT PROJECT
    \end{nrtc}
\end{nrtc}
\end{bwslide}


\begin{bwslide}
\ctitle	{THE LIST OF IMMORTALS\\ (TOO NUMEROUS TO MENTION, BUT $\ldots$)}

\begin{nrtc}
\item	CASS, LUKASIK, STEFFERUD

\item	KILLE, ONIONS, ROBBINS, TURLAND, WALTON, ROE, MICHAELSON, TITCOMBE,
	EASTERBROOK, PAVEL, WORSLEY, SCOTT, WILDER, PAVLOU, KNIGHT,
	HEIN\"{A}NEN, MOORE

\item	NORTHROP, WOLLONGONG, NYSERNet, PSI, UCL, UNott, MITRE
\end{nrtc}
\end{bwslide}


\begin{bwslide}
\ctitle	{OPENLY AVAILABLE}

\begin{nrtc}
\item	NO LICENSES, CONTRACTS, OR LETTERS

\item	SMALL DISTRIBUTION/HANDLING FEE

\item	DO \emph{WHATEVER} YOU WANT, BUT

\item	HOLD-HARMLESS ALL CONTRIBUTORS

\item	AND WE SIGN NOTHING!
\end{nrtc}
\end{bwslide}


\begin{bwslide}
\ctitle	{FOR BETTER OR WORSE}

\begin{nrtc}
\item	THE ISODE IS NOW THE DE FACTO REFERENCE IMPLEMENTATION FOR OSI
\end{nrtc}
\end{bwslide}


\begin{bwslide}
\ctitle	{WHAT THE ISODE IS NOT}

\begin{nrtc}
\item	OSI OVER TCP
    \begin{nrtc}
    \item	ISODE RUNS OVER PURE OSI TRANSPORT STACKS
		(e.g., TP4/CLNP, TP0/X.25, etc.)
    \end{nrtc}

\item	RELATED TO THE INTERNATIONAL ORGANIZATION FOR STANDARDIZATION
    \begin{nrtc}
    \item	DO NOT EXPAND THE ``ISO'' IN ISODE
    \end{nrtc}
\end{nrtc}
\end{bwslide}


\input	isode


\begin{bwslide}
\part*	{PORTABILITY PHILOSOPHY}\bf

\begin{nrtc}
\item	USE \unix/ AND C

\item	NO KERNEL MODIFICATIONS
\end{nrtc}
\end{bwslide}


\begin{bwslide}
\ctitle	{LANGUAGES AND OPERATING SYSTEMS}

\begin{nrtc}
\item	KNOWN PORTS FOR BERKELEY \unix/ (4BSD):
    \begin{nrtc}
    \item	VAXen, SUNs, Pyramids, RTs, etc.
    \end{nrtc}

\item	KNOWN PORTS FOR AT\&T \unix/ (SVR2 and SVR3):
    \begin{nrtc}
    \item	SGI, 3Bs, 386s, RT (AIX)
    \end{nrtc}

\item	RELY ON ``COMPATIBILITY'' LIBRARY TO SMOOTH OVER DIFFERENCES

\item	ASSUMES A DECENT C COMPILER (EITHER K\&R OR ANSI)
    \begin{nrtc}
    \item	NO C++ SUPPORT
    \end{nrtc}
\end{nrtc}
\end{bwslide}


\begin{bwslide}
\ctitle	{OTHER PLATFORMS}

\begin{nrtc}
\item	OTHER ORGANIZATIONS HAVE DONE ``PORTS'' TO OTHER TARGETS
    \begin{nrtc}
    \item	DOS

    \item	MVS
    \end{nrtc}
    BUT NONE HAVE EVER BEEN INCORPORATED!

\item	INCLUSION BACK INTO MAIN RELEASE IS HINDERED BECAUSE
	CHANGES MUST BE MODULO TO \emph{CURRENT} RELEASE

\item	AND, PEOPLE KEEP ASKING ABOUT A PORT TO VMS
\end{nrtc}
\end{bwslide}


\begin{bwslide}
\ctitle	{COMMON GOTCHA's}

\begin{nrtc}
\item	C COMPILER/LOADER MUST HANDLE NAMES THAT ARE
    \begin{nrtc}
    \item	LONGER THAN 32~CHARACTERS

    \item	NON-UNIQUE IN FIRST 32~CHARS

    \item	PRE-PROCESSOR MUST HANDLE OVER 3000 DEFINES
    \end{nrtc}

\item	NEED \verb"fork" AND \verb"exec" TO DO DYNAMIC SERVERS

\item	NEED SYNCHRONOUS MULTIPLEXING (e.g., \verb"select")

\item	NEED DYNAMIC MEMORY ALLOCATION (e.g., \verb"malloc")

\item	NEED \verb"make" WITH
    \begin{nrtc}
    \item	\verb"-f file" CAPABILITY

    \item	\verb"sh expr" CAPABILITY
    \end{nrtc}
\end{nrtc}
\end{bwslide}


\begin{bwslide}
\part*	{SOFTWARE ARCHITECTURE}\bf

\begin{nrtc}
\item	SEVERAL APPLICATION ENVIRONMENTS

\item	ALL INTERFACES ARE EXPOSED
\end{nrtc}
\end{bwslide}


\begin{bwslide}
\ctitle	{THE APPLICATION ENVIRONMENT}

\vskip.5in
\diagram[p]{figureH-6}
\end{bwslide}


\begin{bwslide}
\ctitle	{AN ALTERNATE ENVIRONMENT:\\ LIGHTWEIGHT PRESENTATION}

\vskip.5in
\diagram[p]{figureH-7}
\end{bwslide}


\begin{bwslide}
\ctitle	{AN ALTERNATE ENVIRONMENT:\\ MHS ARCHITECTURE (c.~1984)}

\vskip.5in
\diagram[p]{figureH-8}
\end{bwslide}


\begin{bwslide}
\ctitle	{CALLING CONVENTIONS}

\begin{nrtc}
\item	MODEL EACH LAYER AS LIBRARY OF C ROUTINES

\item	MODEL EACH USER-INITIATED PRIMITIVE AS A PROCEDURE CALL

\item	MODEL EACH PROVIDER-INITIATED PRIMITIVE AS A STRUCTURE
\end{nrtc}
\end{bwslide}


\begin{bwslide}
\ctitle	{INTERFACE STYLE}

\begin{nrtc}
\item	SERVICES ARE BASED ON A SYNCHRONOUS INTERFACE

\item	CONFIRMED SERVICES USUALLY BLOCK WAITING FOR RESPONSE

\item	HOWEVER, A NON-BLOCKING MODE CAN BE EMPLOYED BY USING
    \begin{nrtc}
    \item	A SUPPLIED ROUTINE WHICH IS A SELECT-PLUS

    \item	WITH SUPPORT FOR QUEUED WRITES
    \end{nrtc}

\item	THERE IS ALSO (A NOT SO-WELL TESTED) ASYNCHRONOUS INTERFACE
\end{nrtc}
\end{bwslide}


\begin{bwslide}
\ctitle	{EXAMPLE: CONFIRMED SERVICE}

\vskip.5in
\diagram[p]{figureH-13}
\end{bwslide}


\begin{bwslide}
\ctitle	{EXAMPLE: CONFIRMED SERVICE (cont.)}

\vskip.5in
\diagram[p]{figureH-14}
\end{bwslide}


\begin{bwslide}
\ctitle	{CONFIRMED SERVICE}\small

\hrule\vskip.15in
\begin{tgrind}\scriptsize
\let\linebox=\relax
\def\_{\ifstring{\char'137}\else\underline{\ }\fi}
\input figureH-9\relax
\end{tgrind}
\end{bwslide}


\begin{bwslide}
\ctitle	{WAIT FOR SOMETHING TO HAPPEN}\small

\hrule\vskip.15in
\begin{tgrind}\scriptsize
\let\linebox=\relax
\def\_{\ifstring{\char'137}\else\underline{\ }\fi}
\input figureH-10\relax
\end{tgrind}
\end{bwslide}


\begin{bwslide}
\part*	{ROADMAP}\bf

\begin{nrtc}
\item	AND NOW, A VERY QUICK TOUR!
\end{nrtc}
\end{bwslide}


\begin{bwslide}
\ctitle	{ROADMAP (cont.)}

\begin{nrtc}
\item	\verb"acsap/" ACSE AND ALS LIBRARY

\item	\verb"compat/" COMPATIBLITY LIBRARY

\item	\verb"config/" CONFIGURATION TEMPLATES 

\item	\verb"dirent/" GETDENTS COMPATILBILITY

\item	\verb"doc/" DOCUMENT SET, INCLUDING:
    \begin{nrtc}
    \item	\verb"manual/" USER'S MANUAL

    \item	\verb"whitepages/" WPP DOCUMENT SET
    \end{nrtc}

\item	\verb"dsap/" X.500 LIBRARY

\item	\verb"ftam/" FTAM LIBRARY

\item	\verb"ftam-ftp/" FTAM TO FTP GATEWAY

\item	\verb"ftam2/" FTAM USER/FILESTORE PROGRAMS

\item	\verb"ftp-ftam/" FTP TO FTAM GATEWAY
\end{nrtc}
\end{bwslide}


\begin{bwslide}
\ctitle	{ROADMAP (cont.)}

\begin{nrtc}
\item	\verb"h/" INCLUDE FILES

\item	\verb"imisc/" EXAMPLE APPLICATION

\item	\verb"others/" ``UNSUPPORTED'' ADDITIONS, INCLUDING:
    \begin{nrtc}
    \item	\verb"mosy/" MANAGED OBJECT COMPILER

    \item	\verb"tsbridge/" TS-BRIDGE
    \end{nrtc}

\item	\verb"pepy/" ORIGINAL ASN.1 COMPILERS

\item	\verb"pepsy/" PRIMARY ASN.1 COMPILER

\item	\verb"psap/" PRESENTATION ABSTRACTIONS

\item	\verb"psap2/" PRESENTATION LIBRARY

\item	\verb"psap2-lpp/" RFC1085 LIBRARY
\end{nrtc}
\end{bwslide}


\begin{bwslide}
\ctitle	{ROADMAP (cont.)}

\begin{nrtc}
\item	\verb"quipu/" DSA/DUA PROGRAMS

\item	\verb"rosap/" ROSE LIBRARY

\item	\verb"rosy/" RO-COMPILER

\item	\verb"rtsap/" RTSE LIBRARY

\item	\verb"snmp/" SNMP LIBRARY AND PROGRAMS

\item	\verb"ssap/" SESSION LIBRARY

\item	\verb"support/" DAEMONS, DATABASES, etc.

\item	\verb"tsap/" TRANSPORT LIBRARY

\item	\verb"vt/" VIRTUAL TERMINAL PROGRAMS
\end{nrtc}
\end{bwslide}


\begin{bwslide}
\ctitle	{FOR FURTHER READING}

\begin{nrtc}
\item	Volume 1: Application Services\\
    \begin{nrtc}
    \item	\verb"acsap" \verb"psap" \verb"rosap" \verb"rtsap" 
    \end{nrtc}

\item	Volume 2: Underlying Services\\
    \begin{nrtc}
    \item	\verb"psap2" \verb"psap2-lpp" \verb"ssap" \verb"tsap"

    \item	\verb"compat" (SORT OF)
    \end{nrtc}

\item	Volume 3: Applications\\
    \begin{nrtc}
    \item	\verb"ftam" \verb"ftam-ftp" \verb"ftam2" \verb"ftp-ftam"
		\verb"imisc" \verb"vt"
    \end{nrtc}

\item	Volume 4: The Applications Cookbook\\
    \begin{nrtc}
    \item	\verb"pepy" \verb"rosy"
    \end{nrtc}

\item	Volume 5: QUIPU\\
    \begin{nrtc}
    \item	\verb"dsap" \verb"quipu"
    \end{nrtc}
\end{nrtc}
\end{bwslide}


\begin{bwslide}
\part	{UNDERLYING ABSTRACTIONS}\bf

\begin{nrtc}
\item	EVEN WITHIN A SINGLE FAMILY OF \unix/,
	THERE IS JUST TOO MUCH VARIANCE IN
    \begin{nrtc}
    \item	CONSTANTS, ALGORITHMS, and INSTALLATION
    \end{nrtc}

\item	SO, NEED TO TAILOR AT
    \begin{nrtc}
    \item	COMPILE-TIME

    \item	RUN-TIME
    \end{nrtc}
\end{nrtc}
\end{bwslide}


\begin{bwslide}
\part*	{COMPILE-TIME}\bf

\begin{nrtc}
\item	TWO FILES ARE EDITED

\item	\verb"config.h" COMPILE-TIME CONSTANTS

\item	\verb"CONFIG.make" DIRECTIVES TO MAKE

\item	EACH DIRECTORY HAS \verb"make/Makefile" COMBINATION
\end{nrtc}
\end{bwslide}


\begin{bwslide}
\ctitle	{CANNED CONFIGURATIONS}

\begin{nrtc}
\item	EACH SUPPLIES TWO FILES
\[\begin{tabular}{l}
A/UX\\
APOLLO\\
BSD4.x\\
HP-UX\\
RISC/OS\\
RT/PC (BSD OR AIX)\\
SVR2\\
SVR3\\
SunOS\\
Ultrix
\end{tabular}\]
\end{nrtc}
\end{bwslide}

\begin{bwslide}
\ctitle	{OPTIONS IN \verb"config.h"}

\begin{nrtc}
\item	OPERATING SYSTEM BASE
    \begin{nrtc}
    \item	BSD: 4.2, 4.3, TAHOE, 4.4, SunOS4

    \item	SV: R2, R3, HP-UX, AIX, A/UX
    \end{nrtc}

\item	NETWORK/TRANSPORT OPTIONS: TCP, X25, TP4
    \begin{nrtc}
    \item	TCP: SOCKETS, WIN, EXOS

    \item	X.25: SunLink X.25, CAMTEC, UBC

    \item	TP4: SunLink OSI, BSD/OSI
    \end{nrtc}
\end{nrtc}
\end{bwslide}


\begin{bwslide}
\ctitle	{OPTIONS IN \verb"config.h" (cont.)}

\begin{nrtc}
\item	GETDENTS COMPATIBILITY

\item	NOTE THAT \verb"config.h" INTERACTS WITH
    \begin{nrtc}
    \item	\verb"manifest.h"

    \item	\verb"general.h"
    \end{nrtc}
\end{nrtc}
\end{bwslide}


\begin{bwslide}
\ctitle	{OPTIONS IN \verb"CONFIG.make"}

\[\begin{tabular}{rl}
BINDIR&		user programs\\
ETCDIR&		administrator files\\
INCDIR&		include files\\
LIBDIR&		object libraries\\
LINTDIR&	 lint libraries\\
LOGDIR&		log files\\
SBINDIR&	 administrator programs\\
MANDIR&		man pages\\
MANOPTS&	canned make directives\\
LSOCKET&	standard libraries\\
SYSTEM&		configuration type
\end{tabular}\]
\end{bwslide}


\begin{bwslide}
\part*	{RUN-TIME}\bf

\begin{nrtc}
\item	STARTED OFF AS OS HARMONIZATION AT THE USER-LEVEL

\item	NOW ITS GOT EVERYTHING AND THE KITCHEN-SINK

\item	CHECK \verb"llib-lcompat" FOR THE CALLING CONVENTIONS
\end{nrtc}
\end{bwslide}


\begin{bwslide}
\ctitle	{FILES OF INTEREST}

\begin{nrtc}
\item	\verb"manifest.h" USER/KERNEL

\item	\verb"logger.h" LOGGING

\item	\verb"isoaddrs.h" ADDRESSING

\item	\verb"tailor.h" TAILORING

\item	\verb"general.h" UTILITIES

\item	\verb"internet.h" TCP ABSTRACTIONS

\item	\verb"dgram.h" DATAGRAM ABSTRACTIONS

\item	\verb"x25.h" X.25 ABSTRACTIONS
\end{nrtc}
\end{bwslide}


\begin{bwslide}
\ctitle	{USER/KERNEL}

\begin{nrtc}
\item	DUP2

\item	BYTE-ORDERING

\item	SELECT
    \begin{nrtc}
    \item	\verb"xselect" HANDLE SPECIAL CASES, ROTTEN FDs

    \item	\verb"selsocket" SELECT WITH SECOND-ACCURACY
    \end{nrtc}

\item	BSD-STYLE SIGNALS
\end{nrtc}
\end{bwslide}


\begin{bwslide}
\ctitle	{USER/KERNEL (cont.)}

\begin{nrtc}
\item	A LOT OF TYPEDEFS FOR:
    \begin{nrtc}
    \item	SIGNAL HANDLING: \verb"SFP", \verb"SFD"

    \item	FD-SETs

    \item	POINTERS: \verb"IP", \verb"IFP", \verb"VFP"
    \end{nrtc}

\item	\verb"NOTOK", \verb"NULLCP", etc.

\item	BRINGS IN SYSTEM TYPE INCLUDE FILE
\end{nrtc}
\end{bwslide}


\begin{bwslide}
\ctitle	{UDVECs}

\begin{quote}\small\begin{verbatim}
struct udvec {                  /* looks like a BSD iovec... */
    caddr_t uv_base;
    int     uv_len;

    int     uv_inline;
};
\end{verbatim}\end{quote}
\end{bwslide}


\begin{bwslide}
\ctitle	{QBUFs}

\begin{quote}\small\begin{verbatim}
struct qbuf {
    struct qbuf *qb_forw;       /* doubly-linked list */
    struct qbuf *qb_back;       /*   .. */

    int     qb_len;             /* length of data */
    char   *qb_data;            /* current pointer into data */
    char    qb_base[1];         /* extensible... */
};

QBFREE (qb);
\end{verbatim}\end{quote}
\end{bwslide}


\begin{bwslide}
\ctitle	{LOGGING}

\begin{quote}\small\begin{verbatim}
LLog _pgm_log = {
    "tsapd.log", NULLCP, NULLCP,
    LLOG_FATAL | LLOG_EXCEPTIONS | LLOG_NOTICE,
    LLOG_FATAL,
    -1,
    LLOGCLS | LLOGCRT | LLOGZER,
    NOTOK
};
LLog *lp = &_pgm_log;

ll_open (lp); ... ll_close (lp);

log_tai (lp, char **argv, int argc);
ll_hdinit (lp, char *prefix);
ll_dbinit (lp, char *prefix);
\end{verbatim}\end{quote}
\end{bwslide}


\begin{bwslide}
\ctitle	{LOGGING (cont.)}

\begin{quote}\small\begin{verbatim}

ll_log (lp, LLOG_level, char *what, char *fmt, ...);

ll_printf (lp, fmt, ...);
ll_sync (lp);

SLOG (lp, event, what, (...));
LLOG (lp, event, (...));
DLOG (lp, event, what, (...));
PLOG (lp, fnx, pe, text, rw);

ll_check (lp);
\end{verbatim}\end{quote}
\end{bwslide}


\begin{bwslide}
\ctitle	{ADDRESSING}

\begin{quote}\small\begin{verbatim}
struct PSAPaddr {
    struct SSAPaddr pa_addr;

#define PSSIZE  64
    int     pa_selectlen;
    char    pa_selector[PSSIZE];
};

struct SSAPaddr {
    struct TSAPaddr sa_addr;

#define SSSIZE  64
    int     sa_selectlen;
    char    sa_selector[SSSIZE];
};
\end{verbatim}\end{quote}
\end{bwslide}


\begin{bwslide}
\ctitle	{ADDRESSING (cont.)}

\begin{quote}\small\begin{verbatim}
struct TSAPaddr {
#define NTADDR  8
    struct NSAPaddr ta_addrs[NTADDR];
    int     ta_naddr;

#define TSSIZE  64
    int     ta_selectlen;
    char    ta_selector[TSSIZE];
};
\end{verbatim}\end{quote}
\end{bwslide}


\begin{bwslide}
\ctitle	{ADDRESSING (cont.)}

\begin{quote}\small\begin{verbatim}
struct NSAPaddr {
    long    na_stack;    /* identifies TS-stack */
#define NA_NSAP 0        /* used to be called na_type */
#define NA_TCP  1
#define NA_X25  2

    long    na_community;  /* identifies community */
                           /* used to be called na_subnet */
#define SUBNET_REALNS           (-1)  /* hard-wired */
#define SUBNET_INT_X25          1       
#define SUBNET_INTERNET         3
#define SUBNET_DYNAMIC          100   /* dynamic start here... */

    union {
/* for realNS, TCP, and X.25... */
    }       na_un;
};
\end{verbatim}\end{quote}
\end{bwslide}


\begin{bwslide}
\ctitle	{ADDRESSING (cont.)}

\begin{nrtc}
\item	AN NSAP IS A NETWORK SERVICE ACCESS POINT

\item	IDENTIFIES AN (LOGICAL) ATTACHMENT TO THE NETWORK

\item	MANY DIFFERENT ADDRESSING AUTHORITIES
\end{nrtc}
\end{bwslide}


\begin{bwslide}
\ctitle	{ADDRESSING (cont.)}

\begin{nrtc}
\item	ALL ACCORDING TO KILLE'S DEFINITIONS:
    \begin{nrtc}
    \item	STRING ENCODING OF PRESENTATION ADDRESS

    \item	INTERIM USE OF NETWORK ADDRESS
    \end{nrtc}
\end{nrtc}
\begin{quote}\small\begin{verbatim}
struct PSAPaddr *str2paddr (str);

char   *_paddr2str (pa, na, compact);

struct NSAPaddr *na2norm (na);
\end{verbatim}\end{quote}
\end{bwslide}


\begin{bwslide}
\ctitle	{A STRING ENCODING OF\\ PRESENTATION ADDRESS}

\smaller
\begin{verbatim}
<address> ::= [[[ <psel> "/" ] <ssel> "/" ] <tsel> "/" ] <naddrs>

<naddrs>  ::= <naddr> "|" <naddrs> | <naddr>

<naddr>   ::= "NS" "+" <hexstring> | <afi> "+" <idi> [ "+" <dsp> ]

<psel>    ::= <selector>
<ssel>    ::= <selector>
<tsel>    ::= <selector>

<selector>::= '"' <ia5string> '"' | "#" digits | "'" <hexstring> "'H" | ""
\end{verbatim}
\end{bwslide}


\begin{bwslide}
\ctitle	{EXAMPLES}

\begin{verbatim}
"256"/NS+a433bb93c1|NS+aa3106

#63/#41/#12/X121+234219200300

'3a'H/Janet=00002340555+CUDF+"892796"

"s"//Internet=10.0.0.6

psinet=0000000100020123456789ab00
\end{verbatim}
\end{bwslide}


\begin{bwslide}
\ctitle	{FOR FURTHER READING}

\begin{nrtc}
\item	An interim approach to use of Network Addresses\\
	Stephen E.~Kille\\
	UCL-CS RN/89/13

\item	A string encoding of Presentation Address\\
	Stephen E.~Kille\\
	UCL-CS RN/89/14
\end{nrtc}
\end{bwslide}


\begin{bwslide}
\ctitle	{QUALITY OF SERVICE}

\begin{quote}\small\begin{verbatim}
struct QOStype {
                                /* transport QOS */
    int     qos_reliability;    /* "reliability" element */
#define HIGH_QUALITY    0
#define LOW_QUALITY     1

                                /* session QOS */
    int     qos_sversion;       /* session version required */
    int     qos_extended;       /* extended control */
    int     qos_maxtime;        /* for SPM response during S-CONNECT */
};
#define NULLQOS ((struct QOStype *) 0)
\end{verbatim}\end{quote}
\end{bwslide}


\begin{bwslide}
\ctitle	{TAILORING}

\begin{quote}\small\begin{verbatim}
void    isodetailor (myname, wantuser);

void    isodesetvar (name, value, dynamic);

void    isodexport (myname);

char   *isodefile (path, file);

char   *getlocalhost ();
\end{verbatim}\end{quote}
\end{bwslide}


\begin{bwslide}
\ctitle	{TAILORING VARIABLES}

\begin{nrtc}
\item	SYSTEM AREAS

\item	LOGGING FILES/PARAMETERS

\item	TRANSPORT-SWITCH

\item	X.25

\item	USER-FRIENDLY NAMESERVICE
\end{nrtc}
\end{bwslide}


\begin{bwslide}
\ctitle	{UTILITIES}

\begin{nrtc}
\item	VARARGS/SYSERR TO STRING

\item	EXPLODE/IMPLODE OCTET STRING TO/FROM PRINTABLE STRING

\item	GET PASSWORD FROM USER

\item	CASE-INSENSITIVE STRING COMPARISON

\item	ADD/REMOVE ENVIRONMENT VARIABLES

\item	STRING ALLOCATION

\item	CRACK COMMAND LINE/STRING INTO COMPONENTS
\end{nrtc}
\end{bwslide}


\begin{bwslide}
\ctitle	{UTILITIES (cont.)}

\begin{nrtc}
\item	BRINGS IN STRINGS INCLUDE FILE
\end{nrtc}
\end{bwslide}


\begin{bwslide}
\ctitle	{TCP ABSTRACTIONS}

\begin{quote}\small\begin{verbatim}
fd = start_tcp_client (sock, priv);
join_tcp_server (fd, sock);

sd = start_tcp_server (sock, backlog, opt1, opt2);
fd = join_tcp_client (sd, sock);

read_tcp_socket
write_tcp_socket
close_tcp_socket
select_tcp_socket

struct hostent *gethostbystring (s);
\end{verbatim}\end{quote}
\end{bwslide}


\begin{bwslide}
\ctitle	{TCP ABSTRACTIONS (cont.)}

\begin{nrtc}
\item	4BSD SOCKETS

\item	WIN/TCP for SVR2

\item	EXOS
\end{nrtc}
\end{bwslide}


\begin{bwslide}
\ctitle	{DATAGRAM ABSTRACTIONS}

\begin{quote}\small\begin{verbatim}
fd = start_udp_client (sock, backlog, opt1, opt2);
join_udp_server (fd, sock);

sd = start_udp_server (sock, backlog, opt1, opt2);
fd = join_udp_client (sd, sock);

read_dgram_socket (fd, struct qbuf **qb);
hack_dgram_socket (fd, sock);
write_dgram_socket (fd, struct qbuf *qb);
close_dgram_socket
select_dgram_socket
CHECK_dgram_socket
\end{verbatim}\end{quote}
DITTO FOR CLTS
\end{bwslide}


\begin{bwslide}
\ctitle	{DATAGRAM (cont.)}

\begin{nrtc}
\item	4BSD SOCKETS
\end{nrtc}
\end{bwslide}


\begin{bwslide}
\ctitle	{X.25 ABSTRACTIONS}

\begin{quote}\small\begin{verbatim}
CONN_DB *gen2if (generic, specific, context);

struct NSAPaddr *if2gen (generic, specific, context);

fd = start_x25_client (na, priv);
join_x25_server (fd, na);

sd = start_x25_server (na, backlog, opt1, opt);
fd = join_x25_client (sd, na);

read_x25_socket
write_x25_socket
close_x25_socket
select_x25_socket
\end{verbatim}\end{quote}
\end{bwslide}


\begin{bwslide}
\ctitle	{X.25 ABSTRACTIONS (cont.)}

\begin{nrtc}
\item	SunLink X.25

\item	CAMTEC/CAMTEC CCL

\item	UBC X.25
\end{nrtc}
\end{bwslide}
