% run this through LaTeX with the appropriate wrapper

\section	{Conclusions}
In this paper we have suggested that it is better to perform
{\em interface translation\/} rather than {\em protocol translation\/}
when one is interested in migrating between two protocol suites.
In our method,
which uses such an interface translation approach,
we implement the interface to the OSI Transport Services on top of the TCP.
This has the additional advantage of facilitating the development of OSI
applications in a robust and mature network environment,
and in allowing us to avoid any additional work in the future when we migrate.
In short,
we are able to make use of a complementary co-existence between the two
suites,
utilizing the best of both.
Our fundamental assumption in doing this is that the lower levels of the OSI
protocol suite will become fully supported as we follow our
migration strategy.

Furthermore,
we have discussed the difficulties inherent in providing interoperability at
the application level,
and concluded that, as a part of a migration strategy,
building the special-purpose gateways required for each pair of related
applications is not a practical approach.
Neither the protocol translation or the interface translation approach is of
any benefit in building these gateways.

Finally,
we have demonstrated that the DDN protocol suite,
because of both its maturity and closeness to to the OSI suite at the TSAP
provides an excellent migration vehicle for those users in need of
immediate electronic communications.
In view of the many new major investments being made in TCP/IP networks
(e.g., by the NSF and NASA),
we feel that our approach,
which emphasizes {\em evolution\/} rather than {\em revolution\/},
is a useful solution.
