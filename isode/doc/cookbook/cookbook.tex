% -*- LaTeX -*-		(really SLiTeX)

\documentstyle[blackandwhite,landscape,oval,pagenumbers,small,tgrind]{NRslides}

\def\emph#1{\underline{#1}}		

\font\xx=cmbx10
\font\yy=cmbx7

\raggedright

\begin{document}

\title	{BUILDING DISTRIBUTED\\ APPLICATIONS IN AN\\ OSI FRAMEWORK}
\author	{Marshall T.~Rose\\ The Wollongong Group}
\date	{April 27, 1988}
\maketitlepage


\begin{bwslide}
\ctitle	{INTRODUCTION}

\begin{nrtc}
\item	LOOSELY COUPLED SYSTEMS THAT ARE BUILT USING REMOTE PROCEDURE CALLS
	(RPC) ARE GAINING POPULARITY, e.g., NFS

\item	THE OSI REMOTE OPERATIONS CONCEPT IS INTENDED TO PROVIDE THIS
	FUNCTIONALITY FOR:
    \begin{nrtc}
    \item	MESSAGING

    \item	DIRECTORY SERVICES

    \item	NETWORK MANAGEMENT

    \item	REMOTE DATABASE ACCESS
    \end{nrtc}
\end{nrtc}
\end{bwslide}


\begin{bwslide}
\ctitle	{MOTIVATION}

\begin{nrtc}
\item	MANY FEEL THAT THIS CAPABILITY MAY BE A KEY FACTOR IN THE OVERALL
	SUCCESS OF OSI STANDARDIZATION

\item	BUT, REMOTE OPERATIONS ARE SUFFICIENTLY GENERAL TO REQUIRE
	ADDITIONAL DISCIPLINE, BEYOND THE ISO/CCITT STANDARDS,
	FOR THEIR USE AS AN RPC MECHANISM
\end{nrtc}
\end{bwslide}


\begin{bwslide}
\ctitle	{THE APPLICATIONS COOKBOOK}

\begin{nrtc}
\item	THE SET OF RULES AND LOCAL IMPLEMENTATION DECISIONS PLACED ON REMOTE
	OPERATIONS TO MAKE THE PROBLEM MANAGEABLE:
    \begin{nrtc}
    \item	LANGUAGE BINDINGS (``C'')

    \item	TOOLS FOR AUTOMATICALLY GENERATING PARTS OF THE
		PROGRAMS WHICH USE REMOTE OPERATIONS

    \item	A RUN-TIME ENVIRONMENT AND SOME BOILERPLATE

    \item	CONVENTIONS FOR NAMING AND ADDRESSING SERVICES AND ENTITIES
    \end{nrtc}

\item	A (SMALL) PART OF THE ISO DEVELOPMENT ENVIRONMENT (ISODE)
    \begin{nrtc}
    \item	VOLUME 4 OF THE USER'S MANUAL
    \end{nrtc}
\end{nrtc}
\end{bwslide}


\begin{bwslide}
\ctitle	{FOREWORD}

\begin{quote}\em
``$\ldots$ The term `holistic' refers to my conviction that what we are
concerned with here is the fundamental interconnectedness of all things.
I do not concern myself with such petty things as fingerprint powder, telltale
pieces of pocket fluff and inane footprints.
I see the solution to each problem as being detectable in the pattern and web
of the whole.
The connections between causes and effects are often much more subtle and
complex than we with our rough and ready understanding of the physical world
might naturally suppose, Mrs.~Rawlinson.

``Let me give you an example.
If you go to an acupuncturist with a toothache he sticks a needle instead into
your thigh.
Do you know why he does that, Mrs.~Rawlinson?

``No, neither do I, Mrs.~Rawlinson, but we intend to find out$\ldots$''
\end{quote}

\raggedright
---DOUGLAS ADAMS, \em Dirk Gentley's Holistic Detective Agency (1987)
\end{bwslide}


\begin{note}\em
\begin{center}
\underline{an audience survey}
\end{center}

who has heard of, is familiar with, or understands:
\begin{quote}
the osi model?

abstract syntax notation one?

remote operations in the context of OSI?

xerox courier, sun rpc, or apollo ncs/nidl?
\end{quote}

who knows how to program under unix using:
\begin{quote}
the C programming language, make, and the shell?
\end{quote}
\end{note}


\begin{bwslide}
\part*	{OUTLINE}\bf

\begin{description}
\item[PART I:]		REVIEW OF BACKGROUND MATERIAL

\item[PART II:]		A MODEL FOR DISTRIBUTED APPLICATIONS

\item[PART III:]	UNDERLYING SERVICES

\item[PART IV:]		STATIC FACILITIES

\item[PART V:]		DYNAMIC FACILITIES

\item[PART VI:]		WHERE NEXT?
\end{description}
\end{bwslide}


% -*- LaTeX -*-		(really SLiTeX)

\begin{bwslide}
\part	{REVIEW OF\\ BACKGROUND MATERIAL}\bf

\begin{nrtc}
\item	THE OSI MODEL

\item	THE UPPER-LAYER ARCHITECTURE

\item	SERVICES AND SERVICE PRIMITIVES
\end{nrtc}
\end{bwslide}


\begin{note}\em
everyone in the audience should be comfortable the material in this modest
review (and perhaps even bored, which is fine)
\end{note}


\begin{bwslide}
\part*	{THE OSI MODEL}\bf

\begin{nrtc}
\item	A LAYERED ARCHITECTURE FOR COMPUTER COMMUNICATIONS

\item	STANDARDIZED IN THE INTERNATIONAL COMMUNITY

\item	NON-PROPRIETARY IN NATURE
\end{nrtc}
\end{bwslide}


\begin{bwslide}
\ctitle	{THE MODEL FROM A COMMUNICATIONS VIEWPOINT}

\vskip.5in
\diagram[p]{figure1}
\end{bwslide}


\begin{bwslide}
\ctitle	{THE MODEL FROM A COMPUTER VIEWPOINT}

\vskip.5in
\diagram[p]{figure2}
\end{bwslide}


\begin{bwslide}
\ctitle	{(OBLIGATORY SLIDE SHOWING)\\ THE 7--LAYER STACK}

\vskip.5in
\diagram[p]{figure3}
\end{bwslide}


\begin{bwslide}
\part*	{THE UPPER-LAYER ARCHITECTURE}\bf

\begin{nrtc}
\item	BY ``UPPER-LAYER'' WE MEAN EVERYTHING ABOVE TRANSPORT:
    \begin{nrtc}
    \item	THE APPLICATION-SPECIFICS OF HOW THE NETWORK IS USED
    \end{nrtc}

\item	UNLIKE OTHER ARCHITECTURES THE SAME UPPER-LAYERS ARE USED
	REGARDLESS OF THE APPLICATION

\item	WHAT DIFFERS IS THE ACTUAL FUNCTIONALITY USED BY THE APPLICATION
\end{nrtc}
\end{bwslide}


\begin{note}\em
it's not clear at this point the effect of connectionless-mode operation on
the upper-layer architecture
\end{note}


\begin{bwslide}
\ctitle	{THE UPPER-LAYER ARCHITECTURE (cont.)}

\vskip.15in
\diagram[p]{figure4}
\end{bwslide}


\begin{bwslide}
\ctitle	{THE OSI APPLICATION LAYER}

\begin{nrtc}
\item	MANY STANDARD SERVICE ELEMENTS
    \begin{nrtc}
    \item	ASSOCIATION CONTROL

    \item	REMOTE OPERATIONS

    \item	RELIABLE TRANSFER

    \item	COMMITMENT, CONCURRENCY AND RECOVERY

    \item	DIRECTORY SERVICES
    \end{nrtc}

\item	ABSTRACT SYNTAX NOTATION ONE (ASN.1)\\
	(really a concept not an element, per se)

\item	THE DISTINCTION WILL BE DISCUSSED LATER ON
\end{nrtc}
\end{bwslide}


\begin{bwslide}
\ctitle	{APPLICATION USE OF UPPER-LAYER SERVICES}

\vskip.5in
\diagram[p]{figure5}
\end{bwslide}


\begin{bwslide}
\ctitle	{APPLICATION SERVICE ELEMENTS}

\begin{nrtc}
\item	A USEFUL MECHANISM FOR DIVIDING RESPONSIBILITY OF THE ``TOTAL''
	APPLICATION PROTOCOL

\item	PROMOTES ``REUSE'' OF APPLICATION LAYER FACILITIES
\end{nrtc}
\end{bwslide}


\begin{bwslide}
\ctitle	{EXAMPLE:\\ FTAM USE OF LOWER-LAYER SERVICES}

\vskip.5in
\diagram[p]{figure24}
\end{bwslide}


\begin{bwslide}
\part*	{SERVICES AND\\ SERVICE PRIMITIVES}\bf

\begin{nrtc}
\item	PEERS COMMUNICATE VIA \emph{SERVICE PRIMITIVES}

\item	A PRIMITIVE IS AN ABSTRACTION
    \begin{nrtc}
    \item	NOT AN INTERFACE
    \end{nrtc}

\item	SERVICE PRIMITIVES, LIKE PROCEDURE CALLS, HAVE TYPED PARAMETERS
\end{nrtc}
\end{bwslide}


\begin{bwslide}
\ctitle	{SERVICES vs. PROTOCOLS}

\vskip.5in
\diagram[p]{figure23}
\end{bwslide}


\begin{bwslide}
\ctitle	{SERVICE}

\begin{nrtc}
\item	IN GENERAL, THERE ARE THREE KINDS OF SERVICES
    \begin{nrtc}
    \item	\emph{CONFIRMED}
	\begin{nrtc}
	\item	IN WHICH A REQUEST ALWAYS RESULTS IN A RESPONSE
	\end{nrtc}

    \item	\emph{UNCONFIRMED}
	\begin{nrtc}
	\item	IN WHICH NO RESPONSE IS RETURNED
	\end{nrtc}

    \item	\emph{PROVIDER-INITIATED}
	\begin{nrtc}
	\item	IN WHICH THE SERVICE PROVIDER INDICATES SOME SITUATION
	\end{nrtc}
    \end{nrtc}

\item	CONFIRMATION IS UNRELATED TO RELIABILITY
\end{nrtc}
\end{bwslide}


\begin{bwslide}
\ctitle	{SERVICE PRIMITIVES}

\begin{nrtc}
\item	EACH LAYER (OR ELEMENT) OFFERS ONE OR MORE SERVICES
    \begin{nrtc}
    \item	e.g., A-ASSOCIATE
    \end{nrtc}

\item	A SERVICE CONSISTS OF ONE OR MORE PRIMITIVES

\item	A CONFIRMED SERVICE HAS FOUR PRIMITIVES
    \begin{nrtc}
    \item	.REQUEST, .INDICATION, .RESPONSE, and .CONFIRMATION
    \end{nrtc}

\item	AN UNCONFIRMED SERVICE HAS TWO PRIMITIVES:
    \begin{nrtc}
    \item	.REQUEST,  and .INDICATION
    \end{nrtc}

\item	A PROVIDER-INITIATED SERVICE HAS ONE PRIMITIVE:
    \begin{nrtc}
    \item	.INDICATION
    \end{nrtc}
\end{nrtc}
\end{bwslide}


\begin{bwslide}
\ctitle	{EXAMPLE: CONFIRMED SERVICE}

\vskip.5in
\diagram[p]{figure6}
\end{bwslide}


\begin{bwslide}
\ctitle	{EXAMPLE: CONFIRMED SERVICE (cont.)}

\vskip.5in
\diagram[p]{figure30}
\end{bwslide}

% -*- LaTeX -*-		(really SLiTeX)

\begin{bwslide}
\part	{A MODEL FOR DISTRIBUTED APPLICATIONS}\bf

\begin{nrtc}
\item	ABSTRACT DATA TYPES

\item	OPERATIONS

\item	ASSOCIATIONS

\item	DESIGN GUIDELINES

\item	IN PERSPECTIVE
\end{nrtc}
\end{bwslide}


\begin{note}\em
this part of the presentation corresponds to chapter~2 in The Application
Cookbook

our focus is on the 1984--style of remote operations ([X.410])
and the newer joint-iso-ccitt work ([ISO~9072/1])
\end{note}


\begin{bwslide}
\ctitle	{USE OF REMOTE OPERATIONS IN OSI}

\begin{nrtc}
\item	{}[ECMA~TR/31] PRESENTS A METHOD FOR USING REMOTE OPERATIONS TO:
    \begin{nrtc}
    \item	SPECIFY THE EXTERNALLY VISIBLE CHARACTERISTICS
		NEEDED FOR INTERCONNECTION

    \item	WHILE AVOIDING UNNECESSARY CONSTRAINTS UPON THE
		INTERNAL DESIGN OF THE SYSTEMS TO BE INTERCONNECTED
    \end{nrtc}

\item	ALTHOUGH THE LATTER HALF OF THIS DOCUMENT (THE PROTOCOL) IS NOW
	OBSOLETE, THE FIRST FOUR SECTIONS (THE METHOD) ARE QUITE RELEVANT

\item	{}[ECMA~TR/31] IS BASED ON [X.410],
	WE TERM THIS ``OLD-STYLE'' ROS

\item	{}[ISO~9072] IS THE NEWER JOINT ISO/CCITT WORK,
	WE TERM THIS ``NEW-STYLE'' ROS
\end{nrtc}
\end{bwslide}


\begin{note}\em
note that ECMA documents are not standards,
though they may be used as contributions to the standards process
\end{note}


\begin{bwslide}
\ctitle	{A BIT OF HISTORY}

\begin{nrtc}
\item	XEROX's COURIER WAS THE FIRST WELL-KNOWN SYSTEM TO USE THIS APPROACH

\item	BUT EVEN IN THE EARLY 70's, SIMILAR IDEAS WERE BEING EXPLORED
	ELSEWHERE (e.g., MIT)

\item	TODAY, SUN's RPC AND APOLLO's NCS ARE CONTINUING IN THIS VEIN
\end{nrtc}
\end{bwslide}


\begin{bwslide}
\part*	{ABSTRACT DATA TYPES}\bf

\begin{nrtc}
\item	REMOTE OPERATIONS ARE A MECHANISM BY WHICH LOOSELY COUPLED SYSTEMS
	INTERACT

\item	BUT, REMOTE OPERATIONS ARE ONLY ONE PART OF A LARGER PICTURE HOWEVER

\item	THE FUNDAMENTAL CONCEPT IS THAT OF THE \emph{ABSTRACT DATA TYPE}
\end{nrtc}
\end{bwslide}


\begin{bwslide}
\ctitle	{ABSTRACT DATA TYPES}

\begin{nrtc}
\item	PUT SIMPLY, AN ABSTRACT DATA TYPE DEFINES BOTH
    \begin{nrtc}
    \item	THE DATA STRUCTURE CONTAINED IN AN OBJECT (SYNTAX), AND

    \item	HOW THAT DATA IS INTERPRETED (SEMANTICS)
    \end{nrtc}

\item	THIS IS HARDLY A NEW CONCEPT
    \begin{nrtc}
    \item	e.g., SMALLTALK, SIMULA, and so on
    \end{nrtc}
\end{nrtc}
\end{bwslide}


\begin{bwslide}
\ctitle	{PROPERTIES OF ABSTRACT DATA TYPES:\\ REPRESENTATION}

\begin{nrtc}
\item	DATA STRUCTURES IN PROGRAMMING LANGUAGES HAVE A \emph{CONCRETE}
	REPRESENTATION
    \begin{nrtc}
    \item	WHICH IS DEFINED BY THE PROGRAMMING LANGUAGE AND THE
		UNDERLYING HARDWARE

    \item	e.g., BYTE-ORDERING, WORD SIZE, etc.
    \end{nrtc}

\item	THE CORRESPONDING ABSTRACT DATA TYPE IS DEFINED IN AN
	IMPLEMENTATION-INDEPENDENT FASHION
    \begin{nrtc}
    \item	TERMED THE \emph{ABSTRACT SYNTAX}
    \end{nrtc}

\item	AN APPLICATION CAN EXPECT THIS TO BEHAVE CONSISTENLY REGARDLESS OF THE
	HARDWARE ON WHICH IT IS RUNNING
\end{nrtc}
\end{bwslide}


\begin{bwslide}
\ctitle	{REPRESENTATION: EXAMPLE}

\vskip.15in
\begin{verbatim}
struct mail_address {
    char   *local;
    char   *domain;

    unsigned char options;
#define default_local 0x01
#define default_host  0x02
};
\end{verbatim}
\end{bwslide}


\begin{bwslide}
\ctitle	{REPRESENTATION: EXAMPLE (cont.)}

\vskip.15in
\begin{verbatim}
Mail-Address ::=
        [APPLICATION 2]
            IMPLICIT SEQUENCE {
                local[0]
                    IMPLICIT GraphicString,

                domain[1]
                    IMPLICIT GraphicString,

                options[2]
                    IMPLICIT BITSTRING {
                        default-local(0), default-host(1)
                    }
                    DEFAULT { default-local, default-host }
            }
\end{verbatim}
\end{bwslide}


\begin{bwslide}
\ctitle	{PROPERTIES OF ABSTRACT DATA TYPES:\\ SERIALIZATION}

\begin{nrtc}
\item	\emph{ABSTRACT TRANSFER NOTATION}:
    \begin{nrtc}
    \item	A WELL-DEFINED SET OF RULES USED TO DEFINE HOW ABSTRACT DATA
		TYPES ARE TRANSMITTED THROUGH THE NETWORK
    \end{nrtc}
\end{nrtc}
\end{bwslide}


\begin{bwslide}
\ctitle	{SERIALIZATION (cont.)}

\begin{nrtc}
\item	CONCEPTUALLY, TWO MAPPINGS OCCUR

\item	FIRST, THE DATA STRUCTURE IS MAPPED TO THE ABSTRACT SYNTAX FOR ITS
	CORRESPONDING ABSTRACT DATA TYPE
    \begin{nrtc}
    \item	THIS IS A LOCAL ISSUE
    \end{nrtc}

\item	SECOND, THE ABSTRACT SYNTAX IS MAPPED TO THE CONCRETE SYNTAX,
	A STREAM OF OCTETS
    \begin{nrtc}
    \item	THE ABSTRACT TRANSFER NOTATION IS USUALLY [ISO~8825]

    \item	OTHER POSSIBILITIES INCLUDE COMPRESSION, ENCRYPTION, etc.
    \end{nrtc}

\item	NOTE THAT THE CONCRETE REPRESENTATION MENTIONED EARLIER FOR
	DATA STRUCTURES IS {\bf NOT\/} THE SAME AS THE CONCRETE SYNTAX
\end{nrtc}
\end{bwslide}


\begin{bwslide}
\ctitle	{PROPERTIES OF ABSTRACT DATA TYPES:\\ OPERATIONS}

\begin{nrtc}
\item	ACCESS TO AN ABSTRACT DATA TYPE IS DEFINED BY A SET OF PRIMITIVE
	ACTIONS

\item	EACH PRIMITIVE ACTION IS TERMED AN \emph{OPERATION}

\item	THIS SET OF OPERATIONS DEFINES THE COMPLETE BEHAVIOR OF AN ABSTRACT
	DATA TYPE
\end{nrtc}
\end{bwslide}


\begin{bwslide}
\ctitle	{PROPERTIES OF ABSTRACT DATA TYPES:\\ OBJECT MODEL}

\begin{nrtc}
\item	SINCE OPERATIONS INTRODUCE A LEVEL OF INDIRECTION,
	USING ABSTRACT DATA TYPES RATHER THAN CONCRETE DATA STUCTURES
	PERMITS ACCESS TO DATA STRUCTURES WITHOUT REGARD TO THEIR ACTUAL
	IMPLEMENTATION
\end{nrtc}
\end{bwslide}


\begin{bwslide}
\part*	{OPERATIONS}\bf

\begin{nrtc}
\item	IN ITS PRIMITIVE FORM,
	AN \emph{OPERATION} IS A SIMPLE REQUEST/REPLY INTERACTION

\item	A \emph{INVOCATION} GENERATES ONE OF THREE OUTCOMES:
    \begin{nrtc}
    \item	A \emph{RESULT}, IF THE OPERATION SUCCEEDS;

    \item	AN \emph{ERROR}, IF THE OPERATION FAILED; or,

    \item	A \emph{REJECTION}, IF THE OPERATION WAS NOT PERFORMED
    \end{nrtc}

\item	OPERATIONS ARE SAID TO BE \emph{TOTAL}, AS THE NORMAL OUTCOME (RESULT),
	AND THE EXCEPTION OUTCOMES (THE ERRORS) ARE WELL-DEFINED AND
	UNAMBIGUOUS
\end{nrtc}
\end{bwslide}


\begin{bwslide}
\ctitle	{PROPERTIES OF OPERATIONS:\\ INVOCATIONS}

\begin{nrtc}
\item	THE OPERATION IS \emph{INVOKED} WHEN IT IS REQUESTED

\item	AN INVOCATION CONSISTS OF:
    \begin{nrtc}
    \item	AN \emph{OPERATION NUMBER} IDENTIFYING THE OPERATION REQUESTED

    \item	AN ARBITRARILY COMPLEX \emph{ARGUMENT}

    \item	AN \emph{INVOCATION IDENTIFIER} DISTINGUISHING THIS INVOCATION
		FROM PREVIOUS INVOCATIONS

    \item	(POSSIBLY) A \emph{LINKED-INVOCATION IDENTIFIER}
    \end{nrtc}
\end{nrtc}
\end{bwslide}


\begin{bwslide}
\ctitle	{LINKED INVOCATIONS}

\begin{nrtc}
\item	INTRODUCED IN THE NEWER JOINT ISO/CCITT WORK

\item	SOMETIMES REFERRED TO AS A ``CALLBACK'' OR A ``REMOTE UPCALL''
\end{nrtc}
\end{bwslide}


\begin{bwslide}
\ctitle	{EXAMPLE:\\ LINKED INVOCATIONS}

\begin{nrtc}
\item	CONSIDER AN OPERATION \verb"Traverse", WITH TWO ARGUMENTS:
    \begin{nrtc}
    \item	THE NAME OF A REMOTE DIRECTORY IN A FILESYSTEM

    \item	THE NUMBER OF AN OPERATION TO INVOKE FOR EACH FILE
		(WITH A HANDLE TO THE FILE)
    \end{nrtc}

\item	TO LIST A REMOTE DIRECTORY:
\begin{verbatim}
Traverse (``directory-name'', ListFile)
\end{verbatim}

\item	TO PRINT EACH FILE IN A REMOTE DIRECTORY:
\begin{verbatim}
Traverse (``directory-name'', PrintFile)
\end{verbatim}
\end{nrtc}
\end{bwslide}


\begin{bwslide}
\ctitle	{PROPERTIES OF OPERATIONS:\\ RESULTS}

\begin{nrtc}
\item	IF THE OPERATION SUCCEEDS, THEN A RESULT IS RETURNED

\item	A RESULT CONSISTS OF:
    \begin{nrtc}
    \item	AN INVOCATION IDENTIFIER CORRESPONDING TO THE OPERATION WHICH
		SUCCEEDED

    \item	(POSSIBLY) AN ARBITRARILY COMPLEX \emph{RESULT}
    \end{nrtc}
\end{nrtc}
\end{bwslide}


\begin{note}\em
actually, on success a result \emph{may optionally} be returned as some
operations are defined to not return any result

this violates the totality principle, a solution is discussed later on
\end{note}


\begin{bwslide}
\ctitle	{PROPERTIES OF OPERATIONS:\\ ERRORS}

\begin{nrtc}
\item	IF THE OPERATION FAILS, THEN AN ERROR IS RETURNED

\item	AN ERROR CONSISTS OF:
    \begin{nrtc}
    \item	AN INVOCATION IDENTIFIER CORRESPONDING TO THE OPERATION WHICH
		FAILED

    \item	AN \emph{ERROR NUMBER} IDENTIFYING THE ERROR ENCOUNTERED

    \item	(POSSIBLY) AN ARBITRARILY COMPLEX \emph{PARAMETER}
    \end{nrtc}

\item	NOTE THAT ERRORS DO NOT NECESSARILY INDICATE BAD BEHAVIOR!
    \begin{nrtc}
    \item	THEY CAN OCCUR AS A PART OF CORRECT AND NORMAL OPERATIONS

	\item	HENCE, THINK OF THEM AS EXCEPTIONS
    \end{nrtc}
\end{nrtc}
\end{bwslide}


\begin{bwslide}
\ctitle	{PROPERTIES OF OPERATIONS:\\ REJECTIONS}

\begin{nrtc}
\item	IF THE OPERATION CAN NOT BE PERFORMED, THEN A REJECTION IS RETURNED

\item	A REJECTION CONSISTS OF:
    \begin{nrtc}
    \item	AN INVOCATION IDENTIFIER CORRESPONDING TO THE OPERATION WHICH
		WAS REJECTED

    \item	A \emph{REASON} EXPLAINING WHY THE OPERATION WAS REJECTED
	\begin{nrtc}
	\item	e.g., MISTYPED PARAMETERS
	\end{nrtc}
    \end{nrtc}

\item	SOME REJECTIONS ARE USER-INITIATED, OTHERS ARE PROVIDER-INITIATED
\end{nrtc}
\end{bwslide}


\begin{bwslide}
\part*	{ASSOCIATIONS}\bf

\begin{nrtc}
\item	AN \emph{ASSOCIATION} IS A BINDING BETWEEN TWO ENTITIES,
	THE \emph{INITIATOR} AND THE \emph{RESPONDER}

\item	ASSOCIATIONS EXIST AT THE APPLICATION LAYER AND
	RELY ON AN UNDERLYING CONNECTION

\item	ASSOCIATIONS MAY BE SYMMETRIC, i.e., THEY DON'T HAVE TO FOLLOW A
	CLIENT/SERVER MODEL
\end{nrtc}
\end{bwslide}


\begin{bwslide}
\ctitle	{ASSOCIATIONS (cont.)}

\begin{nrtc}
\item	THE BINDING OCCURS IN A TWO-STEP PROCESS

\item	FIRST, THE INITIATOR DETERMINES WHICH SERVICE IT REQUIRES,
	AND ASKS (DIRECTORY SERVICES) TO MAP THIS SERVICE ONTO
	ENTITIES AVAILABLE ON THE NETWORK

\item	SECOND, BASED ON THE INITIATOR'S COMMUNICATION NEEDS
	(QUALITY OF SERVICE), AN ASSOCIATION WILL BE BOUND TO ONE OF
	THOSE ENTITIES WHICH BECOMES THE RESPONDER
\end{nrtc}
\end{bwslide}


\begin{bwslide}
\part*	{DESIGN GUIDELINES}\bf

\begin{nrtc}
\item	THE CHARACTERISTICS OF OPERATIONS WILL VARY WIDELY BETWEEN APPLICATIONS

\item	HOWEVER, THERE ARE TWO ISSUES OF UNIVERSAL INTEREST TO BE CONSIDERED
\end{nrtc}
\end{bwslide}


\begin{bwslide}
\ctitle	{RELIABILITY CHARACTERISTICS}

\begin{nrtc}
\item	UNCERTAINTY DURING EXECUTION OF OPERATIONS IS ALWAYS PRESENT

\item	THIS IS PARTICULARLY TROUBLESOME IF THE NETWORK ``BREAKS''
	AFTER A REQUEST IS RECEIVED BY THE RESPONDER BUT BEFORE
	THE INITIATOR RECEIVES THE REPLY

\item	ONE SCHEME OF CLASSIFYING THE RELIABILITY REQUIREMENTS OF AN OPERATION
	IS:
    \begin{nrtc}
    \item	EXACTLY ONCE

    \item	AT LEAST ONCE (IDEMPOTENT)

    \item	AT MOST ONCE
    \end{nrtc}

\item	IMPLEMENTING THESE SEMANTICS IS POSSIBLE USING THE INVOCATION
	IDENTIFIER
    \begin{nrtc}
    \item	BUT IS THE RESPONSBILITY OF THE USER OF REMOTE OPERATIONS
    \end{nrtc}
\end{nrtc}
\end{bwslide}


\begin{note}\em
note that ``initiator'' here doesn't necessarily mean the entity which
initiated the association

i.e., an entity can start an association, and then it's peer could possibly
initiate all of the operations
\end{note}


\begin{bwslide}
\ctitle	{RELIABILITY CHARACTERISTIC:\\ EXACTLY ONCE}

\begin{nrtc}
\item	RESPONDER
    \begin{nrtc}
    \item	KEEPS TRACK OF THE INVOCATION IDENTIFIERS OF ALL PERFORMED
		OPERATIONS

    \item	WHEN PROCESSING AN INVOCATION, IF AN INVOCATION IDENTIFIER IS
		REPEATED, REJECT THE OPERATION 
    \end{nrtc}

\item	INITIATOR
    \begin{nrtc}
    \item	REPEATEDLY INVOKES THE OPERATION USING THE SAME INVOCATION
		IDENTIFIER UNTIL EITHER

    \item	A CONFIRMATION (RESULT OR ERROR) IS RECEIVED, OR

    \item	A REJECTION (DUPLICATE OPERATION) IS RECEIVED
    \end{nrtc}

\item	A ROS BUG: REJECTION DOES NOT INCLUDE THE VALUE OF THE PREVIOUS RESULT!
\end{nrtc}
\end{bwslide}


\begin{bwslide}
\ctitle	{RELIABILITY CHARACTERISTIC:\\ AT LEAST ONCE}

\begin{nrtc}
\item	RESPONDER
    \begin{nrtc}
    \item	KEEPS NO INFORMATION REGARDING PREVIOUSLY PERFORMED OPERATIONS
    \end{nrtc}

\item	INITIATOR
    \begin{nrtc}
    \item	REPEATEDLY INVOKES THE OPERATION (WITH ANY INVOCATION
		IDENTIFIER) UNTIL

    \item	A CONFIRMATION (RESULT OR ERROR) IS RECEIVED
    \end{nrtc}
\end{nrtc}
\end{bwslide}


\begin{bwslide}
\ctitle	{RELIABILITY CHARACTERISTIC:\\ AT MOST ONCE}

\begin{nrtc}
\item	RESPONDER
    \begin{nrtc}
    \item	KEEPS NO INFORMATION REGARDING PREVIOUSLY PERFORMED OPERATIONS
    \end{nrtc}

\item	INITIATOR
    \begin{nrtc}
    \item	INVOKES THE OPERATION (WITH ANY INVOCATION IDENTIFIER)
		EXACTLY ONCE
    \end{nrtc}
\end{nrtc}
\end{bwslide}


\begin{bwslide}
\ctitle	{KEEPING TOTAL OPERATIONS TOTAL}

\begin{nrtc}
\item	IN THE OSI FRAMEWORK, IT IS POSSIBLE TO DEFINE OPERATIONS WHICH:
    \begin{nrtc}
    \item 	RETURN A RESULT, BUT NO ERRORS

    \item	RETURN ONLY ERRORS
    \end{nrtc}

\item	THIS CAN POTENTIALLY VIOLATE THE TOTALITY PRINCIPLE
    \begin{nrtc}
    \item	(ALL OUTCOMES ARE WELL-DEFINED AND UNAMBIGUOUS)
    \end{nrtc}
    AS AN OPERATION WHICH SUCCEEDS BUT RETURNS NO RESULT WILL RETURN NOTHING!

\item	THIS IN TURN LEADS TO PROBLEMS WHEN THE INITIATOR TRIES TO DETERMINE
	IF THE OPERATION SUCCEEDED OR NOT (HOW LONG TO WAIT FOR AN ERROR?)

\item	SOLUTION: OPERATIONS SHOULD ALWAYS BE ABLE TO RETURN A RESULT,
	EVEN IF THAT RESULT IS \verb"NULL"
\end{nrtc}
\end{bwslide}


\begin{note}\em
note that this totality issue is a philosophical one

some may argue that it is valid only for a class of applications
\end{note}


\begin{bwslide}
\part*	{IN PERSPECTIVE}\bf

\begin{nrtc}
\item	IDEALLY WOULD LIKE TO HIDE ALL (OR MOST) OF THIS FORMALISM FROM
	THE PROGRAMMER

\item	INSTEAD, WE'D LIKE TO PRESENT A SIMPLE PROCEDURE CALL MODEL IN WHICH
	WE DEFINE:
        \begin{nrtc}
	\item	THE INTERFACE TO EACH OPERATION AS A SUBROUTINE CALL

	\item	WITH KNOWN ARGUMENT TYPES
	\end{nrtc}
\end{nrtc}
\end{bwslide}


\begin{bwslide}
\ctitle	{AN EXAMPLE (cont.)}

\vskip.15in
\begin{verbatim}
int     op_CMIP_m__ConfirmedEventReport (sd, in, out, roi)
int     sd;
struct type_CMIP_EventReportArgument *in;
struct type_CMIP_EventReportResult *out;
struct RoSAPindication *roi;
\end{verbatim}
\end{bwslide}


\begin{bwslide}
\ctitle	{AN EXAMPLE (cont.)}

\vskip.15in
\begin{verbatim}
struct type_CMIP_EventReportArgument {
    struct type_CMIP_ObjectClass *managedObjectClass;

    struct type_CMIP_ObjectInstance *managedObjectInstance;

    struct type_CMIP_EventTypeID *eventType;

    struct type_UNIV_GeneralizedTime *eventTime;

    struct type_CMIP_EventInfo *eventInfo;
};
\end{verbatim}
\end{bwslide}

% run this through LaTeX with the appropriate wrapper

\chapter	{The ISODE Services Database}\label{isoservices}
The database \file{isoservices} in the ISODE \verb"ETCDIR" directory
(usually \file{/usr/etc/})
contains a simple mapping between
textual descriptions of services, service selectors, and local programs.

\[\fbox{\begin{tabular}{lp{0.8\textwidth}}
\bf NOTE:&	Use of this database is deprecated.
		Consult Chapter~\ref{services} on page~\pageref{services}
		of \volone/ for further information.
\end{tabular}}\]

The database itself is an ordinary ASCII text file containing information
regarding the known services on the host.
Each line contains
\begin{itemize}
\item	the name of an entity and the provider on which the entity resides;

\item	the selector used to identify the entity to the provider,
	interpreted as a:
    \begin{describe}
    \item[number,]	if the selector starts with a hash-mark (`\verb"#"').
			More precisely, this denotes the so-called
			GOSIP method for denoting selectors, which
			uses a two octet, network byte-order representation.

    \item[ascii string,]
			if the selector appears in double-quotes (`\verb|"|').
			The usual escape mechanisms can be used to
			introduce non-printable characters.

    \item[octet string,]
			if all else fails.  The standard ``explosion''
			encoding is used, each octet in the string is
			represented by a two-digit hexadecimal quantity.
    \end{describe}
	and,

\item	the program and argument vector to \man execve(2) when the service is
	requested.
\end{itemize}
Blanks and/or tab characters are used to separate items.
All items after the first two are interpreted as an argument vector.
However, double-quotes may be used to prevent separation for items containing
embedded whitspace.
The sharp character (`\verb"#"') at the beginning of a line indicates a
commentary line.

\section	{Accessing the Database}\label{isoservent}
The \man libicompat(3n) library contains the routines used to access the
database.
These routines ultimately manipulate an \verb"isoservent" structure,
which is the internal form.
\begin{quote}\index{isoservent}\small\begin{verbatim}
struct isoservent {
    char   *is_entity;
    char   *is_provider;

#define ISSIZE  64
    int     is_selectlen;
    char    is_selector[ISSIZE];

    char  **is_vec;
    char  **is_tail;
};
\end{verbatim}\end{quote}
The elements of this structure are:
\begin{describe}
\item[\verb"is\_entity":] the name of the entity;

\item[\verb"is\_provider":] the name of the provider on which the entity
resides;

\item[\verb"is\_selector"\verb"is\_selectlen":] the selector used to
identify the entity to the provider
(the element \verb"is_port" is an alias for this concept,
used to denote the entity to the provider by means of a two-octet number
specified in network-byte order);

\item[\verb"is\_vec":] the \man execve(2) vector;
and,

\item[\verb"is\_tail":] the next free slot in \verb"is_vec".
\end{describe}

The routine \verb"getisoservent" reads the next entry in the database,
opening the database if necessary.
\begin{quote}\index{getisoservent}\small\begin{verbatim}
struct isoservent *getisoservent ()
\end{verbatim}\end{quote}
It returns the manifest constant \verb"NULL" on error or end-of-file.

The routine \verb"setisoservent" opens and rewinds the database.
\begin{quote}\index{setisoservent}\small\begin{verbatim}
int     setisoservent (f)
int     f;
\end{verbatim}\end{quote}
The parameter to this procedure is:
\begin{describe}
\item[\verb"f":] the ``stayopen'' indicator,
if non-zero, then the database will remain open over subsequent calls to the
library.
\end{describe}
The routine \verb"endisoservent" closes the database.
\begin{quote}\index{endisoservent}\small\begin{verbatim}
int     endisoservent ()
\end{verbatim}\end{quote}
Both of these routines return non-zero on success and zero otherwise.

There are two routines used to fetch a particular entry in the database.
The routine \verb"getisoserventbyname" maps textual descriptions into the
internal form.
\begin{quote}\index{getisoserventbyname}\small\begin{verbatim}
struct isoservent *getisoserventbyname (entity, provider)
char   *entity,
       *provider;
\end{verbatim}\end{quote}
The parameters to this procedure are:
\begin{describe}
\item[\verb"entity":] the entity providing the desired service;
and,

\item[\verb"provider":] the provider supporting the named \verb"entity".
\end{describe}
On a successful return,
the \verb"isoservent" structure describing that service is returned.
On failure, the manifest constant \verb"NULL" is returned instead.

The routine \verb"getisoserventbyselector" performs the inverse function.
\begin{quote}\index{getisoserventbyselector}\small\begin{verbatim}
struct isoservent *getisoserventbyselector (provider,
                        selector, selectlen)
char   *provider,
       *selector;
int     selectlen;
\end{verbatim}\end{quote}
The parameters to this procedure are:
\begin{describe}
\item[\verb"provider":] the provider supporting the desired entity;
and,

\item[\verb"selector"/\verb"selectlen":] the selector on the provider
where the desired entity resides.
\end{describe}
On a successful return,
an \verb"isoservent" structure describing the entity residing on the provider
is returned.

The routine \verb"getisoserventbyport" performs a similar function.
\begin{quote}\index{getisoserventbyport}\small\begin{verbatim}
struct isoservent *getisoserventbyport (provider, port)
char   *provider;
unsigned short port;
\end{verbatim}\end{quote}
The parameters to this procedure are:
\begin{describe}
\item[\verb"provider":] the provider supporting the desired entity;
and,

\item[\verb"port":] the port on the provider (in network-byte order)
where the desired entity resides.
\end{describe}
On a successful return,
an \verb"isoservent" structure describing the entity residing on the provider
is returned.

% -*- LaTeX -*-		(really SLiTeX)

\begin{bwslide}
\part	{STATIC FACILITIES}\bf

\begin{nrtc}
\item	REMOTE OPERATIONS SPECIFICATION

\item	STUB GENERATOR

\item	STRUCTURE GENERATOR

\item	ELEMENT PARSER
\end{nrtc}
\end{bwslide}


\begin{note}\em
this part of the presentation corresponds to part~iii of The Applications
Cookbook

note that some of the facilities described herein are also useful for services
which aren't ROS-based, e.g., ODIF
\end{note}


\begin{bwslide}
\ctitle	{STATIC FACILITIES\\ OVERVIEW}

\vskip.15in
\diagram[p]{figure9}
\end{bwslide}


\begin{bwslide}
\part*	{REMOTE OPERATIONS SPECIFICATION}\bf

\begin{nrtc}
\item	A ``TYPICAL'' STANDARD FOR AN APPLICATION CONTAINS:
    \begin{nrtc}
    \item	A SERVICE DEFINITION, AND

    \item	A PROTOCOL SPECIFICATION
    \end{nrtc}

\item	THE PROTOCOL SPECIFICATION CONTAINS A FORMAL DESCRIPTION OF THE REMOTE
OPERATIONS USED BY THE APPLICATION
\end{nrtc}
\end{bwslide}


\begin{bwslide}
\ctitle	{EXAMPLE:\\ NETWORK MANAGEMENT}

\begin{nrtc}
\item	{}[ISO~9596/2] DEFINES A ``COMMON MANAGEMENT INFORMATION PROTOCOL''
	USED FOR NETWORK MANAGEMENT
    \begin{nrtc}
    \item	NOTE THAT CMIP IS NOT A ``DONE DEAL''
    \end{nrtc}

\item	IT CONTAINS INFORMATION FOR BINDING AND OPERATIONS

\item	WE'LL LOOK AT A PORTION OF THE FORMAL DESCRIPTION
\end{nrtc}
\end{bwslide}


\begin{bwslide}
\ctitle	{BINDING}\small

\vskip.15in
\begin{tgrind}
\let\linebox=\relax
\input figure11\relax
\end{tgrind}
\end{bwslide}


\begin{bwslide}
\ctitle	{OPERATIONS}\small

\vskip.15in
\begin{tgrind}
\let\linebox=\relax
\input figure12\relax
\end{tgrind}
\end{bwslide}


\begin{bwslide}
\ctitle	{OPERATIONS (cont.)}\small

\vskip.15in
\begin{tgrind}
\let\linebox=\relax
\input figure13\relax
\end{tgrind}
\end{bwslide}


\begin{bwslide}
\ctitle	{ERRORS}\small

\vskip.15in
\begin{tgrind}
\let\linebox=\relax
\input figure14\relax
\end{tgrind}
\end{bwslide}


\begin{bwslide}
\part*	{STUB GENERATOR}\bf

\begin{nrtc}
\item	WHAT WE WOULD LIKE: MAGIC!

\item	WHAT WE REALLY GET: HARD WORK.
\end{nrtc}
\end{bwslide}


\begin{bwslide}
\ctitle	{CONCEPT: STUBS}

\begin{nrtc}
\item	A PROCEDURE WHICH IS CALLED LOCALLY BUT EXECUTES REMOTELY

\item	IN OUR CONTEXT, A SYNCHRONOUS STUB:
    \begin{nrtc}
    \item	INVOKES THE OPERATION

    \item	AWAITS A RESPONSE

    \item	RETURNS A RESULT OR ERROR
    \end{nrtc}

\item	AN ASYNCHRONOUS STUB:
    \begin{nrtc}
    \item	INVOKES THE OPERATION, AND EVENTUALLY

    \item	DISPATCHES A RESULT OR ERROR HANDLER
    \end{nrtc}

\item	WHAT TO DO ABOUT REJECTIONS, NETWORK PROBLEMS, ETC?
\end{nrtc}
\end{bwslide}


\begin{bwslide}
\ctitle	{ROSY}

\begin{nrtc}
\item	REMOTE OPERATIONS STUB-GENERATOR (YACC-BASED)

\item	INPUT:
    \begin{nrtc}
    \item	A RO SPEC
    \end{nrtc}

\item	OUTPUT:
    \begin{nrtc}
    \item	AN ASN.1 SPEC

    \item	STUB DEFINITIONS FOR C

    \item	C DATA STRUCTURES FOR RUN-TIME ENVIRONMENT

    \item	STUB DEFINITIONS FOR LINT
    \end{nrtc}
\end{nrtc}
\end{bwslide}


\begin{bwslide}
\ctitle	{EXAMPLE:\\ NETWORK MANAGEMENT}\small

\vskip.15in
\begin{verbatim}
% rosy cmip.ry
rosy 3.2 #17 (gonzo) of Fri Jan  8 13:42:05 PST 1988
CMIP operations: m-EventReport m-ConfirmedEventReport m-LinkedReply
     m-Confirmed-Get m-Set m-ConfirmedSet m-Action m-ConfirmedAction

CMIP errors: noSuchObject accessDenied syncNotSupported invalidFilter
     noSuchMgmtInfoId invalidMgmtInfoValue getListError setListError
     noSuchAction processingFailure noSuchEventType

CMIP types: EventReportArgument EventReportResult LinkedReplyArgument
     GetArgument GetResult SetArgument SetResult ActionArgument ActionResult
     NoSuchObject SyncNotSupported InvalidFilter NoSuchMgmtInfoId
     InvalidMgmtInfoValue GetListError MISGetInfoStatus MgmtInfoIdError
     SetListError MISSetInfoStatus MgmtInfoError ErrorStatus NoSuchAction
     ProcessingFailure NoSuchEventType ObjectClass ObjectInstance CMISSync
     CMISFilter FilterItem AccessControl EventTypeId EventInfo MgmtInfo
     MgmtInfoId MgmtInfoValue ActionTypeId ActionInfo SpecificErrorInfo
\end{verbatim}
\end{bwslide}


\begin{bwslide}
\ctitle	{EXAMPLE:\\ STUB DEFINITIONS FOR C}\small

\vskip.15in
\begin{verbatim}
#define operation_CMIP_m__ConfirmedEventReport  1

#define stub_CMIP_m__ConfirmedEventReport(sd,id,in,rfx,efx,class,roi) \
RyStub ((sd), table_CMIP_Operations, \
        operation_CMIP_m__ConfirmedEventReport, (id), \
        (caddr_t) (in), (rfx), (efx), (class), (roi))

#define op_CMIP_m__ConfirmedEventReport(sd,in,out,rsp,roi) \
RyOperation ((sd), table_CMIP_Operations, \
        operation_CMIP_m__ConfirmedEventReport, \
        (caddr_t) (in), (out), (rsp), (roi))

#define error_CMIP_noSuchObject 1
\end{verbatim}
\end{bwslide}


\begin{bwslide}
\ctitle	{EXAMPLE:\\ C DATA STRUCTURES FOR\\ RUN-TIME ENVIRONMENT}\small

\vskip.15in
\begin{verbatim}
struct RyOperation table_CMIP_Operations[] = {
    ...

                                        /* OPERATION m-ConfirmedEventReport */
    "m-ConfirmedEventReport", operation_CMIP_m__ConfirmedEventReport,
        encode_CMIP_m__ConfirmedEventReport_argument,
        decode_CMIP_m__ConfirmedEventReport_argument,
        1, encode_CMIP_m__ConfirmedEventReport_result,
           decode_CMIP_m__ConfirmedEventReport_result,
           free_CMIP_m__ConfirmedEventReport_result,
        errors_CMIP_m__ConfirmedEventReport,

    ...

    NULL
};
\end{verbatim}
\end{bwslide}


\begin{bwslide}
\ctitle	{EXAMPLE:\\ C DATA STRUCTURES FOR\\ RUN-TIME ENVIRONMENT\\ (cont.)}
\small

\vskip.15in
\begin{verbatim}
static struct RyError *errors_CMIP_m__ConfirmedEventReport[] = {
    &table_CMIP_Errors[0],
    &table_CMIP_Errors[10],
    &table_CMIP_Errors[1],
    &table_CMIP_Errors[4],
    &table_CMIP_Errors[5]
};
\end{verbatim}
\end{bwslide}


\begin{bwslide}
\ctitle	{EXAMPLE:\\ C DATA STRUCTURES FOR\\ RUN-TIME ENVIRONMENT\\ (cont.)}
\small

\vskip.15in
\begin{verbatim}
struct RyError table_CMIP_Errors[] = {
                                        /* ERROR noSuchObject */
    "noSuchObject", error_CMIP_noSuchObject,
        encode_CMIP_noSuchObject_parameter,
        decode_CMIP_noSuchObject_parameter,
        free_CMIP_noSuchObject_parameter,

    ...

    NULL
};
\end{verbatim}
\end{bwslide}


\begin{bwslide}
\ctitle	{EXAMPLE:\\ STUB DEFINITIONS FOR LINT}\small

\vskip.15in
\begin{verbatim}
int     stub_CMIP_m__ConfirmedEventReport (sd, id, in, rfx, efx, class, roi)
int     sd,
        id,
        class;
struct type_CMIP_EventReportArgument* in;
IFP     rfx,
        efx;
struct RoSAPindication *roi;
{
    return RyStub (sd, table_CMIP_Operations,
                operation_CMIP_m__ConfirmedEventReport, id,
                (caddr_t) in, rfx, efx, class, roi);
}
\end{verbatim}
\end{bwslide}

\begin{bwslide}
\ctitle	{EXAMPLE:\\ STUB DEFINITIONS FOR LINT (cont.)}\small

\vskip.15in
\begin{verbatim}
int     op_CMIP_m__ConfirmedEventReport (sd, in, out, rsp, roi)
int     sd;
struct type_CMIP_EventReportArgument* in;
caddr_t *out;
int    *rsp;
struct RoSAPindication *roi;
{
    return RyOperation (sd, table_CMIP_Operations,
                operation_CMIP_m__ConfirmedEventReport,
                (caddr_t) in, out, rsp, roi);
}
\end{verbatim}
\end{bwslide}

\begin{bwslide}
\ctitle	{ROSY LIMITATIONS}

\begin{nrtc}
\item	SOMEWHAT LIMITED IN THE FRONT-END, CURRENTLY DOESN'T RECOGNIZE
    \begin{nrtc}
    \item	\verb"BIND" AND \verb"UNBIND" MACROS

    \item	\verb"OBJECT IDENTIFIER" NOTATION FOR OPERATION CODES
    \end{nrtc}

\item	IGNORES THE \verb"LINKED" CLAUSE IN OPERATIONS
\end{nrtc}
\end{bwslide}


\begin{bwslide}
\part*	{STRUCTURE GENERATOR}\bf

\begin{nrtc}
\item	WHAT WE WOULD LIKE: MAGIC!

\item	WHAT WE REALLY GET: HARD WORK.
\end{nrtc}
\end{bwslide}


\begin{bwslide}
\ctitle	{SERIALIZING DATA STRUCTURES}

\begin{nrtc}
\item	{}[ISO~8825] (ASN.1 ENCODING)
	SAYS HOW TO MAP THE ABSTRACT SYNTAX TO THE CONCRETE SYNTAX

\item	HOW TO MAP DATA STRUCTURES TO THE ABSTRACT SYNTAX?
    \begin{nrtc}
    \item	\verb"struct { ... }" $\rightarrow$ \verb"EventReportArgument"
    \end{nrtc}
\end{nrtc}
\end{bwslide}


\begin{bwslide}
\ctitle	{A SOLUTION}

\begin{nrtc}
\item	GENERATE C STRUCTURES DIRECTLY FROM ABSTRACT SYNTAX

\item	GENERATE TRANSLATOR TO DO THE MAPPING
\end{nrtc}
\end{bwslide}


\begin{bwslide}
\ctitle	{SIMPLE TYPES}

\begin{nrtc}
\item	\verb"BOOLEAN" $\rightarrow$ \verb"char"

\item	\verb"INTEGER" $\rightarrow$ \verb"int"
\end{nrtc}
\end{bwslide}


\begin{bwslide}
\ctitle	{SIMPLE TYPES (cont.)}

\begin{nrtc}
\item	FOR RANGE-LIMITED INTEGERs,
\begin{verbatim}
ErrorStatus ::=
        INTEGER {
            accessDenied(2),
            noSuchMgmtInfoId(5),
            invalidMgmtInfoValue(7)
        }
\end{verbatim}

SYMBOLIC VALUES ARE DEFINED AS WELL
\begin{verbatim}
struct type_CMIP_ErrorStatus {
    int     parm;
#define	int_CMIP_ErrorStatus_accessDenied 2
#define	int_CMIP_ErrorStatus_noSuchMgmtInfoId 5
#define	int_CMIP_ErrorStatus_invalidMgmtInfoValue 7
};
\end{verbatim}
\end{nrtc}
\end{bwslide}


\begin{bwslide}
\ctitle	{SIMPLE TYPES (cont.)}

\begin{nrtc}
\item	\verb"BIT STRING" $\rightarrow$ \verb"struct PElement"
\begin{verbatim}
BIT STRING {
    eventReportInvoker(0),
    ...

#define	bit_CMIP_FunctionalUnits_eventReportInvoker 0
#define	bits_CMIP_FunctionalUnits \
        "\020\01eventReportInvoker..."
\end{verbatim}

\item	\verb"OCTET STRING" $\rightarrow$ \verb"struct qbuf"

\item	\verb"OBJECT IDENTIFIER" $\rightarrow$ \verb"struct OIDentifier"
\end{nrtc}
\end{bwslide}


\begin{bwslide}
\ctitle	{COMPLEX TYPES:\\ SEQUENCE OF}

\begin{nrtc}
\item	A LINKED LIST
\begin{verbatim}
SEQUENCE OF
    MgmtInfoId
\end{verbatim}

WITH SOME GENERATED NAMES
\begin{verbatim}
struct element_CMIP_0 {
    struct type_CMIP_MgmtInfoId *element_CMIP_1;

    struct element_CMIP_0 *next;
} *element_CMIP_0;
\end{verbatim}
\end{nrtc}
\end{bwslide}


\begin{bwslide}
\ctitle	{COMPLEX TYPES:\\ SEQUENCE}

\begin{nrtc}
\item	A ``SIMPLE'' STRUCTURE
\begin{verbatim}
MgmtInfoIdError ::=
        SEQUENCE {
            errorStatus[0]
                IMPLICIT ErrorStatus,

            mgmtInfoId[1]
                MgmtInfoId
        }
\end{verbatim}

USING TAGS FOR NAMES, WHEN POSSIBLE
\begin{verbatim}
struct type_CMIP_MgmtInfoIdError {
    struct type_CMIP_ErrorStatus *errorStatus;

    struct type_CMIP_MgmtInfoId *mgmtInfoId;
};
\end{verbatim}

\item	SETS ARE HANDLED ANALAGOUSLY
\end{nrtc}
\end{bwslide}


\begin{bwslide}
\ctitle	{COMPLEX TYPES:\\ CHOICE}

\begin{nrtc}
\item	A STRUCTURE WITH A TAG AND A UNION
\begin{verbatim}
ObjectClass ::=
        CHOICE {
            globalForm[0]
                IMPLICIT OBJECT IDENTIFIER,

            nonSpecificForm[1]
                IMPLICIT OCTET STRING
        }
\end{verbatim}

e.g.,
\begin{verbatim}
struct type_CMIP_ObjectClass {
    int     offset;
#define type_CMIP_ObjectClass_globalForm 1
#define type_CMIP_ObjectClass_nonSpecificForm 2

    union {
        struct OIDentifier *globalForm;
        struct qbuf *nonSpecificForm;
    }       un;
};
\end{verbatim}
\end{nrtc}
\end{bwslide}


\begin{bwslide}
\ctitle	{DEFAULT/OPTIONAL}

\begin{nrtc}
\item	A VERY SLICK FACILITY WOULD BE TO SUPPORT THE \verb"DEFAULT" AND
	\verb"OPTIONAL" CLAUSES FOR COMPLEX TYPES

\item	BUT, IMPLEMENTATION IS PROBLEMATIC:
    \begin{nrtc}
    \item	NEED ASN.1 VALUE PARSING IN FRONT-END

    \item	NEED EXTENSIVE SYMBOL TABLE SEMANTICS IN BACK-END
    \end{nrtc}

\item	SO, A SIMPLE APPROACH IS TAKEN
    \begin{nrtc}
    \item	SCALARS ARE HANDLED DIRECTLY:
	\begin{nrtc}
	\item	BOOLEANS, INTEGERS
	\end{nrtc}

    \item	NON-SCALARS ARE EXAMINED FOR INEQUALITY TO \verb"NULL"
    \end{nrtc}
\end{nrtc}
\end{bwslide}


\begin{bwslide}
\ctitle	{HEURISTICS}

\begin{nrtc}
\item	FOR CONSTRUCTED TYPES, IF ONLY ONE MEMBER,\\ PULL IT UP
\begin{verbatim}
TestInfoIdError ::=
        SEQUENCE {
            errorStatus[0]
                IMPLICIT ErrorStatus
        }
\end{verbatim}
\verb"TestInfoIdError" $\rightarrow$ \verb"struct type_CMIP_ErrorStatus"

\item	TRY TO USE TAGS WHENEVER POSSIBLE FOR STRUCTURE NAMES
\end{nrtc}
\end{bwslide}


\begin{bwslide}
\ctitle	{POSY}

\begin{nrtc}
\item	PEPY OPTIONAL STRUCTURE-GENERATOR (YACC-BASED)

\item	INPUT:
    \begin{nrtc}
    \item	AN ASN.1 SPEC
    \end{nrtc}

\item	OUTPUT:
    \begin{nrtc}
    \item	AN AUGMENTED ASN.1 SPEC

    \item	C STRUCTURE DEFINITIONS

    \item	``FREE'' ROUTINES
    \end{nrtc}
\end{nrtc}
\end{bwslide}


\begin{bwslide}
\ctitle	{EXAMPLE:\\ NETWORK MANAGEMENT}\small

\vskip.15in
\begin{verbatim}
% posy -f -h -o cmip-asn.py cmip.py
posy 3.2 #15 (gonzo) of Fri Jan  8 12:03:11 PST 1988
CMIP types: EventReportArgument EventReportResult LinkedReplyArgument
     GetArgument GetResult SetArgument SetResult ActionArgument ActionResult
     NoSuchObject SyncNotSupported InvalidFilter NoSuchMgmtInfoId
     InvalidMgmtInfoValue GetListError MISGetInfoStatus MgmtInfoIdError
     SetListError MISSetInfoStatus MgmtInfoError ErrorStatus NoSuchAction
     ProcessingFailure NoSuchEventType ObjectClass ObjectInstance CMISSync
     CMISFilter FilterItem AccessControl EventTypeId EventInfo MgmtInfo
     MgmtInfoId MgmtInfoValue ActionTypeId ActionInfo SpecificErrorInfo
\end{verbatim}
\end{bwslide}


\begin{bwslide}
\ctitle	{EXAMPLE:\\ C STRUCTURE DEFINITIONS}\small

\vskip.15in
\begin{verbatim}
struct type_CMIP_EventReportArgument {
    struct type_CMIP_ObjectClass *managedObjectClass;

    struct type_CMIP_ObjectInstance *managedObjectInstance;

    struct type_CMIP_EventTypeID *eventType;

    struct type_UNIV_GeneralizedTime *eventTime;

    struct type_CMIP_EventInfo *eventInfo;
};

#define	type_CMIP_NoSuchObject	OIDentifier
\end{verbatim}
\end{bwslide}


\begin{bwslide}
\ctitle	{EXAMPLE:\\ ``FREE'' ROUTINES}\small

\vskip.15in
\begin{verbatim}
free_CMIP_EventReportArgument (arg)
struct type_CMIP_EventReportArgument *arg;
{
    if (arg == NULL)
        return;

    if (arg -> managedObjectClass)
        free_CMIP_ObjectClass (arg -> managedObjectClass),
            arg -> managedObjectClass = NULL;

...

    if (arg)
        free ((char *) arg);
}

#define	free_CMIP_NoSuchObject	oid_free
\end{verbatim}
\end{bwslide}


\begin{bwslide}
\ctitle	{POSY LIMITATIONS}

\begin{nrtc}
\item	STEMS FROM A LACK OF INTELLIGENCE WHEN DEALING WITH COMPLEX
	ASN.1 VALUE NOTATION:
    \begin{nrtc}
    \item	USES ``NULL INEQUALITY'' RULE FOR \verb"OPTIONAL"

    \item	HANDLES \verb"DEFAULT" ONLY FOR SCALARS
    \end{nrtc}
\end{nrtc}
\end{bwslide}


\begin{bwslide}
\part*	{ELEMENT PARSER}\bf

\begin{nrtc}
\item	WE NOW KNOW ABOUT
    \begin{nrtc}
    \item	DATA STRUCTURES\ \ \verb"struct { ... }"

    \item	ABSTRACT SYNTAX\ \ \verb"EventReportArgument"

    \item	CONCRETE SYNTAX\ \ \verb"1f8a ..."
    \end{nrtc}

\item	THE \emph{PRESENTATION ELEMENT} TIES THESE TOGETHER
\end{nrtc}
\end{bwslide}


\begin{bwslide}
\ctitle	{PRESENTATION ELEMENTS}

\begin{nrtc}
\item	AN INTERNAL FORM FOR AN INSTANCE OF A TYPE DESCRIBED BY ABSTRACT
	SYNTAX

\item	CAN REPRESENT ANY ASN.1 TYPE AS EITHER
    \begin{nrtc}
    \item	A STRING OF OCTETS OR BITS

    \item	A LINKED-LIST OF PRESENTATION ELEMENTS
    \end{nrtc}

\item	THE CONCEPUTAL MAPPING IS:
    \begin{nrtc}
    \item	\verb"struct { ... }" $\rightarrow$ \verb"EventReportArgument"
    \end{nrtc}

\item	THE ACTUAL MAPPING IS:
    \begin{nrtc}
    \item	\verb"struct { ... }" $\rightarrow$ \verb"struct PElement"
    \end{nrtc}
\end{nrtc}
\end{bwslide}


\begin{bwslide}
\ctitle	{PEPY}

\begin{nrtc}
\item	PRESENTATION ELEMENT PARSER (YACC-BASED)

\item	INPUT:
    \begin{nrtc}
    \item	AN AUGMENTED ASN.1 SPEC
    \end{nrtc}

\item	OUTPUT:
    \begin{nrtc}
    \item	AN ENCODER

    \item	A DECODER

    \item	A PRETTY-PRINTER
    \end{nrtc}
\end{nrtc}
\end{bwslide}


\begin{bwslide}
\ctitle	{EXAMPLE:\\ NETWORK MANAGEMENT}\small

\vskip.15in
\begin{verbatim}
% pepy cmip-asn.py
pepy 3.2 #15 (gonzo) of Fri Jan  8 12:03:11 PST 1988
CMIP encode none none: EventReportArgument EventReportResult
     LinkedReplyArgument GetArgument GetResult SetArgument SetResult
     ActionArgument ActionResult NoSuchObject SyncNotSupported InvalidFilter
     NoSuchMgmtInfoId InvalidMgmtInfoValue GetListError MISGetInfoStatus
     MgmtInfoIdError SetListError MISSetInfoStatus MgmtInfoError ErrorStatus
     NoSuchAction ProcessingFailure NoSuchEventType ObjectClass ObjectInstance
     CMISSync CMISFilter FilterItem AccessControl EventTypeId EventInfo
     MgmtInfo MgmtInfoId MgmtInfoValue ActionTypeId ActionInfo
     SpecificErrorInfo

CMIP none decode none: EventReportArgument EventReportResult
     LinkedReplyArgument GetArgument GetResult SetArgument SetResult
     ActionArgument ActionResult NoSuchObject SyncNotSupported InvalidFilter
     NoSuchMgmtInfoId InvalidMgmtInfoValue GetListError MISGetInfoStatus
     MgmtInfoIdError SetListError MISSetInfoStatus MgmtInfoError ErrorStatus
     NoSuchAction ProcessingFailure NoSuchEventType ObjectClass ObjectInstance
     CMISSync CMISFilter FilterItem AccessControl EventTypeId EventInfo
     MgmtInfo MgmtInfoId MgmtInfoValue ActionTypeId ActionInfo
     SpecificErrorInfo
\end{verbatim}
\end{bwslide}


\begin{bwslide}
\ctitle	{STATIC FACILITIES:\\ REVIEW}

\vskip.15in
\diagram[p]{figure9}
\end{bwslide}

% -*- LaTeX -*-		(really SLiTeX)

\begin{bwslide}
\part	{DYNAMIC FACILITIES}\bf

\begin{nrtc}
\item	RUN-TIME ENVIRONMENT

\item	BOILERPLATE FOR INITIATORS

\item	BOILERPLATE FOR RESPONDERS

\item	DEFINING A NEW SERVICE
\end{nrtc}
\end{bwslide}


\begin{note}\em
this part of the presentation corresponds to part~iv along with section~3.1
of The Applications Cookbook
\end{note}


\begin{bwslide}
\ctitle	{DYNAMIC FACILITIES\\ OVERVIEW}

\vskip.15in
\diagram[p]{figure10}
\end{bwslide}


\begin{bwslide}
\part*	{RUN-TIME ENVIRONMENT}\bf

\begin{nrtc}
\item	RUN-TIME ENVIRONMENT RESPONSIBLE FOR ``CIVILIZING'' REMOTE OPERATIONS
	SERVICE

\item	IMPORTANT TRADE-OFF:
    \begin{nrtc}
    \item	FLEXIBILITY FOR SIMPLICITY
    \end{nrtc}

\item	IN \emph{THE COOKBOOK}, \verb"ROSYLIB" IMPLEMENTS THE RUN-TIME
	ENVIRONMENT
\end{nrtc}
\end{bwslide}


\begin{bwslide}
\ctitle	{DATA STRUCTURES}

\begin{nrtc}
\item	RECALL THAT ROSY GENERATES TABLES IN ADDITION TO STUBS

\item	FOR EACH OPERATION, A TABLE IS DEFINED CONTAINING:
    \begin{nrtc}
    \item	NAME AND NUMBER OF OPERATION

    \item	ARGUMENT ENCODE/DECODE ROUTINES

    \item	RESULT ENCODE/DECODE/FREE ROUTINES

    \item	ERROR TABLE
    \end{nrtc}
\end{nrtc}\small

\begin{verbatim}
    ...

                                        /* OPERATION m-ConfirmedEventReport */
    "m-ConfirmedEventReport", operation_CMIP_m__ConfirmedEventReport,
        encode_CMIP_m__ConfirmedEventReport_argument,
        decode_CMIP_m__ConfirmedEventReport_argument,
        1, encode_CMIP_m__ConfirmedEventReport_result,
           decode_CMIP_m__ConfirmedEventReport_result,
           free_CMIP_m__ConfirmedEventReport_result,
        errors_CMIP_m__ConfirmedEventReport,

    ...
\end{verbatim}
\end{bwslide}


\begin{bwslide}
\ctitle	{DATA STRUCTURES (cont.)}

\begin{nrtc}
\item	FOR EACH ERROR, A TABLE IS DEFINED CONTAINING:
    \begin{nrtc}
    \item	NAME AND NUMBER OF ERROR

    \item	PARAMETER ENCODE/DECODE/FREE ROUTINES
    \end{nrtc}
\end{nrtc}\small

\begin{verbatim}
    ...

                                        /* ERROR noSuchObject */
    "noSuchObject", error_CMIP_noSuchObject,
        encode_CMIP_noSuchObject_parameter,
        decode_CMIP_noSuchObject_parameter,
        free_CMIP_noSuchObject_parameter,

    ...
\end{verbatim}
\end{bwslide}


\begin{bwslide}
\ctitle	{A TABLE-DRIVEN APPROACH}

\begin{nrtc}
\item	RUN-TIME ENVIRONMENT SHOULD BE GENERALIZED TO WORK WITH THE LARGEST
	POSSIBLE SET OF APPLICATIONS USING REMOTE OPERATIONS

\item	A TABLE-DRIVEN APPROACH PERMITS US TO DECOUPLE OPERATION-SPECIFIC
	INFORMATION
    \begin{nrtc}
    \item	WHAT ARGUMENTS REPRESENTED, HOW THEY ARE ENCODED, etc.
    \end{nrtc}
	FROM OPERATION-GENERIC INFORMATION
    \begin{nrtc}
    \item	THE OPERATION TAKES ARGUMENTS WHICH MUST BE ENCODED, etc.
    \end{nrtc}
\end{nrtc}
\end{bwslide}


\begin{bwslide}
\ctitle	{STUBS REVISITED}

\begin{nrtc}
\item	STUBS DEFINED BY ROSY CALL EITHER THE \verb"RyOperation" OR THE
	\verb"RyStub" ROUTINE

\item	THE \verb"RyOperation" ROUTINE IMPLEMENTS A ``POLICY''
    \begin{nrtc}
    \item	OPERATION CLASS: SYNCHRONOUS

    \item	INVOCATION IDENTIFIER: UNIQUE NUMBER

    \item	LINKED-INVOCATION ID: NONE

    \item	PRIORITY: NONE
    \end{nrtc}

\item	THE \verb"RyStub" ROUTINE IMPLEMENTS A LESS RESTRICTIVE POLICY:
    \begin{nrtc}
    \item	USER SELECTS OPERATION CLASS AND INVOCATION IDENTIFIER
    \end{nrtc}
\end{nrtc}
\end{bwslide}


\begin{bwslide}
\ctitle	{ASYNCHRONOUS STUBS}

\begin{nrtc}
\item	RECALL
{\small
\begin{verbatim}
#define stub_CMIP_m__ConfirmedEventReport(sd,id,in,rfx,efx,class,roi) \
RyStub ((sd), table_CMIP_Operations, \
        operation_CMIP_m__ConfirmedEventReport, (id), \
        (caddr_t) (in), (rfx), (efx), (class), (roi))
\end{verbatim}}

\item	THE MEANING OF THE PARAMETERS:
    \begin{nrtc}
    \item	\verb"sd": ASSOCIATION-DESCRIPTOR

    \item	\verb"id": INVOCATION IDENTIFIER

    \item	\verb"in": ARGUMENT FOR OPERATION

    \item	\verb"rfx": DISPATCH ROUTINE FOR RESULT

    \item	\verb"efx": DISPATCH ROUTINE FOR ERRORS OR REJECTIONS

    \item	\verb"class": (A)SYNCHRONOUS

    \item	\verb"roi": LOCAL SERVICE INTERFACE INFORMATION
    \end{nrtc}
\end{nrtc}
\end{bwslide}


\begin{bwslide}
\ctitle	{INSIDE RYSTUB}

\begin{nrtc}
\item	RUDIMENTARY PARAMETER CHECK

\item	BUILD PRESENTATION ELEMENT FOR ARGUMENT

\item	BUILD \emph{ACTIVATION BLOCK}

\item	ISSUE RO-INVOKE.REQUEST

\item	IF ASYNCHRONOUS, RETURN

\item	LOOP, WAITING FOR ANY RESPONSE:
    \begin{nrtc}
    \item	RO-INVOKE.INDICATION:
	\begin{nrtc}
	\item	 PUSH ACTIVATION BLOCK FOR DISPATCHING OPERATION
	\end{nrtc}

    \item	RO-RESULT, ERROR, OR REJECTION.INDICATION:
	\begin{nrtc}
	\item	PARSE OPERATION RESULT OR ERROR PARAMETER AND CALL DISPATCH
		ROUTINE
\begin{verbatim}
result = (*fnx) (sd, id, reason, value, roi);
\end{verbatim}

	\item	IF RESPONSE WAS FOR US, RETURN
	\end{nrtc}
    \end{nrtc}
\end{nrtc}
\end{bwslide}


\begin{bwslide}
\ctitle	{SYNCHRONOUS STUBS}

\begin{nrtc}
\item	RECALL
\begin{verbatim}
#define op_CMIP_m__ConfirmedEventReport(sd,in,out,rsp,roi) \
RyOperation ((sd), table_CMIP_Operations, \
        operation_CMIP_m__ConfirmedEventReport, \
        (caddr_t) (in), (out), (rsp), (roi))
\end{verbatim}

\item	THE MEANING OF THE PARAMETERS:
    \begin{nrtc}
    \item	\verb"sd": ASSOCIATION-DESCRIPTOR

    \item	\verb"in": ARGUMENT FOR OPERATION

    \item	\verb"out": RESULT FOR OPERATION OR PARAMETER FOR ERROR

    \item	\verb"rsp": TELLS HOW TO INTERPRET \verb"out"

    \item	\verb"roi": REJECTION INFORMATION FOR OPERATION
    \end{nrtc}
\end{nrtc}
\end{bwslide}


\begin{bwslide}
\ctitle	{INSIDE RYOPERATION}

\begin{nrtc}
\item	RUDIMENTARY PARAMETER CHECK

\item	BUILD PRESENTATION ELEMENT FOR ARGUMENT

\item	BUILD \emph{ACTIVATION BLOCK}

\item	ISSUE RO-INVOKE.REQUEST

\item	LOOP, WAITING FOR SOME RESPONSE:
    \begin{nrtc}
    \item	RO-INVOKE.INDICATION:
	\begin{nrtc}
	\item	 PUSH ACTIVATION BLOCK FOR DISPATCHING OPERATION
	\end{nrtc}

    \item	RO-RESULT, ERROR, OR REJECTION.INDICATION:
	\begin{nrtc}
	\item	IF FOR US: PARSE OPERATION RESULT OR ERROR PARAMETER AND RETURN

	\item	IF NOT FOR US: STUFF INFORMATION INTO CORRESPONDING ACTIVATION
		BLOCK
	\end{nrtc}
    \end{nrtc}
\end{nrtc}
\end{bwslide}


\begin{bwslide}
\ctitle	{RESPONDERS}

\begin{nrtc}
\item	REGISTER A DISPATCH ROUTINE FOR EACH OPERATION USING \verb"RyDispatch"
\begin{verbatim}
RyDispatch (sd, ryo, op, fnx, roi)
\end{verbatim}

\item	THE MEANING OF THE PARAMETERS:
    \begin{nrtc}
    \item	\verb"sd": ASSOCIATION-DESCRIPTOR

    \item	\verb"ryo": OPERATION TABLE

    \item	\verb"op": OPERATION NUMBER

    \item	\verb"fnx": DISPATCH ROUTINE

    \item	\verb"roi": FAILURE INDICATOR
    \end{nrtc}
\end{nrtc}
\end{bwslide}


\begin{bwslide}
\ctitle	{OPERATION DISPATCH}

\begin{nrtc}
\item	WHILE WAITING FOR ``SOMETHING TO HAPPEN'', A RO-INVOKE.INDICATION
	CAUSES AN ACTIVATION BLOCK TO BE PUSHED

\item	THE PRESENTATION ELEMENT FOR THE ARGUMENT IS PARSED INTO A STRUCTURE
	AND THE DISPATCH ROUTINE IS CALLED:
\begin{verbatim}
result = (*fnx) (sd, ryo, rox, in, roi)    
\end{verbatim}

\item	DISPATCH ROUTINE HAS THREE OPTIONS
    \begin{nrtc}
    \item	\verb"RyDsResult": TO RETURN A RESULT

    \item	\verb"RyDsError": TO RETURN AN ERROR

    \item	\verb"RyDsUReject": TO REJECT AN OPERATION
    \end{nrtc}
\end{nrtc}
\end{bwslide}


\begin{bwslide}
\ctitle	{CLEAN-UP}

\begin{nrtc}
\item	WHEN AN ASSOCIATION IS ABORTED (e.g., DUE TO NETWORK FAILURE),
	SOMETHING MUST BE DONE WITH ACTIVATION BLOCKS

\item	IDEALLY, WOULD LIKE TO:
    \begin{nrtc}
    \item	RETAIN THIS STATE,

    \item	RE-ESTABLISH THE ASSOCIATION, AND

    \item	CONTINUE WHERE WE LEFT OFF
    \end{nrtc}

\item	NO SUCH LUCK, ACTIVATION BLOCKS ARE FLUSHED!
    \begin{nrtc}
    \item	A LOT OF HARD ISSUES NEED TO BE RESOLVED IN ORDER TO DO THE
		``RIGHT THING''

    \item	ACTUALLY, FOR ASYNCHRONOUS INVOCATIONS, A REJECTION IS
		PASSED UP
    \end{nrtc}
\end{nrtc}
\end{bwslide}


\begin{bwslide}
\part*	{BOILERPLATE FOR INITIATORS}\bf

\begin{nrtc}
\item	THE PROBLEM WITH BOILERPLATE IS THAT IT'S BORING

\item	SO, WE'LL CONSIDER ONLY THE HIGHLIGHTS
\end{nrtc}
\end{bwslide}


\begin{bwslide}
\ctitle	{INVOKING AN OPERATION}

\begin{nrtc}
\item	ALTHOUGH \emph{THE COOKBOOK} TRIED TO MAKE THINGS SIMPLE,
	CALLING A STUB IS NOT EASY
\end{nrtc}
\end{bwslide}


\begin{bwslide}
\ctitle	{ASYNCHRONOUS INVOCATION}\small

\hrule\vskip.15in
\begin{tgrind}
\let\linebox=\relax
\def\_{\ifstring{\char'137}\else\underline{\ }\fi}
\input figure21\relax
\end{tgrind}
\end{bwslide}


\begin{bwslide}
\ctitle	{ASYNCHRONOUS INVOCATION (cont.)}\small

\hrule\vskip.15in
\begin{tgrind}
\let\linebox=\relax
\def\_{\ifstring{\char'137}\else\underline{\ }\fi}
\input figure32\relax
\end{tgrind}
\end{bwslide}


\begin{bwslide}
\ctitle	{ASYNCHRONOUS INVOCATION (cont.)}\small

\hrule\vskip.15in
\begin{tgrind}
\let\linebox=\relax
\def\_{\ifstring{\char'137}\else\underline{\ }\fi}
\input figure22\relax
\end{tgrind}
\end{bwslide}


\begin{bwslide}
\ctitle	{ASYNCHRONOUS INVOCATION (cont.)}\small

\hrule\vskip.15in
\begin{tgrind}
\let\linebox=\relax
\def\_{\ifstring{\char'137}\else\underline{\ }\fi}
\input figure31\relax
\end{tgrind}
\end{bwslide}


\begin{bwslide}
\ctitle	{ASYNCHRONOUS INVOCATION (cont.)}\small

\hrule\vskip.15in
\begin{tgrind}
\let\linebox=\relax
\def\_{\ifstring{\char'137}\else\underline{\ }\fi}
\input figure33\relax
\end{tgrind}
\end{bwslide}


\begin{bwslide}
\ctitle	{SIMPLIFIED ASYNCHRONOUS INVOCATION}\small

\hrule\vskip.15in
\begin{tgrind}
\let\linebox=\relax
\def\_{\ifstring{\char'137}\else\underline{\ }\fi}
\input figure34\relax
\end{tgrind}
\end{bwslide}


\begin{bwslide}
\ctitle	{SYNCHRONOUS INVOCATION}\small

\hrule\vskip.15in
\begin{tgrind}
\let\linebox=\relax
\def\_{\ifstring{\char'137}\else\underline{\ }\fi}
\input figure15\relax
\end{tgrind}
\end{bwslide}


\begin{bwslide}
\ctitle	{SYNCHRONOUS INVOCATION (cont.)}\small

\hrule\vskip.15in
\begin{tgrind}
\let\linebox=\relax
\def\_{\ifstring{\char'137}\else\underline{\ }\fi}
\input figure16\relax
\end{tgrind}
\end{bwslide}


\begin{bwslide}
\ctitle	{SYNCHRONOUS INVOCATION (cont.)}\small

\hrule\vskip.15in
\begin{tgrind}
\let\linebox=\relax
\def\_{\ifstring{\char'137}\else\underline{\ }\fi}
\input figure17\relax
\end{tgrind}
\end{bwslide}


\begin{bwslide}
\part*	{BOILERPLATE FOR RESPONDERS}\bf

\begin{nrtc}
\item	THE PROBLEM WITH BOILERPLATE IS THAT IT'S BORING

\item	SO, WE'LL CONSIDER ONLY THE HIGHLIGHTS
\end{nrtc}
\end{bwslide}


\begin{bwslide}
\ctitle	{RESPONDING TO AN OPERATION}

\begin{nrtc}
\item	WHEN AN ACTIVATION BLOCK IS PUSHED,
	THE DISPATCH ROUTINE IS CALLED
\end{nrtc}
\end{bwslide}


\begin{bwslide}
\ctitle	{RESPONDING TO AN OPERATION (cont.)}\small

\hrule\vskip.15in
\begin{tgrind}
\let\linebox=\relax
\def\_{\ifstring{\char'137}\else\underline{\ }\fi}
\input figure18\relax
\end{tgrind}
\end{bwslide}


\begin{bwslide}
\ctitle	{RESPONDING TO AN OPERATION (cont.)}\small

\hrule\vskip.15in
\begin{tgrind}
\let\linebox=\relax
\def\_{\ifstring{\char'137}\else\underline{\ }\fi}
\input figure19\relax
\end{tgrind}
\end{bwslide}


\begin{bwslide}
\ctitle	{RESPONDING TO AN OPERATION (cont.)}\small

\hrule\vskip.15in
\begin{tgrind}
\let\linebox=\relax
\def\_{\ifstring{\char'137}\else\underline{\ }\fi}
\input figure20\relax
\end{tgrind}
\end{bwslide}


\begin{bwslide}
\part*	{DEFINING A NEW SERVICE}\bf

\begin{nrtc}
\item	FINALLY, NEED TO IDENTIFY THE APPLICATION TO THE NETWORK

\item	THINGS TO BE DEFINED:
    \begin{nrtc}
    \item	ABSTRACT SYNTAX

    \item	APPLICATION CONTEXT NAME

    \item	APPLICATION ENTITY TITLE

    \item	PRESENTATION ADDRESS

    \item	LOCAL PROGRAM
    \end{nrtc}
\end{nrtc}
\end{bwslide}


\begin{bwslide}
\ctitle	{ABSTRACT SYNTAX}

\begin{nrtc}
\item	DESCRIBES THE DATA STRUCTURES BEING EXCHANGED BY THE SERVICE

\item	DEFINED IN THE \verb"isobjects(5)" FILE:
\begin{verbatim}
"ips-osi-mips cmip pci"    1.0.9596.2.1.1
\end{verbatim}
\end{nrtc}
\end{bwslide}


\begin{bwslide}
\ctitle	{APPLICATION CONTEXT NAME}

\begin{nrtc}
\item	DESCRIBES THE ELEMENTS AND PROTOCOL BEING USED BY THE SERVICE

\item	DEFINED IN THE \verb"isobjects(5)" FILE:
\begin{verbatim}
"iso cmip"    1.0.9596.2.2.1
\end{verbatim}
\end{nrtc}
\end{bwslide}


\begin{bwslide}
\ctitle	{NAMES AND ADDRESSES}

\begin{nrtc}
\item	APPLICATION ENTITY TITLE UNIQUELY NAMES AN ENTITY IN THE NETWORK

\item	PRESENTATION ADDRESS LOCATES AN ENTITY IN THE NETWORK

\item	DEFINED IN THE \verb"isoentities(5)" FILE:
\begin{verbatim}
gonzo-noc mgmtinfobase    1.17.4.1.9    ""    ""    #519
                          NS 49000002608c456561fe04
                          X.25 23422233939909
\end{verbatim}

\item	USED BY INITIATORS AND STATIC RESPONDERS (MULTIPLE ASSOCIATION)
\end{nrtc}
\end{bwslide}


\begin{bwslide}
\ctitle	{LOCAL PROGRAM}

\begin{nrtc}
\item	IDENTIFIES THE PROGRAM ON THE LOCAL SYSTEM WHICH IMPLEMENTS THE SERVICE

\item	DEFINED IN THE \verb"isoservices(5)" FILE:
\begin{verbatim}
"tsap/cmip"    #519    /usr/etc/ros.cmip args...
\end{verbatim}

\item	USED BY LISTENER TO FIND DYNAMIC RESPONDERS (SINGLE ASSOCIATION)
\end{nrtc}
\end{bwslide}


\begin{bwslide}
\ctitle	{DYNAMIC FACILITIES:\\ REVIEW}

\vskip.15in
\diagram[p]{figure10}
\end{bwslide}

% -*- LaTeX -*-		(really SLiTeX)

\begin{bwslide}
\part	{WHAT NOW?}\bf

\begin{nrtc}
\item	COMPARISON TO SUN RPC/XDR

\item	COMPARISON TO APOLLO NCS
\end{nrtc}
\end{bwslide}


\begin{bwslide}
\ctitle	{GUIDELINES}

\begin{nrtc}
\item	NOT TRYING TO SAY WHICH IS BETTER

\item	MERELY TRYING TO COMPARE AND CONTRAST

\item	REFERENCE DOCUMENT IS USED AS BASELINE\\
	(SOME INFORMATION MAY BE DATED)
\end{nrtc}
\end{bwslide}


\begin{bwslide}
\part*	{COMPARISON TO\\ SUN RPC/XDR}\bf

\begin{nrtc}
\item	ALTHOUGH NOT THE FIRST RPC SYSTEM DEPLOYED,
	CERTAINLY THE FIRST ``POPULARIZATION'' OF AN RPC SYSTEM

\item	SUN RPC/XDR IS BEST (UN)KNOWN FOR MAKING NFS POSSIBLE
\end{nrtc}
\end{bwslide}


\begin{bwslide}
\ctitle	{REFERENCE DOCUMENT}

\begin{nrtc}
\item	REMOTE PROCEDURE CALL PROGRAMMING GUIDE
    \begin{nrtc}
    \item	VERSION: REVISION B OF 17 FEBRUARY 1986

    \item	SOURCE: SMI DOCUMENTATION SET
    \end{nrtc}
\end{nrtc}
\end{bwslide}


\begin{bwslide}
\ctitle	{SYNTAX CHARACTERISTICS}

\begin{nrtc}
\item	NO FORMAL ABSTRACT SYNTAX, PER SE
    \begin{nrtc}
    \item	APPLICATION PROTOCOL DEFINES DATA STRUCTURES EXCHANGED

    \item	INITIALLY, NO STUB COMPILER\\ (THERE IS ONE NOW)
    \end{nrtc}

\item	SERIALIZATION METHOD IS CALLED XDR, EXTERNAL DATA REPRESENTATION
    \begin{nrtc}
    \item	CANONICAL FORM WITH IMPLICIT TAGS

    \item	MOST QUANTITIES PADDED TO 32--BIT BOUNDARIES

    \item	ANY DATA TYPE CAN BE SERIALIZED
    \end{nrtc}
\end{nrtc}
\end{bwslide}


\begin{bwslide}
\ctitle	{PROTOCOL CHARACTERISTICS}

\begin{nrtc}
\item	PROTOCOL IS SIMPLE REQUEST/REPLY INTERACTION
    \begin{nrtc}
    \item	BY CONVENTION, EACH APPLICATION HAS A ``NULL'' PROCEDURE    
    \end{nrtc}

\item	MULTIPLE TRANSPORT PROTOCOLS SUPPORTED
    \begin{nrtc}
    \item	ALTHOUGH UDP (DoD USER DATAGRAM PROTOCOL) IS THE MOST COMMON

    \item	THIS IMPACTS, e.g., THE SIZE OF ARGUMENTS THAT CAN BE PASSED
		IN A REQUEST
    \end{nrtc}

\item	SUPPORT FOR BROADCAST MEDIA
\end{nrtc}
\end{bwslide}


\begin{bwslide}
\ctitle	{BINDING CHARACTERISTICS}

\begin{nrtc}
\item	SERVICE IS IDENTIFIED BY
    \begin{nrtc}
    \item	PROGRAM NUMBER (32--BITS)

    \item	VERSION NUMBER (32--BITS)
    \end{nrtc}

\item	MAPPING OF SERVICE TO NETWORK ADDRESS IS DONE THROUGH PORT MAPPER

\item	SUPPORT FOR DIFFERENT AUTHENTICATION SCHEMES
\end{nrtc}
\end{bwslide}


\begin{bwslide}
\part*	{COMPARISON TO\\ APOLLO NCS}\bf

\begin{nrtc}
\item	ANOTHER ENTRY INTO THE POPULAR RPC MARKET

\item	EMPHASIZES OBJECT-BASED ABSTRACTIONS
\end{nrtc}
\end{bwslide}


\begin{bwslide}
\ctitle	{REFERENCE DOCUMENT}

\begin{nrtc}
\item	NETWORK COMPUTING SYSTEM: A TECHNICAL OVERVIEW
    \begin{nrtc}
    \item	VERSION: FEBRUARY 1987 (DOCUMENT 002402--322)

    \item	SOURCE: APOLLO WHITE PAPER
    \end{nrtc}
\end{nrtc}
\end{bwslide}


\begin{bwslide}
\ctitle	{SYNTAX CHARACTERISTICS}

\begin{nrtc}
\item	A FORMAL SYNTAX IS USED
    \begin{nrtc}
    \item	NETWORK INTERFACE DEFINITION LANGUAGE (NIDL)

    \item	ANY DATA TYPE CAN BE DESCRIBED

    \item	NIDL COMPILER PRODUCES ``C'' AND ``PASCAL'' BINDINGS
    \end{nrtc}

\item	SERIALIZATION (APOLLO CALLS IT \emph{MARSHALLING}) IS BASED ON THE
    \begin{nrtc}
    \item	``RECEIVER MAKES IT RIGHT''
    \end{nrtc}
    PRINCIPLE
\end{nrtc}
\end{bwslide}


\begin{bwslide}
\ctitle	{PROTOCOL CHARACTERISTICS}

\begin{nrtc}
\item	A LIGHTWEIGHT TRANSACTION PROTOCOL IS USED
    \begin{nrtc}
    \item	CAN DISTINGUISH (NON-)IDEMPOTENT OPERATIONS
    \end{nrtc}

\item	MULTIPLE TRANSPORT PROTOCOLS SUPPORTED
    \begin{nrtc}
    \item	EMPHASIZING USE OF THE BSD SOCKET ABSTRACTION
    \end{nrtc}

\item	SUPPORT FOR MULTI-TASKING
\end{nrtc}
\end{bwslide}


\begin{bwslide}
\ctitle	{BINDING CHARACTERISTICS}

\begin{nrtc}
\item	SERVICE (OBJECT) IS IDENTIFIED BY \emph{UNIQUE ID} OR
	\emph{INTERFACE NAME} 
    \begin{nrtc}
    \item	REPLICATION AND CONSISTENCY IS CONSIDERED
    \end{nrtc}

\item	MAPPING OF SERVICE TO NETWORK ADDRESS IS DONE THROUGH LOCATION BROKER

\item	SUPPORT FOR DIFFERENT AUTHENTICATION SCHEMES

\item	SUPPORT FOR AUTHORIZATION PLANNED
\end{nrtc}
\end{bwslide}



\begin{bwslide}
\part*	{ISODE AVAILABILITY}\bf

\begin{nrtc}
\item	AVAILABILITY INFORMATION
    \begin{nrtc}
    \item	USPS

    \item	DARPA/NSF INTERNET
    \end{nrtc}

\item	DISCUSSION GROUPS
\end{nrtc}
\end{bwslide}


\begin{bwslide}
\ctitle	{AVAILABILITY INFORMATION:\\ USPS}

\begin{nrtc}
\item	VERSION 3 AVAILABLE OCTOBER 14, 1987
    \begin{nrtc}
    \item	CONTAINS THE UNDERLYING SERVICES

    \item	CONTAINS THE SKELETAL FACILITIES DISCUSSED TODAY
    \end{nrtc}

\item	SEND CHECK OR PURCHASE ORDER FOR 200 US DOLLARS TO:
    \[\begin{tabular}{l}
	ISODE DISTRIBUTION\\
	DEPARTMENT OF ELECTRICAL ENGINEERING\\
	UNIVERSITY OF DELAWARE\\
	NEWARK, DE  19716\\[0.25in]
	TELCO: 302--451--1163
    \end{tabular}\]

\item	DISTRIBUTION CONTAINS:
    \begin{nrtc}
    \item	1600bpi TAR TAPE

    \item	3 VOLUME DOCUMENTATION SET
    \end{nrtc}
\end{nrtc}
\end{bwslide}


\begin{bwslide}
\ctitle	{AVAILABILITY INFORMATION:\\ DARPA/NSF INTERNET}

\begin{nrtc}
\item	VERSION 3.5 (BETA) AVAILABLE MARCH 11, 1988
    \begin{nrtc}
    \item	CONTAINS \emph{THE COOKBOOK} (VOLUME 4)

    \item	AND THE FACILITIES DISCUSSED TODAY
    \end{nrtc}

\item	USE ANONYMOUS FTP
    \[\begin{tabular}{ll}
	HOST&	louie.udel.edu\\
	FILE&	portal/isode-beta.tar.Z\\
	MODE&	binary
    \end{tabular}\]

\item	FILE IS A COMPRESSED TAR IMAGE

\item	NEED \LaTeX{} AND A LASER PRINTER
\end{nrtc}
\end{bwslide}


\begin{bwslide}
\ctitle	{DISCUSSION GROUPS}

\begin{nrtc}
\item	THE GROUP:
    \begin{nrtc}
    \item	ISODE@NRTC.NORTHROP.COM
    \end{nrtc}

\item	LIST ADDITIONS:
    \begin{nrtc}
    \item	ISODE-REQUEST@NRTC.NORTHROP.COM
    \end{nrtc}

\item	BUG REPORTS
    \begin{nrtc}
    \item	BUG-ISODE@NRTC.NORTHROP.COM
    \end{nrtc}
\end{nrtc}
\end{bwslide}


\begin{bwslide}
\part*	{BIBLIOGRAPHY}%\bf

\begin{description}
\item[{[CCITT X.410]}]	Message Handling Systems: Remote Operations
			and Reliable Transfer Server, October, 1984

\item[{[ECMA TR/31]}]	Remote Operations: Concepts, Notation and
			Connection-Oriented Mappings, December, 1985
    \begin{quote}\em
    sections 1--4 are relevant, the latter sections have been superceded by
    the more recent work below
    \end{quote}
\end{description}
\end{bwslide}


\begin{bwslide}
\ctitle	{BIBLIOGRAPHY (cont.)}

\begin{description}
\item[{[ISO 8824]}]	Information Processing Systems~---~Open Systems
			Interconnection: Specification of Abstract Syntax
			Notation One (ASN.1), May, 1986

\item[{[ISO 8825]}]	Information Processing Systems~---~Open Systems
			Interconnection: Specification of Basic Encoding
			Rules for Abstract Syntax One (ASN.1), May, 1986

\item[{[ISO 9072]}]	Information Processing Systems~---~Text
			Communication~---~MOTIS~---~Remote Operations,
			February, 1987
    \begin{description}
    \item[{[ISO 9072/1]}]	Part 1: Model, Notation and Service
				Definition

    \item[{[ISO 9072/2]}]	Part 2: Protocol Specification
    \end{description}
\end{description}
\end{bwslide}


\begin{bwslide}
\part*	{ACKNOWLEDGEMENTS}\bf

\begin{quote}\em
this presentation is based on experiences in implementing the ISO
Development Environment (ISODE),
an openly available implementation of the upper-layers of OSI

many have made significant contributions to the content and quality of
this presentation, notably:
\begin{nrtc}
\item	at ISI: Bob Braden

\item	at NMA: Einar Stefferud

\item	at Nott: Julian Onions

\item	at TWG: Danielle Bercel, Shaun Jalalian, Keith McCloghrie,
	and Don Wolski 

\item	at UCL: Steve Kille

\item	at UWisc: Nancy Hall
\end{nrtc}

also, UNIX is a trademark of at\&t bell laboratories
\end{quote}
\end{bwslide}


\end{document}
