% run this through SLiTeX

\documentstyle
    [blackandwhite,landscape,oval,pagenumbers,plain,small,tgrind]{NRslides}

\makeatletter
\def\@maketitle{%
    \newpage
    \null
    \setbox\z@=\vbox{%
	\ \vskip .75em
	\begin{center}
	    {\huge\bf \@title \par}%	%%% was \Large
	    \vskip .5em
	    {\Large\bf			%%% was \large
		\lineskip=.25em 
		\begin{tabular}[t]{c}
		    \@author 
		\end{tabular}
		\par
	    }%
	    \vskip .25em		%%% was .5em
	    {\Large\bf \@date}%		%%% was \large
	\end{center}
	\par
	\vskip .75em
    }%
    \if@ovaltitle
	\title@ht=\ht\z@	\title@wd=\wd\z@
	\title@@ht=\ht\z@	\title@@wd=\wd\z@
	\divide\title@@ht by2	\divide\title@@wd by2
	\unitlength=1sp
    \fi
    \box\z@
    \if@ovaltitle
	\vskip -\title@ht\unitlength
	{\centering
	    \begin{picture}(\title@wd,\title@ht)
		\put(\title@@wd,\title@@ht){\oval(\title@wd,\title@ht)}
	    \end{picture}
	\par}
    \fi
}
\makeatother

\raggedright

\begin{document}

\title	{FILE TRANSFER, ACCESS\\ AND MANAGEMENT}
\author	{Marshall T.~Rose}
\date	{September 1, 1987}
\maketitlepage


\begin{bwslide}
\ctitle	{INTRODUCTION}

\begin{nrtc}
\item	FTAM - FILE TRANSFER, ACCESS AND MANAGEMENT

\item	THE OPEN SYSTEMS FILE SERVICE

\item	CURRENTLY A {\bf DRAFT INTERNATIONAL STANDARD}\\
	SOON TO ACHIEVE FULL STANDARD STATUS
\end{nrtc}
\end{bwslide}


\begin{bwslide}
\ctitle	{THE OPEN SYSTEMS FILE SERVICE}

\begin{nrtc}
\item	NOT ``JUST'' FILE TRANSFER

\item	THE BASIC BUILDING BLOCK FOR OSI
    \begin{nrtc}
    \item	FILESTORE TO FILESTORE TRANSFER

    \item	WORKSTATION FILE RETRIEVAL

    \item	DISKLESS WORKSTATION PROTOCOL

    \item	SPECIAL APPLICATIONS (e.g., PRINTING, SPOOLING)

    \item	REMOTE DATABASE ACCESS
    \end{nrtc}
\end{nrtc}
\end{bwslide}


\begin{note}\em
as one of the OSI applications,
FTAM is exciting
as we get to see how all of the lower layer services are used

because of its scope,
FTAM is challenging
as it has a very large operating charter
(simple file transfer, database access, remote filesystems, and so on)

it is a largish undertaking:
the spec is about 350~pages,
the ISODE implementation (which isn't complete) is about 40K~lines of C code
(about 25\% machine-generated, thank goodness)
\end{note}


\begin{note}\em
an audience survey, who has:

heard of FTAM?

read the FTAM specification?

*understood* the FTAM specification?

heard of association control (sometimes incorrectly called CASE)?

heard of unix and the C programming language?
\end{note}


\begin{bwslide}
\part*	{OUTLINE}

\begin{nrtc}\bf
\item	THE VIRTUAL FILESTORE

\item	THE FILE SERVICE

\item	THE FILE PROTOCOL

\item	ISSUES IN IMPLEMENTING THE VIRTUAL FILESTORE

\item	ISSUES IN IMPLEMENTING A CLIENT OF THE VIRTUAL FILESTORE

\item	FTAM STATUS
\end{nrtc}
\end{bwslide}

% run this through SLiTeX with the appropriate wrapper

\begin{bwslide}
\part	{THE VIRTUAL FILESTORE}

\begin{nrtc}\bf
\item	PHILOSOPHY

\item	FILE ATTRIBUTES

\item	ACTIVITY ATTRIBUTES

\item	DOCUMENT TYPES
\end{nrtc}
\end{bwslide}


\begin{note}\em
this section corresponds roughly to iso/dis 8571/2,
but the concepts are explained in almost an entirely different way

this description is done via successive refinement:
concepts are introduced and continuously expanded

hopefully,
this is less intimidating than the way the standard presents things$\ldots$
\end{note}


\begin{bwslide}
\part*	{PHILOSOPHY}\bf

\begin{nrtc}
\item	AS WITH ALL ``OPEN SYSTEM'' SERVICES
    \begin{nrtc}
    \item	DESCRIBES A CONCEPTUAL MODEL OF THE VIRTUAL SERVICE

    \item	SPECIFIES THE SERVICE AND THE PROTOCOL\\
		INDEPENDENT OF ACTUAL LOCAL SYSTEMS
	\begin{nrtc}
	\item	PROGRAMATIC INTERFACES ARE NOT SPECIFIED
	\end{nrtc}
    \end{nrtc}

\item	THE FUNDAMENTAL ABSTRACTION: THE VIRTUAL FILESTORE

\item	A CONCEPTUAL MODEL OF A FILE SERVICE ON A LOCAL SYSTEM (LOCALSTORE)

\item	DIFFICULT TASK~---~EXISTING FILE SERVICES ARE QUITE DIFFERENT

\item	POTENTIALLY VERY REWARDING!
\end{nrtc}
\end{bwslide}


\begin{bwslide}
\ctitle	{RELATIONSHIP OF THE VIRTUAL FILESTORE\\ AND LOCALSTORE}

\vskip.5in
\diagram[p]{figure1}
\end{bwslide}


\begin{note}\em
why a virtual filestore?

it is unacceptable to choose any existing real filestore as the basis
for the file service (all lack one thing or another)

hence, it is desirable to devise a model which can reasonably express any
existing real filestore.
\end{note}


\begin{bwslide}
\ctitle	{ELEMENTS}\bf

\begin{nrtc}
\item	A (VIRTUAL) FILESTORE IS A COLLECTION OF FILES

\item	A FILENAME IDENTIFIES EXACTLY ONE FILE IN THE FILESTORE

\item	THERE IS NO EXPLICIT RELATIONSHIP BETWEEN DIFFERENT FILES IN THE
	FILESTORE
    \begin{nrtc}
    \item	i.e., NO DIRECTORY STRUCTURE (A {\bf BIG} MISTAKE)
    \end{nrtc}

\item	FILES HAVE
    \begin{nrtc}
    \item	ATTRIBUTES (e.g., OWNERSHIP INFORMATION)

    \item	CONTENTS (e.g., RANDOM-ACCESS RECORDS)
    \end{nrtc}
\end{nrtc}
\end{bwslide}


\begin{bwslide}
\ctitle	{ELEMENTS -- ATTRIBUTES}

\begin{nrtc}
\item	TWO KINDS OF ATTRIBUTES ARE DEFINED

\item	FILE ATTRIBUTES, WHICH EXIST ON A PER-FILE BASIS
    \begin{nrtc}
        \item	SIMULTANEOUS CLIENTS OF THE FILESTORE SEE THE SAME INFORMATION

	\item	e.g., THE NAME OF THE FILE
    \end{nrtc}

\item	ACTIVITY ATTRIBUTES, WHICH EXIST ON A PER-CLIENT BASIS
    \begin{nrtc}
    \item	INTERACTIONS BY A CLIENT ARE NOT DIRECTLY VISIBLE TO OTHER
		CLIENTS

    \item	e.g., HOW THE FILE IS BEING TRAVERSED
    \end{nrtc}

\item	THE CLIENT INTERACTS ON AT MOST ONE FILE
    \begin{nrtc}
    \item	THE ``SELECTED'' FILE
    \end{nrtc}
\end{nrtc}
\end{bwslide}


\begin{bwslide}
\ctitle	{ELEMENTS -- CONTENTS}

\begin{nrtc}
\item	TYPICALLY, FILES ARE DEFINED IN TERMS OF A ``DOCUMENT TYPE''

\item	STATIC CHARACTERISTICS
    \begin{nrtc}
    \item	THE COMPOSITION OF THE FILE IN TERMS OF FILE ACCESS DATA
		UNITS (FADUs)\\
		e.g., A SEQUENTIAL COLLECTION OF RECORDS

    \item	THE STRUCTURE OF EACH DATA UNIT (DUs)\\
		e.g., EACH RECORD CONTAINS A PERSONNEL RECORD
    \end{nrtc}

\item	DYNAMIC CHARACTERISTICS
    \begin{nrtc}
    \item	HOW DATA UNITS ARE ENCODED ON THE NETWORK

    \item	HOW DATA UNITS ARE REFERENCED (e.g., CURRENT POSITION)
    \end{nrtc}
\end{nrtc}
\end{bwslide}


\begin{bwslide}
\part*	{FILE ATTRIBUTES}\bf

\begin{nrtc}
\item	FOUR GROUPS OF FILE ATTRIBUTES

\item	KERNEL GROUP (REQUIRED)
    \begin{nrtc}
    \item	NECESSARY FOR FILE SELECTION AND BASIC FILE TRANSFER
    \end{nrtc}

\item	STORAGE GROUP (OPTIONAL)
    \begin{nrtc}
    \item	DESCRIBES THE STORAGE CHARACTERISTICS FOR THE FILE
    \end{nrtc}

\item	SECURITY GROUP (OPTIONAL)
    \begin{nrtc}
    \item	DESCRIBES THE ACCESS CONTROL MECHANISMS FOR THE FILE
    \end{nrtc}

\item	PRIVATE GROUP (OPTIONAL)
    \begin{nrtc}
    \item	A MECHANISM TO CAPTURE NON-STANDARD (PROPRIETARY)
		MECHANISMS THAT CAN NOT BE OTHERWISE REPRESENTED
    \end{nrtc}
\end{nrtc}
\end{bwslide}


\begin{note}\em
definitions of types is rather loose at this point in the presentation;
e.g., ``string'' is usually an asn.1 graphicstring

the emphasis at the moment is on the concept,
not on the actual abstract data type
\end{note}


\begin{bwslide}
\ctitle	{KERNEL GROUP}

\begin{nrtc}
\item	FILENAME: A SEQUENCE OF STRINGS
    \begin{nrtc}
    \item	MAPPING TO THE LOCALSTORE NAMING CONVENTIONS IS A
		``LOCAL IMPLEMENTATION CHOICE''
    \end{nrtc}

\item	CONTENTS TYPE: STRUCTURING INFORMATION
    \begin{nrtc}
    \item	THE FILE STRUCTURE (A {\bf LOT} MORE LATER)
    \end{nrtc}
	
\end{nrtc}
\end{bwslide}


\begin{bwslide}
\ctitle	{STORAGE GROUP}

\begin{nrtc}
\item	STORAGE ACCOUNT: A STRING
    \begin{nrtc}
    \item	ENTITY ACCRUING FILE STORAGE CHARGES
    \end{nrtc}	

\item	IDENTITY OF USER AND THE DATE/TIME OF
    \begin{nrtc}
    \item	FILE CREATION

    \item	LAST READ AND LAST MODIFICATION OF FILE CONTENTS

    \item	LAST MODIFICATION OF FILE ATTRIBUTES
    \end{nrtc}

\item	FILE AVAILABILITY
    \begin{nrtc}
    \item	IMMEDIATE (FILE IS ``ON-LINE'')

    \item	DEFFERRED (ACCESS TO FILE MAY ENCOUNTER DELAY,
		e.g., AWAITING ARCHIVE RETRIEVAL)
    \end{nrtc}
\end{nrtc}
\end{bwslide}


\begin{bwslide}
\ctitle	{STORAGE GROUP (cont.)}

\begin{nrtc}
\item	PERMITTED ACTIONS
    \begin{nrtc}
    \item	DESCRIBES THE TYPES OF DATA ACCESS THAT CAN BE PERFORMED ON
		THE FILE

    \item	HOW DATA UNITS MAY BE ACCESSED
		(READ, WRITE, EXTEND, etc.)

    \item	HOW THE FILE MAY BE TRAVERSED
		(MOVING FROM ONE DATA UNIT TO ANOTHER)
    \end{nrtc}

\item	FILESIZE (IN OCTETS)
    \begin{nrtc}
    \item	AN ESTIMATE OF THE TOTAL SIZE OF THE FILE'S CONTENTS
    \end{nrtc}
	

\item	FUTURE FILESIZE (IN OCTETS)
    \begin{nrtc}
    \item	A SOFT LIMIT ON THE TOTAL SIZE OF THE FILE'S CONTENTS
    \end{nrtc}
\end{nrtc}
\end{bwslide}


\begin{bwslide}
\ctitle	{SECURITY GROUP}

\begin{nrtc}
\item	ACCESS CONTROL (AN ACCESS CONTROL LIST)\\
	FOR EACH ELEMENT OF THE LIST:
    \begin{nrtc}
    \item	FILE ACTIONS PERMITTED

    \item	ENTITY PERMITTED TO REQUEST ACTION (OPTIONAL)

    \item	PASSWORD REQUIRED TO VALIDATE ACTION
    \end{nrtc}

\item	ENCRYPTION NAME
    \begin{nrtc}
    \item	DEFINES HOW FILE WAS ENCRYPTED

    \item	FILES ARE TRANSFERRED IN ENCRYPTED FORM

    \item	REQUIRES A REGISTRATION AUTHORITY TO BE ESTABLISHED
    \end{nrtc}

\item	LEGAL QUALIFICATIONS
    \begin{nrtc}
    \item	DEFINES THE ``LEGAL STATUS'' OF THE FILE

    \item	MEANT TO BE USED WITH A NATIONAL PRIVACY LEGISLATION
    \end{nrtc}
\end{nrtc}
\end{bwslide}


\begin{bwslide}
\ctitle	{PRIVATE GROUP}

\begin{nrtc}
\item	A ``CATCH-ALL''

\item	USE IS STRONGLY DISCOURAGED
\end{nrtc}
\end{bwslide}


\begin{bwslide}
\part*	{ACTIVITY ATTRIBUTES}\bf

\begin{nrtc}
\item	ACTIVITY ATTRIBUTES ARE ALSO DEFINED IN TERMS OF GROUPS\\
	KERNEL, STORAGE, AND SECURITY (NO PRIVATE GROUP, OBVIOUSLY)

\item	THESE ARE USUALLY INITIALIZED WHEN A FILE IS EITHER
    \begin{nrtc}
    \item	SELECTED

    \item	OPENED FOR TRANSFER/ACCESS
    \end{nrtc}
\end{nrtc}
\end{bwslide}


\begin{note}\em
relationship of actions is rather loose at this point in the presentation;
selection, open, transfer/access

the next section will formalize these terms
\end{note}


\begin{bwslide}
\ctitle	{KERNEL GROUP ACTIVITY ATTRIBUTES}

\begin{nrtc}
\item	ACTIVE CONTENTS TYPE
    \begin{nrtc}
    \item	THE CONTENTS TYPE
    \end{nrtc}

\item	CURRENT ACCESS REQUEST
    \begin{nrtc}
    \item	THOSE PERMITTED ACTIONS WHICH ARE REQUESTED BY THE CLIENT

    \item	CONTENTS: READ, INSERT, REPLACE, EXTEND, ERASE

    \item	ATTRIBUTES: READ, CHANGE, AND DELETE FILE
    \end{nrtc}

\item	CURRENT LOCATION

\item	CURRENT PROCESSING MODE
    \begin{nrtc}
    \item	ACTIONS ON THE CONTENTS WHICH THE CLIENT WISHES TO PERFORM
		(READ, INSERT, REPLACE, EXTEND, ERASE, LOCATE)
    \end{nrtc}
	
\item	CURRENT APPLICATION ENTITY TITLE
    \begin{nrtc}
    \item	A GLOBAL IDENTIFIER FOR THE ENTITY PROVIDING THE FILE SERVICE
		(A FILESTORE OR FILESERVER)
    \end{nrtc}
\end{nrtc}
\end{bwslide}


\begin{bwslide}
\ctitle	{STORAGE GROUP ACTIVITY ATTRIBUTES}

\begin{nrtc}
\item	CURRENT ACCOUNT
    \begin{nrtc}
    \item	THE CLIENT'S ACCOUNT WHEN THE FILE SERVICE WAS INITIATED
		(MAY BE CHANGED WHEN A FILE IS SELECTED)
    \end{nrtc}

\item	CURRENT ACCESS CONTEXT
    \begin{nrtc}
    \item	HOW THE FILE STRUCTURE IS COMMUNICATED ON THE NETWORK
		(MUCH MORE ON THIS LATER)
    \end{nrtc}

\item	CURRENT CONCURRENCY CONTROL
    \begin{nrtc}
    \item	HOW SIMULTANEOUS CLIENTS INTERACT WHEN ACCESSING THE FILE

    \item	FOR EACH ACTION: SHARED, EXCLUSIVE, NOT REQUIRED, NO ACCESS
    \end{nrtc}
\end{nrtc}
\end{bwslide}


\begin{bwslide}
\ctitle	{SECURITY GROUP ACTIVITY ATTRIBUTES}

\begin{nrtc}
\item	ACTIVE LEGAL QUALIFICATION

\item	CURRENT INITIATOR IDENTITY
    \begin{nrtc}
    \item	THE CLIENT'S IDENTITY WHEN THE FILE SERVICE WAS INITIATED
    \end{nrtc}

\item	CURRENT ACCESS PASSWORDS
    \begin{nrtc}
    \item	THE ACCESS LIST WHICH APPLIES TO THE CLIENT
    \end{nrtc}
\end{nrtc}
\end{bwslide}


\begin{bwslide}
\part*	{DOCUMENT TYPES}\bf

\begin{nrtc}
\item	STATIC CHARACTERISTICS
    \begin{nrtc}
    \item	THE FILE ACCESS STRUCTURE (CONSTRAINT SET NAME)

    \item	THE PRESENTATION STRUCTURE (ABSTRACT SYNTAX NAME)
    \end{nrtc}

\item	DYNAMIC CHARACTERISTICS
    \begin{nrtc}
    \item	THE TRANSFER STRUCTURE (TRANSFER SYNTAX NAME)

    \item	A IDENTIFICATION STRUCTURE (ACCESS CONTEXTS)
    \end{nrtc}

\item	``REGISTERED'' AND REFERENCED VIA A UNIQUE IDENTIFIER
\end{nrtc}
\end{bwslide}


\begin{bwslide}
\ctitle	{FILE ACCESS STRUCTURE}

\begin{nrtc}
\item	ANY FILE'S CONTENT CAN BE DESCRIBED AS A TREE

\item	EACH NODE IN THE TREE CONTAINS
    \begin{nrtc}
    \item	A DESCRIPTOR (A NAME AND DISTANCE TO PARENT)

    \item	OPTIONALLY, A DATA UNIT (DEFINED BY THE PRESENTATION STRUCTURE)

    \item	OPTIONALLY, CHILDREN (OTHER NODES)
    \end{nrtc}

\item	THE ROOT NODE DEFINES THE ``STARTING'' POINT FOR THE FILE

\item	NEED A WAY TO LIMIT THE COMPLEXITY OF THE TREE
\end{nrtc}
\end{bwslide}


\begin{bwslide}
\ctitle	{CONSTRAINT SETS}

\begin{nrtc}
\item	DEFINES THE STRUCTURE OF THE TREE AND HOW ACTIONS ON THE FILE
	(e.g., WRITE, ERASE) CHANGE THE STRUCTURE AND POSITION

\item	SEVERAL KINDS
    \begin{nrtc}
    \item	UNSTRUCTURED

    \item	SEQUENTIAL FLAT

    \item	ORDERED FLAT

    \item	ORDERED FLAT WITH UNIQUE NAMES

    \item	ORDERED HIERARCHICAL

    \item	GENERAL HIERARCHICAL

    \item	GENERAL HIERARCHICAL WITH UNIQUE NAMES
    \end{nrtc}
\end{nrtc}
\end{bwslide}


\begin{bwslide}
\ctitle	{EXAMPLE: UNSTRUCTURED CONSTRAINT SET}

\vskip.5in
\diagram[p]{figure2}
\end{bwslide}


\begin{note}\em
a unnamed root node with a data unit but no children

file can be transferred as a whole, or extended

access to a part is not permitted
\end{note}


\begin{bwslide}
\ctitle	{EXAMPLE: SEQUENTIAL FLAT CONSTRAINT SET}

\vskip.5in
\diagram[p]{figure3}
\end{bwslide}


\begin{note}\em
a two-level tree:\\
a root with no data unit but with zero or more children;
and,
each child has a data unit but no children

data unit is identified based on position in the file (relation to siblings)

insertions occur at end of file

erase at root node to empty file

ordered-flat differs by naming each child and identifying data units based
on the name
\end{note}


\begin{bwslide}
\ctitle	{EXAMPLE: GENERAL HIERARCHICAL CONSTRAINT SET}

\vskip.5in
\diagram[p]{figure4}
\end{bwslide}


\begin{note}\em
hierarchical: a tree of arbitrary structure

at a given level, nodes have the same ``type'' of name

insert as sister (sibling):\\
the data unit becomes the next child visited when using preorder traversal
(not valid for the root node, obviously)

insert as child:\\
the data unit becomes the first child visited when using preorder traversal

note difference between fadu and data(du)
\end{note}


\begin{bwoverlay}
\ctitle	{EXAMPLE: GENERAL HIERARCHICAL CONSTRAINT SET}

\vskip.5in
\diagram[p]{figure4a}
\end{bwoverlay}


\begin{bwoverlay}
\ctitle	{EXAMPLE: GENERAL HIERARCHICAL CONSTRAINT SET}

\vskip.5in
\diagram[p]{figure4b}
\end{bwoverlay}


\begin{bwoverlay}
\ctitle	{EXAMPLE: GENERAL HIERARCHICAL CONSTRAINT SET}

\vskip.5in
\diagram[p]{figure4c}
\end{bwoverlay}


\begin{bwslide}
\ctitle	{PRESENTATION STRUCTURE}

\begin{nrtc}
\item	STRUCTURE OF EACH DATA UNIT IS DEFINED USING ABSTRACT SYNTAX NOTATION
	ONE (ASN.1)

\item	SPECIFICATION CAN BE SIMPLE, e.g., A STRING OF OCTETS

\item	OR COMPLEX, e.g., A PERSONNEL RECORD
\end{nrtc}
\end{bwslide}


\begin{bwslide}
\ctitle	{TRANSFER STRUCTURE}

\begin{nrtc}
\item	DATA UNITS ARE COMPOSED OF ``DATA ELEMENTS''

\item	EACH DATA ELEMENT MAPS DIRECTLY TO A ``WRITE'' TO THE NETWORK

\item	A TRANSFER STRUCTURE IS SAID TO BE ``SELF-DELIMITING'' IF EACH
	DATA UNIT MAPS TO EXACTLY ONE DATA ELEMENT

\item	OTHERWISE, A 1:n RATIO IS USED AS AN EFFICIENCY ``HACK''
    \begin{nrtc}
    \item	DATA ELEMENTS ARE CONCATENATED TO FORM A SINGLE DATA UNIT
    \end{nrtc}
\end{nrtc}
\end{bwslide}


\begin{bwslide}
\ctitle	{IDENTIFICATION STRUCTURE}

\begin{nrtc}
\item	ACCESS CONTEXT:
	AN ALGORITHM FOR DEFINING A ASPECT OF THE FILE STRUCTURE

\item	ACTIONS TAKEN IN THE CONTEXT OF THE CURRENT POSITION\\ (i.e., NODE)

\item	RECURSIVE ACTIONS PERFORMED IN PREORDER TRAVERSAL

\item	IMPORTANT DISTINCTION
    \begin{nrtc}
    \item	ALL DATA UNITS~---~THE NODE'S DATA UNIT AND DATA BELONGING
		TO ALL CHILDREN OF THE NODE

    \item	SINGLE DATA UNIT~---~THE NODE'S DATA UNIT (IGNORE CHILDREN)
    \end{nrtc}

\item	SEVERAL DEFINED (OF COURSE)
\end{nrtc}
\end{bwslide}


\begin{bwslide}
\ctitle	{EXAMPLE: ACCESS CONTEXTS}

\vskip.5in
\diagram[p]{figure4}
\end{bwslide}


\begin{note}\em
unstructured single data unit (US):\\
transmit the node's data

unstructured all data units (UA):\\
transmit all data in the fadu

flat single data unit (FS):\\
transmit the single node's name and all data in the fadu

flat one level data units (FL):\\
transmit names/data from all nodes at a given level having data

flat all data units (FA):\\
transmit names/data from all nodes having data

hierarchical no data units (HN):\\
transmit name, data, and structure

hierarchical all data units (HA):\\
transmit name, data, and structure
\end{note}


\begin{bwslide}
\part*	{SUMMARY}\bf

\begin{nrtc}
\item	THE VIRTUAL FILESTORE IS THE OPEN SYSTEMS ABSTRACTION OF A LOCALSTORE

\item	FILES CONTAIN ATTRIBUTES AND STRUCTURING INFORMATION IN ADDITION TO
	``TYPED'' DATA

\item	FILES ARE DISTINGUISHED BY NAME

\item	SOME ATTRIBUTES ARE DYNAMIC, ON A PER-CLIENT BASIS

\item	STRUCTURE IS BASED ON A HIERARCHICAL MODEL

\item	DATA AND STRUCTURE ARE SEPARATE AND DISTINCT

\item	DOCUMENT TYPES PROVIDE AN ABBREVIATED METHOD FOR REFERRING TO THE
	FILE STRUCTURE
\end{nrtc}
\end{bwslide}

% run this through SLiTeX with the appropriate wrapper

\begin{bwslide}
\part	{THE FILE SERVICE}

\begin{nrtc}\bf
\item	MODEL OF OPERATION

\item	REGIMES AND SERVICES
\end{nrtc}
\end{bwslide}


\begin{bwslide}
\part*	{MODEL OF OPERATION}\bf

\begin{nrtc}
\item	FILE SERVICE INITIATOR AND RESPONDER

\item	REGIMES

\item	SERVICE PRIMITIVES AND COMMON PARAMETERS
\end{nrtc}
\end{bwslide}


\begin{bwslide}
\ctitle	{FILE SERVICE INITIATOR AND RESPONDER}

\begin{nrtc}
\item	A CLIENT-SERVER MODEL IS USED
    \begin{nrtc}
    \item	THE PROTOCOL EXCHANGE IS ASYMMETRIC

    \item	NOTE: CLIENT COULD BE ANOTHER FILESTORE
    \end{nrtc}

\item	AN INITIATOR IS A USER ENTITY WHICH REQUESTS THE FILE SERVICE

\item	A RESPONDER IS A USER ENTITY WHICH IMPLEMENTS THE VIRTUAL FILESTORE

\item	THIS PAIRING OF USERS IS TERMED AN FTAM ACTIVITY

\item	THE PROVIDER IS THE FILE SERVICE ABSTRACTION,
	IT IMPLEMENTS THE FILE SERVICE BY USING THE FILE PROTOCOL
\end{nrtc}
\end{bwslide}


\begin{bwslide}
\ctitle	{REGIMES}

\begin{nrtc}
\item	THE ASSOCIATION BETWEEN THE TWO ENTITIES PROGRESS THROUGH A NUMBER
	OF STAGES, TERMED ``REGIMES''

\item	A REGIME DETERMINES WHICH COMPONENTS OF THE FILE SERVICE MAY BE
	REQUESTED

\item	REGIMES NEST IN AN ORDERLY FASHION
\end{nrtc}
\end{bwslide}


\begin{note}\em
following are a lot of standard osi modeling mechanisms$\ldots$
\end{note}


\begin{bwslide}
\ctitle	{SERVICE PRIMITIVES AND COMMON PARAMETERS}

\begin{nrtc}
\item	THE INITIATOR AND RESPONDER COMMUNICATE VIA SERVICE PRIMITIVES

\item	A PRIMITIVE IS AN ABSTRACTION (NOT AN INTERFACE)

\item	IN GENERAL, THERE ARE THREE KINDS OF SERVICES
    \begin{nrtc}
    \item	CONFIRMED, IN WHICH A REQUEST ALWAYS RESULTS IN A RESPONSE

    \item	UNCONFIRMED, IN WHICH NO RESPONSE IS RETURNED

    \item	PROVIDER-INITIATED,
		IN WHICH THE SERVICE PROVIDER INDICATES SOME ABNORMAL
		CONDITION
    \end{nrtc}

\item	CONFIRMATION IS UNRELATED TO RELIABILITY

\item	SERVICE PRIMITIVES, LIKE PROCEDURE CALLS, HAVE TYPED PARAMETERS
\end{nrtc}
\end{bwslide}



\begin{bwslide}
\ctitle	{SERVICE PRIMITIVES}

\begin{nrtc}
\item	A SERVICE CONSISTS OF ONE OR MORE PRIMITIVES

\item	EACH PRIMITIVE IS PREFIXED WITH ``F-''

\item	A CONFIRMED SERVICE HAS FOUR PRIMITIVES
    \begin{nrtc}
    \item	.REQUEST, .INDICATION, .RESPONSE, and .CONFIRMATION
    \end{nrtc}

\item	AN UNCONFIRMED SERVICE HAS TWO PRIMITIVES:
    \begin{nrtc}
    \item	.REQUEST,  and .INDICATION
    \end{nrtc}

\item	A PROVIDER-INITIATED SERVICE HAS ONE PRIMITIVE:
    \begin{nrtc}
    \item	.INDICATION
    \end{nrtc}
\end{nrtc}
\end{bwslide}


\begin{bwslide}
\ctitle	{EXAMPLE: CONFIRMED SERVICE}

\vskip.5in
\diagram[p]{figure5}
\end{bwslide}


\begin{bwslide}
\ctitle	{COMMON PARAMETERS TO SERVICE PRIMITIVES}

\begin{nrtc}
\item	STATE RESULT: INDICATES IF A REGIME CHANGE IS SUCCESSFUL

\item	ACTION RESULT: INDICATES IF A SERVICE PRIMITIVE IS SUCCESSFUL

\item	DIAGNOSTICS: PROVIDES DETAILED INFORMATION ON THE FAILURE
	(IF ANY) OF A CONFIRMED SERVICE

\item	CHARGING: A RESOURCE, CHARGING UNIT, and CHARGE VALUE\\
	(INTERPRETATION IS UNDEFINED BY FTAM)

\item	IDENTITY OF FILE ACCESS DATA UNIT (FADU)

\item	PLUS: FILE ATTRIBUTES, REQUESTED ACCESS, etc.
\end{nrtc}
\end{bwslide}



\begin{bwslide}
\part*	{REGIMES AND SERVICES}\bf

\begin{nrtc}
\item	AS NOTED EARLIER,
	THE INNER-MOST REGIME DETERMINES WHICH SERVICE PRIMITIVES
	(AND HENCE SERVICES) ARE ACCESSIBLE

\item	THERE ARE FOUR REGIMES
    \begin{nrtc}
    \item	APPLICATION ASSOCIATION

    \item	FILE SELECTION

    \item	FILE OPEN

    \item	DATA TRANSFER
    \end{nrtc}
\end{nrtc}
\end{bwslide}


\begin{bwslide}
\ctitle	{NESTED REGIMES}

\vskip.5in
\diagram[p]{figure6}
\end{bwslide}


\begin{note}\em
as regimes and services are so hopelessly interwined,
we now bounce back-and-forth between the two in this part of the presentation
\end{note}


\begin{bwslide}
\ctitle	{APPLICATION ASSOCIATION REGIME}

\begin{nrtc}
\item	FTAM REGIME ESTABLISHMENT SERVICE
    \begin{nrtc}
    \item	WHEN TWO APPLICATIONS ARE BOUND BY AN ASSOCIATION,
		AN FTAM REGIME IS ESTABLISHED    
    \end{nrtc}

\item	FTAM REGIME TERMINATION SERVICE

\item	FTAM REGIME ABORT SERVICE

\item	DURING REGIME ESTABLISHMENT,
	PARAMETERS REGARDING THE USE OF THE FILE SERVICE ARE MANDATED OR
	NEGOTIATED
    \begin{nrtc}
    \item	SERVICE LEVEL (RELIABLE/USER-CORRECTABLE)

    \item	SERVICE CLASS

    \item	FUNCTIONAL UNITS

    \item	ATTRIBUTE GROUPS (KERNEL, STORAGE, etc.)
    \end{nrtc}
\end{nrtc}
\end{bwslide}


\begin{note}\em
observation: having defined a massive service,
we now frantically seek ways to delimit what actually gets used!
\end{note}


\begin{bwslide}
\ctitle	{SERVICE CLASS}

\begin{nrtc}
\item	FTAM SUPPORTS MANY SERVICES\\
	NEED A WAY TO CHOOSE A SUBSET OF THE SERVICES A INITIATOR DESIRES

\item	FIVE SERVICES CLASSES
    \begin{nrtc}
    \item	FILE TRANSFER

    \item	FILE ACCESS

    \item	FILE MANAGEMENT

    \item	FILE TRANSFER AND MANAGEMENT

    \item	UNCONSTRAINED
    \end{nrtc}

\item	THE SERVICE CLASS IS SELECTED BY THE INITIATOR DURING CONNECTION
	ESTABLISHMENT
\end{nrtc}
\end{bwslide}


\begin{bwslide}
\ctitle	{FUNCTIONAL UNITS}

\begin{nrtc}
\item	FUNCTIONAL UNITS, WHICH ARE NEGOTIABLE, PROVIDE A WAY TO FURTHER
	DELIMIT THE SERVICES NEEDED BY AN INITIATOR

\item	A FUNCTIONAL UNIT DEFINES WHICH SERVICES ARE AVAILABLE DURING
	THE LIFETIME OF THE FTAM REGIME

\item	THE SERVICE LEVEL AND CLASS OFTEN MANDATE CERTAIN FUNCTIONAL UNITS
\end{nrtc}
\end{bwslide}


\begin{bwslide}
\ctitle	{FUNCTIONAL UNITS (cont.)}

\begin{nrtc}
\item	KERNEL: REGIME ESTABLISHMENT/TERMINATION, FILE SELECTION/DESELECTION

\item	READ: FILE OPEN/CLOSE, READ BULK DATA

\item	WRITE: FILE OPEN/CLOSE, WRITE BULK DATA

\item	FILE ACCESS: LOCATE/ERASE FADU

\item	LIMITED FILE MANAGEMENT: CREATE/DELETE FILES, READ ATTRIBUTES

\item	ENHANCED FILE MANAGEMENT: CHANGE ATTRIBUTES

\item	GROUPING: BEGIN/END A COLLECTION OF REQUESTS

\item	RECOVERY: RECOVER PREVIOUS REGIME, CHECKPOINTING

\item	RESTART: RESTART DATA TRANSFER, CHECKPOINTING
\end{nrtc}
\end{bwslide}




\begin{bwslide}
\ctitle	{FILE SELECTION REGIME}

\begin{nrtc}
\item	WHEN A FILE IS BOUND TO FTAM REGIME,
	THE FILE SELECTION REGIME IS ESTABLISHED BY EITHER
    \begin{nrtc}
    \item	FILE SELECTION SERVICE:
		A FILE IS SELECTED, IF IT ALREADY EXISTS

    \item	FILE CREATION SERVICE:
		A FILE IS (OPTIONALLY) CREATED AND THENCE SELECTED
    \end{nrtc}

\item	ONCE A FILE IS SELECTED,
	FILE MANAGEMENT FUNCTIONS MAY BE PERFORMED BY
    \begin{nrtc}
    \item	READ ATTRIBUTE SERVICE

    \item	CHANGE ATTRIBUTE SERVICE
    \end{nrtc}

\item	AFTER ANY FILE MANAGEMENT FUNCTIONS,
	THE FILE MAY BE OPENED FOR TRANSFER AND/OR ACCESS
\end{nrtc}
\end{bwslide}


\begin{bwslide}
\ctitle	{FILE SELECTION REGIME (cont.)}

\begin{nrtc}
\item	THE FILE SELECTION REGIME IS TERMINATED BY EITHER
    \begin{nrtc}
    \item	FILE DESELECTION SERVICE:
		THE FILE IS SIMPLY UNBOUND FROM THE FTAM REGIME

    \item	FILE DELETION SERVICE:
		THE FILE IS REMOVED FROM THE FILESTORE, AND HENCE UNBOUND
    \end{nrtc}
\end{nrtc}
\end{bwslide}


\begin{bwslide}
\ctitle	{FILE OPEN REGIME}

\begin{nrtc}
\item	FILE OPEN SERVICE
    \begin{nrtc}
    \item	WHEN A FILE IS TO BE TRANSFERRED OR ACCESSED,
		THE FILE OPEN REGIME IS ESTABLISHED    
    \end{nrtc}

\item	THIS BINDS THE STRUCTURE OF THE FILE

\item	ONCE A FILE IS OPENED,
	FILE ACCESS FUNCTIONS MAY BE PERFORMED
    \begin{nrtc}
    \item	LOCATE FADU SERVICE

    \item	ERASE FADU SERVICE
    \end{nrtc}

\item	AFTER ANY FILE ACCESS FUNCTIONS,
	DATA TRANSFER MAY OCCUR

\item	THE FILE CLOSE SERVICE TERMINATES THE FILE OPEN REGIME
\end{nrtc}
\end{bwslide}


\begin{bwslide}
\ctitle	{DATA TRANSFER REGIME}

\begin{nrtc}
\item	FINALLY, WHEN DATA IS TO BE ACTUALLY TRANSFERRED,
	THE DATA TRANSFER REGIME IS ESTABLISHED

\item	THIS INVOKES A ``BULK DATA'' TRANSFER MECHANISM FOR FADUs
    \begin{nrtc}
    \item	READ BULK DATA SERVICE

    \item	WRITE BULK DATA SERVICE

    \item	DATA UNIT TRANSFER SERVICE

    \item	END OF DATA TRANSFER SERVICE

    \item	END OF TRANSFER SERVICE

    \item	CANCEL DATA TRANSFER SERVICE
    \end{nrtc}
\end{nrtc}
\end{bwslide}


\begin{bwslide}
\ctitle	{THE GROUPING SERVICE}

\begin{nrtc}
\item	TYPICALLY MANY FILE OPERATIONS HAVE THREE ACTIONS
    \begin{nrtc}
    \item	ACQUIRE THE FILE FOR DATA TRANSFER

    \item	PERFORM THE DATA TRANSFER

    \item	RELEASE THE FILE
    \end{nrtc}

\item	THE FIRST AND LAST ACTIONS CAN EACH BE VIEWED AS BEING INDIVISIBLE

\item	GROUPING PERMITS PRIMITIVES TO BE ``BUNDLED TOGETHER''
	IN ORDER TO IMPLEMENT ONE OF THESE TWO ACTIONS

\item	GROUPING IS MANDATED BY MOST FILE CLASSES
\end{nrtc}
\end{bwslide}


\begin{bwslide}
\ctitle	{THE GROUPING SERVICE (cont.)}

\begin{nrtc}
\item	THE TYPICAL ``ACQUIRE THE FILE'' GROUP:
    \begin{nrtc}
    \item	F-BEGIN-GROUP

    \item	F-SELECT

    \item	F-OPEN

    \item	F-END-GROUP
    \end{nrtc}

\item	THE TYPICAL ``RELEASE THE FILE'' GROUP:
    \begin{nrtc}
    \item	F-BEGIN-GROUP

    \item	F-CLOSE

    \item	F-DESELECT

    \item	F-END-GROUP
    \end{nrtc}
\end{nrtc}
\end{bwslide}


\begin{note}\em
more complicated groups will be examined later on

this should pretty much illustrate the distinction between state results and
actions results:
\begin{quote}
if a state-change operation fails, they all fail

otherwise if an operation fails, processing continues
\end{quote}

grouping is constrained to certain commonly used combinations
(setup and cleanup)
\end{note}


\begin{note}\em
not discussed due to time constraints:

service level: reliable, user-correctable

recover service for regime recreation

checkpoint service for mark insertion

restart service for transfer restoration

we're pressed for time,
so no examples here, later on during the implementation part of the talk,
different applications will be sketched
\end{note}


\begin{bwslide}
\part*	{SUMMARY}\bf

\begin{nrtc}
\item	THE FILE SERVICE EXISTS BETWEEN AN INITIATOR, RESPONDER, AND PROVIDER

\item	THE FILE SERVICE PROGRESSES THROUGH A NUMBER OF NESTED REGIMES,
	WHICH DETERMINE WHICH PARTS OF THE SERVICE MAY BE INVOKED

\item	THE SERVICE IS FURTHER LIMITED BY NEGOTIATION OF SERVICE ELEMENTS

\item	ALL SERVICES CENTER ON A SELECTED FILE WHICH IS TRANSFERRED,
	ACCESSED, OR MANAGED

\item	THE SERVICES ARE SUFFICIENTLY GENERAL TO SUPPORT A WIDE RANGE OF
	FILE ACTIVITIES
\end{nrtc}
\end{bwslide}

% run this through SLiTeX with the appropriate wrapper

\begin{bwslide}
\part	{THE FILE PROTOCOL}

\begin{nrtc}\bf
\item	ELEMENTS OF PROCEDURE

\item	DEFINITION AND ENCODING OF DATA UNITS

\item	FTAM USE OF LOWER-LAYER SERVICES

\item	EXAMPLES
\end{nrtc}
\end{bwslide}


\begin{bwslide}
\part*	{ELEMENTS OF PROCEDURE}%%%\bf

\begin{nrtc}
\item	THE FILE SERVICE PROVIDER EXECUTES THE FILE PROTOCOL

\item	THE PROVIDER IS ACTUALLY TWO PEER ENTITIES

\item	ASSOCIATION CONTROL IS USED TO MANAGE THE END-TO-END ASSOCIATION
	BETWEEN FILE USERS

\item	PRESENTATION SERVICES ARE USED TO EXCHANGE DATA IN A
	MACHINE-INDEPENDENT FASHION

\item	COMMITMENT, CONCURRENCY AND RECOVERY (CCR) SERVICES CAN ALSO BE USED
	FOR THE FILE TRANSFER CLASS

\item	ALL DATA UNITS (FPDUs and FADUs) ARE EXPRESSED IN TERMS OF
	ABSTRACT SYNTAX NOTATION ONE (ASN.1)
\end{nrtc}
\end{bwslide}


\begin{bwslide}
\ctitle	{FILE SERVICE REQUESTS}

\begin{nrtc}
\item	THE VALIDITY OF THE REQUEST IS VERIFIED
    \begin{nrtc}
    \item	i.e., CHECK NEGOTIATED FUNCTIONAL UNITS, INNER-MOST REGIME,
		INTERNAL STATE, and so on
    \end{nrtc}

\item	THE PARAMETERS OF THE REQUEST ARE ENCODED IN A
	FILE PROTOCOL DATA UNIT (FPDU)

\item	THE FPDU IS GIVEN TO THE PRESENTATION PROVIDER FOR DELIVERY
	TO THE REMOTE SYSTEM

\item	THE PROVIDER UPDATES ITS INTERNAL STATE
\end{nrtc}
\end{bwslide}


\begin{bwslide}
\ctitle	{ON RECEIPT OF A FILE PROTOCOL DATA UNIT}

\begin{nrtc}
\item	THE VALIDITY OF THE FPDU IS VERIFIED
    \begin{nrtc}
    \item	i.e., CHECK NEGOTIATED FUNCTIONAL UNITS, INNER-MOST REGIME,
		INTERNAL STATE,	and so on
    \end{nrtc}

\item	THE PARAMETERS OF THE FPDU ARE ENCODED IN A SERVICE .INDICATION
	OR .CONFIRMATION EVENT

\item	THE EVENT IS GIVEN TO THE FILE SERVICE USER

\item	THE PROVIDER UPDATES ITS INTERNAL STATE

\item	FADUs (DETERMINED BY PRESENTATION CONTEXT) ARE GIVEN DIRECTLY TO THE
	USER
\end{nrtc}
\end{bwslide}


\begin{bwslide}
\part*	{DEFINITION AND ENCODING OF DATA UNITS}\bf

\begin{nrtc}
\item	TWO KINDS OF DATA UNITS ARE EXCHANGED IN THE FILE SERVICE

\item	FILE PROTOCOL DATA UNITS (FPDUs) ARE EXCHANGED WITHIN THE
	FILE SERVICE PROVIDER

\item	FILE ACCESS DATA UNITS (FADUs) ARE EXCHANGED BY THE USERS OF THE
	FILE SERVICE
\end{nrtc}
\end{bwslide}


\begin{bwslide}
\ctitle	{ABSTRACT SYNTAX NOTATION ONE (ASN.1)}

\begin{nrtc}
\item	ABSTRACT SYNTAX NOTATION ONE (ASN.1) IS USED TO DESCRIBE THE STRUCTURE
	AND ENCODING OF DATA UNITS

\item	ASN.1 IS A DATA STRUCTURE DESCRIPTION LANGUAGE AND AN ENCODING
	SPECIFICATION
    \begin{nrtc}
    \item	IT IS USED TO DESCRIBE DATA STRUCTURES INDEPENDENT OF A
		GIVEN MACHINE'S INTERNAL REPRESENTATION

    \item	IT ALSO DEFINES HOW TO UNIVERSALLY ENCODE THOSE STRUCTURES
		AS THEY ARE TRANSMITTED FROM ONE MACHINE TO ANOTHER
    \end{nrtc}
\end{nrtc}
\end{bwslide}


\begin{bwslide}
\part*	{FTAM USE OF LOWER-LAYER SERVICES}\bf

\begin{nrtc}
\item	ASSOCIATION CONTROL

\item	PRESENTATION SERVICES

\item	SESSION SERVICES

\item	COMMITMENT, CONCURRENCY AND RECOVERY
\end{nrtc}
\end{bwslide}


\begin{bwslide}
\ctitle	{FTAM USE OF LOWER-LAYER SERVICES (cont.)}

\vskip.5in
\diagram[p]{figure7}
\end{bwslide}


\begin{bwslide}
\ctitle	{ASSOCIATION CONTROL}

\begin{nrtc}
\item	ASSOCIATION CONTROL IS USED BY
    \begin{nrtc}
    \item	FTAM REGIME ESTABLISHMENT SERVICE: A-ASSOCIATE

    \item	FTAM REGIME TERMINATION SERVICE: A-RELEASE

    \item	FTAM REGIME ABORT SERVICE: A-(U-)ABORT, A-P-ABORT
    \end{nrtc}

\item	NOTE THAT ASSOCIATION CONTROL MAPS DIRECTLY ONTO PRESENTATION
	SERVICES
    \begin{nrtc}
    \item	A PART OF THE APPLICATION LAYER
		(SO-CALLED COMMON APPLICATION SERVICE ENTITY)
    \end{nrtc}
\end{nrtc}
\end{bwslide}


\begin{bwslide}
\ctitle	{ADDRESSES AND APPLICATION ENTITY TITLES}

\begin{nrtc}
\item	INITIATOR PROVIDES
    \begin{nrtc}
    \item	DESCRIPTION OF FILE SERVICE DESIRED,
		e.g., ``gremlin-filestore''
    \end{nrtc}
        
\item	AND (SOMEHOW) PERFORMS TWO MAPPINGS
    \begin{nrtc}
    \item	DESCRIPTOR TO APPLICATION ENTITY TITLE PROVIDING SERVICE:
		CURRENTLY AN OBJECT IDENTIFIER

    \item	AET TO PRESENTATION ADDRESS:
		CURRENTLY P-SELECTOR, S-SELECTOR, T-SELECTOR, AND A LIST OF
		NETWORK ADDRESSES
    \end{nrtc}

\item	IN THE FUTURE, DIRECTORY SERVICES ARE USED
\end{nrtc}
\end{bwslide}


\begin{bwslide}
\ctitle	{PRESENTATION SERVICES}

\begin{nrtc}
\item	PRESENTATION SERVICES ARE USED BY THE REMAINING FTAM REGIMES: P-DATA

\item	FURTHER
    \begin{nrtc}
    \item	THE FILE OPEN SERVICE MAY REQUIRE: P-ALTER-CONTEXT

    \item	THE CANCEL DATA SERVICE REQUIRES: P-RESYNCHRONIZE

    \item	THE CHECKPOINT SERVICE REQUIRES: P-SYNC-MINOR

    \item	THE RESTART SERVICE REQUIRES: P-ALTER-CONTEXT
    \end{nrtc}

\item	IN ADDITION, ASSOCIATION CONTROL REQUIRES:
	P-CONNECT, P-RELEASE, P-U-ABORT, P-P-ABORT
\end{nrtc}
\end{bwslide}


\begin{bwslide}
\ctitle	{PRESENTATION CONTEXTS}

\begin{nrtc}
\item	ASSOCIATION CONTROL PCI (PRESENTATION CONTEXT INFORMATION)

\item	FTAM PCI

\item	IF THE PRESENTATION CONTEXT MANAGEMENT SERVICE IS UNAVAILABLE, THEN
    \begin{nrtc}
    \item	FTAM REQUESTS A CONTEXT FOR EACH DOCUMENT TYPE THAT MIGHT BE
		EXCHANGED
    \end{nrtc}

\item	FTAM PROVIDES BOTH THE ABSTRACT SYNTAX AND TRANSFER SYNTAX
	OF EACH CONTEXT
\end{nrtc}
\end{bwslide}


\begin{bwslide}
\ctitle	{COMMITMENT, CONCURRENCY AND RECOVERY}

\begin{nrtc}
\item	IF THE FILE TRANSFER CLASS IS SELECTED, AS A USER OPTION,
	THE ISO COMMITMENT, CONCURENCY, AND RECOVERY PROTOCOL CAN BE USED

\item	NEEDED FOR ATOMIC TRANSFER OF FILES (BUT NOT REALLY NEEDED FOR
	RESUMPTION OF FILE TRANSFER)

\item	PERSONAL OPINION
    \begin{nrtc}
    \item	A TREMENDOUS ``OVERKILL'' FOR ATOMIC FILE TRANSFER

    \item	NOT REALLY WELL-DEFINED AT THIS POINT
    \end{nrtc}
\end{nrtc}
\end{bwslide}


\begin{bwslide}
\ctitle	{SESSION SERVICES}

\begin{nrtc}
\item	THE FILE PROVIDER DOES NOT USE SESSION SERVICES DIRECTLY

\item	HOWEVER MOST PRESENTATION SERVICES MAP DIRECTLY ONTO SESSION SERVICES

\item	HENCE: AT LEAST S-CONNECT, S-DATA, S-RELEASE, S-U-ABORT, AND S-P-ABORT
	ARE REQUIRED

\item	AND OPTIONALLY: S-TYPED-DATA, S-RESYNCHRONIZE and S-SYNC-MINOR ARE
	ALSO REQUIRED
\end{nrtc}
\end{bwslide}


\begin{bwslide}
\part*	{EXAMPLES}\small

\begin{verbatim}
wrote F-INITIALIZE-request, context 1
{
   {
      service-class transfer-and-management-class,
      functional-units { read, write, limited-file-management,
                         enhanced-file-management },
      attribute-groups { storage, security },
      contents-type-list {
         { document-types { 1.0.8571.6.3, 1.17.3.6.1, 1.17.3.6.8 } }
      },
      initiator-identity "ANON",
      filestore-password { "mrose" }
   }
}
\end{verbatim}
\end{bwslide}


\begin{bwslide}\small
\begin{verbatim}
wrote AARQapdu, context 9
{
   protocolVersion { version1 },
   calledAEtitle 1.17.4.3.1,
   applicationContextName 1.0.8571.2.1,
   userInformation {
      data-value-identifier { indirect-reference 1 },
      encodings {
         single-ASN1-type {
            [3] '03'H,
            [4] '0136'H,
            [5] '06c0'H,
            [7] {
               [0] {
                  [APPLICATION 7] '28c27b0603'H,
                  [APPLICATION 7] '39030601'H,
                  [APPLICATION 7] '39030608'H
               }
            },
            [APPLICATION 4] "ANON",
            [APPLICATION 6] { "mrose" }
         }
      }
   }
}
\end{verbatim}
\end{bwslide}


\begin{bwslide}\small
\begin{verbatim}
wrote CPppdu
{
   { nonx410mode },
   [2] {
      [2] TRUE,
      [3] {
         { 1, 1.0.8571.1.1, { 1.0.8825 } },
         { 3, 1.0.8571.2.4, { 1.0.8571.3.4 } },
         { 5, 1.17.3.2.0, { 1.17.3.3.0 } },
         { 7, 1.17.3.2.2, { 1.17.3.3.0 } },
         { 9, 1.0.8650.2.1, { 1.0.8825 } }
      },
      [4] { 1.0.8571.1.1, 1.0.8825 },
      [5] { version-1 },
      {
         {
            {
               data-value-identifier { indirect-reference 9 },
               encodings {
                  single-ASN1-type {
                     [0] '0780'H,
                     [1] { 1.17.4.3.1 },
                     [3] { 1.0.8571.2.1 },
\end{verbatim}
\end{bwslide}


\begin{bwslide}\small
\begin{verbatim}
                     [4] {
                        [UNIVERSAL 8] {
                           1,
                           [0] {
                              [0] {
                                 [3] '03'H,
                                 [4] '0136'H,
                                 [5] '06c0'H,
                                 [7] {
                                    [0] {
                                       [APPLICATION 7] '28c27b0603'H,
                                       [APPLICATION 7] '39030601'H,
                                       [APPLICATION 7] '39030608'H
                                    }
                                 },
                                 [APPLICATION 4] "ANON",
                                 [APPLICATION 6] { "mrose" }
                              }
                           }
                        }
                     }
                  }
               }
            }
         }
      }
   }
}
\end{verbatim}
\end{bwslide}


\begin{bwslide}\small
\begin{verbatim}
---> (: dump of SPDU 0xb8404, errno=0xffffffff mask=0x409f
---> LI/ 281
---> CODE/ CONNECT
---> REFERENCE/ <USER "gremlin", COMMON  "870601045854Z", ADDITIONAL 0>
---> OPTIONS/ 0x0<>
---> TSDU/ INITIATOR: 65528, RESPONDER: 65528
---> VERSION/ 0x1
---> ISN/ 1
---> REQUIREMENTS/ 0x22<DUPLEX,RESYNC>
---> USER DATA/ 225 bytes
---> )
\end{verbatim}
\end{bwslide}


\begin{bwslide}
\part*	{SUMMARY}\bf

\begin{nrtc}
\item	THE FILE SERVICE PROVIDER IS A ``STATE MACHINE'' COMPOSED OF TWO
	PEERS EXECUTING THE FILE PROTOCOL

\item	IN ADDITION TO RESOURCES ON THEIR HOST SYSTEMS,
	THEY USE THE ASSOCIATION CONTROL AND PRESENTATION SERVICES

\item	ASN.1 IS USED TO DEFINE AND ENCODE THE DATA UNITS WHICH ARE EXCHANGED
\end{nrtc}
\end{bwslide}

% run this through SLiTeX with the appropriate wrapper

\begin{bwslide}
\part	{ISSUES IN IMPLEMENTING THE VIRTUAL FILESTORE}

\begin{nrtc}\bf
\item	A VIRTUAL FILESTORE FOR UNIX: TWO APPROACHES

\item	MAPPING THE NATIVE FILESYSTEM TO THE VIRTUAL FILESTORE
\end{nrtc}
\end{bwslide}


\begin{note}\em
one reason that the file protocol is relatively simple to implement
is that it is unconcerned with how the virtual filestore maps to the
realstore

this is the problem of the file service responder, which uses the file
service to transmit filestore abstractions, but must use the realstore
to ``emulate'' the virtual filestore

deciding on how to perform these mappings is perhaps the hardest part
of implementing ftam on a real system:
\begin{quote}
very difficult to transfer non-trivial file structures

but, the virtual filestore should make the problem $N$ rather than $N*N$
\end{quote}
\end{note}


\begin{bwslide}
\ctitle	{DIGRESSION: THE FTAM LIBRARY}

\begin{nrtc}
\item	APPROXIMATE LINES OF CODE (FOR THOSE WHO CARE)
    \begin{nrtc}
    \item	C: 23000

    \item	ASN.1: 2100 (COMPILED TO 11000 LINES OF C)
    \end{nrtc}

\item	SUPPORTS BOTH INITIATOR AND RESPONDER

\item	SERVICE LEVEL: RELIABLE

\item	SERVICE CLASSES: TRANSFER, ACCESS, MANAGEMENT, TRANSFER AND MANAGEMENT

\item	FUNCTIONAL UNITS: KERNEL, READ, WRITE, ACCESS, LIMITED FILE MANAGEMENT,
	ENHANCED FILE MANAGEMENT, GROUPING
    \begin{nrtc}
    \item	RESTRICTION: GROUPING IS REQUIRED
    \end{nrtc}

\item	ATTRIBUTE GROUPS: KERNEL, STORAGE AND SECURITY
\end{nrtc}
\end{bwslide}


\begin{bwslide}
\ctitle	{IMPLEMENTATION NOTES}

\begin{nrtc}
\item	PROCEDURE CALLS USED FOR SERVICE INTERFACE
    \begin{nrtc}
    \item	CONFIRMED SERVICES BLOCK UNTIL RESPONSE RECEIVED

    \item	F-WAIT (PSEUDO) SERVICE ADDED TO WAIT FOR .INDICATIONs
    \end{nrtc}

\item	LIBRARY SUPPORTS MULTIPLE ASSOCIATIONS
\end{nrtc}
\end{bwslide}


\begin{bwslide}
\ctitle	{AUTOMATIC TOOLS IN SUPPORT OF ASN.1}

\begin{nrtc}
\item	BECAUSE ASN.1 IS A ``FORMAL'' LANGUAGE, IT IS POSSIBLE TO WRITE
	PROGRAMS WHICH UNDERSTAND ASN.1 DESCRIPTIONS

\item	ONE EXAMPLE OF SUCH A PROGRAM IS ``PEPY'', AN ASN.1--COMPILER THAT
	PRODUCES ENCODERS, DECODERS, AND PRETTY-PRINTERS OF DATA STRUCTURES
	DEFINED BY ASN.1

\item	PEPY IS SUFFICIENTLY POWERFUL TO PRODUCE A DECODER OF THE FULL FPDU
	AND FADU SPECIFICATION (IT WAS USED EXTENSIVELY FOR THIS PURPOSE)

\item	PEPY HAS BEEN USED IN OTHER PROJECTS AND HAS PROVEN USEFUL IN
	FINDING ERRORS IN THE PDU SPECIFICATIONS
\end{nrtc}
\end{bwslide}


\begin{note}\em
actually, there is one *small* part of the FPDU specification that PEPY
doesn't support
\end{note}


\begin{bwslide}
\part*	{A VIRTUAL FILESTORE\\ FOR UNIX:\\ TWO APPROACHES}\bf

\begin{nrtc}
\item	REGARDLESS OF THE NUMBER OF DIFFERENT LOCALSTORES THAT EXIST,
	IT SHOULDN'T BE SURPRISING THAT THERE IS MORE THAN ONE WAY TO MAP A
	VIRTUAL FILESTORE ONTO A LOCALSTORE

\item	LET'S CONSIDER THE UNIX LOCALSTORE (BECAUSE OF ITS SIMPLICITY) AND
	CONSIDER TWO DIFFERENT WAYS OF IMPLEMENTING THE VIRTUAL FILESTORE
\end{nrtc}
\end{bwslide}


\begin{bwslide}
\ctitle	{REVIEW: THE UNIX LOCALSTORE}

\begin{nrtc}
\item	FILES IN UNIX ARE BYTE STREAMS, WITHOUT ANY INHERENT STRUCTURING
	INFORMATION (e.g., RECORD LENGTHS, FORMATS, etc.)

\item	TWO TYPES OF FILES ARE OF INTEREST
    \begin{nrtc}
    \item	REGULAR FILES WHICH CONTAIN DATA

    \item	DIRECTORY FILES WHICH CONTAIN POINTERS TO OTHER FILES
    \end{nrtc}

\item	THE FILESYSTEM IS HIERARCHICAL (TREE-LIKE)

\item	THERE ARE RELATIVELY FEW OPERATIONS
    \begin{nrtc}
    \item	OPEN, READ, WRITE, CLOSE: MANIPULATE REGULAR FILES

    \item	LINK, UNLINK: MANIPULATE DIRECTORY FILES

    \item	CHOWN, CHMOD, UTIMES: CHANGE FILE ATTRIBUTES

    \item	STAT: RETURN INFORMATION ABOUT A FILE
    \end{nrtc}
\end{nrtc}
\end{bwslide}


\begin{bwslide}
\ctitle	{THE ``ALTERNATE FILESYSTEM'' APPROACH}

\begin{nrtc}
\item	OBSERVATION: THE UNIX LOCALSTORE ISN'T POWERFUL ENOUGH TO PERMIT
	A DIRECT MAPPING TO THE FILESTORE

\item	SOLUTION: ``PARTITION OFF'' A PART OF THE LOCAL FILESYSTEM,
	INACCESSIBLE TO MOST USERS, AND GIVE THE FTAM RESPONDER FULL REIGN
	THERE
\end{nrtc}
\end{bwslide}


\begin{bwslide}
\ctitle	{MAPPINGS}

\begin{nrtc}
\item	START WITH A ``TOP-LEVEL'' DIRECTORY

\item	TREAT EACH FILE IN THE VIRTUAL FILESTORE AS A UNIX DIRECTORY
    \begin{nrtc}
    \item	SINCE A FTAM FILENAME IS A SEQUENCE OF GraphicString's,
		USE A DIRECTORY FOR EACH STRING IN THE SEQUENCE
    \end{nrtc}

\item	IN THE FILE'S DIRECTORY
    \begin{nrtc}
    \item	HAVE A UNIX FILE \verb"attributes" WHICH CONTAINS
		DEFINITIONS FOR ALL OF THE FILE'S ATTRIBUTES

    \item	HAVE UNIX FILES CALLED \verb"descriptor" AND \verb"DU"
		WHICH DESCRIBE THE ROOT FADU

    \item	FOR EACH CHILD OF THE ROOT, HAVE A DIRECTORY
		(NAMED \verb"1", \verb"2", etc.)
    \end{nrtc}
\end{nrtc}
\end{bwslide}


\begin{bwslide}
\ctitle	{EXAMPLE: THE ``ALTERNATE FILESYSTEM'' APPROACH}

\vskip.5in
\diagram[p]{figure4}
\end{bwslide}


\begin{bwslide}
\diagram[p]{figure8}
\end{bwslide}


\begin{bwslide}
\ctitle	{BENEFITS OF THE ``ALTERNATE FILESYSTEM'' APPROACH}

\begin{nrtc}
\item	THE UNIX FILESYSTEM CAN HOST A DATABASE WHICH IS POWERFUL ENOUGH TO
	MODEL THE FULL STRUCTURE OF FILES IN THE VIRTUAL FILESTORE

\item	BY DISTINGUISHING BETWEEN FTAM ATTRIBUTES AND UNIX ATTRIBUTES,
	THE FULL SET OF FTAM ATTRIBUTES CAN POTENTIALLY BE SUPPORTED
\end{nrtc}
\end{bwslide}


\begin{bwslide}
\ctitle	{DISADVANTAGES OF\\ THE ``ALTERNATE FILESYSTEM'' APPROACH}

\begin{nrtc}
\item	NOT A ``INTEGRATED'' APPROACH
    \begin{nrtc}
    \item	USERS WISHING TO ACCESS FILES RESIDING IN THE LOCAL VIRTUAL
		FILESTORE MUST STILL USE FTAM (SLOW AND REDUNDANT)
    \end{nrtc}

\item	NOT VERY EFFICIENT USE OF UNIX RESOURCES

\item	REQUIRES SOME RE-INVENTING OF THE WHEEL
\end{nrtc}
\end{bwslide}


\begin{bwslide}
\ctitle	{THE ``INTEGRATED'' APPROACH}

\begin{nrtc}
\item	IS THERE A WAY TO MAP UNIX FILE STRUCTURE AND ATTRIBUTES DIRECTLY
	TO THE VIRTUAL FILESTORE?

\item	CAUTION:
    \begin{quote}\em
	SOMETIMES WHEN YOU TRY TO TURN AN APPLE INTO AN ORANGE YOU GET BACK A
	LEMON!    
    \end{quote}
\end{nrtc}
\end{bwslide}


\begin{bwslide}
\part*	{MAPPING THE NATIVE FILESYSTEM TO THE VIRTUAL FILESTORE}\bf

\begin{nrtc}
\item	QUESTION: CAN AN ARBITRARY UNIX FILE APPEAR TO BE A FILE IN
	THE VIRTUAL FILESTORE?

\item	IMPLICATIONS
    \begin{nrtc}
    \item	THE UNIX FILE CONTENTS MAP TO AN FTAM FILE STRUCTURE

    \item	THE UNIX FILESYSTEM ATTRIBUTES MAP TO VIRTUAL FILESYSTEM
		ATTRIBUTES
    \end{nrtc}
\end{nrtc}
\end{bwslide}


\begin{bwslide}
\ctitle	{MAPPINGS}

\begin{nrtc}
\item	INITIALLY, CONSIDER ONLY
    \begin{nrtc}
    \item	UNSTRUCTURED BINARY FILES (FTAM-3)

    \item	UNSTRUCTURED TEXT FILES (NBS-2)

    \item	FILE DIRECTORY FILES (NBS-9)
    \end{nrtc}

\item	AUTHENTICATION
    \begin{nrtc}
    \item	FTAM INITIATOR MAPS TO USER ENTRY IN THE UNIX PASSWORD FILE

    \item	``ANON'' INITIATOR PROVIDED FOR GUEST (RESTRICTED) ACCESS
    \end{nrtc}

\item	MAPPING BETWEEN USER/GROUP IDs (UIDs/GIDs) AND FTAM ENTITIES
    \begin{nrtc}
    \item	PASSWORD OR GROUP FILE USED TO MAP NUMBER TO NAME

    \item	IF ENTRY NOT FOUND, NUMBER CONVERTED TO STRING
    \end{nrtc}

\item	MAPPING BETWEEN UNIX TIME AND FTAM DATE/TIMES
    \begin{nrtc}
    \item	STRAIGHT-FORWARD (AT LAST)
    \end{nrtc}
\end{nrtc}
\end{bwslide}


\begin{bwslide}
\ctitle	{ATTRIBUTE MAPPINGS}

\begin{nrtc}
\item	FILENAME
    \begin{nrtc}
    \item	A SINGLE STRING (WHAT MOST PROFILES SPECIFY ANYWAY)

    \item	INTERPRETED RELATIVE TO THE USER'S HOME DIRECTORY 

    \item	CHANGING THIS ATTRIBUTE RENAMES THE FILE
    \end{nrtc}

\item	CONTENTS TYPE
    \begin{nrtc}
    \item	BASED ON FILE TYPE (MORE DETAIL LATER)
    \end{nrtc}

\item	ACCOUNT
    \begin{nrtc}
    \item	THE FILE'S GID (GROUP ID) ATTRIBUTE
    \end{nrtc}

\item	DATES AND TIMES
    \begin{nrtc}\small
    \item	OF CREATION AND LAST MODIFICATION: THE FILE'S LAST MODIFICATION
		TIME

    \item	OF LAST READ ACCESS: THE FILE'S LAST ACCESS TIME

    \item	OF LAST ATTRIBUTE MODIFICATION: THE INODE'S LAST CHANGE TIME

    \item	NOT STRICTLY ACCURATE SINCE WRITING TO A FILE CHANGES EACH
    \end{nrtc}
\end{nrtc}
\end{bwslide}


\begin{bwslide}
\ctitle	{ATTRIBUTE MAPPINGS (cont.)}

\begin{nrtc}
\item	IDENTITIES:
    \begin{nrtc}
    \item	OF CREATOR: THE FILE'S OWNER

    \item	OF OTHERS: DEPENDS ON THE FILE'S MODE, OTHERWISE UNAVAILABLE
    \end{nrtc}

\item	FILE-AVAILABILITY
    \begin{nrtc}
    \item	IMMEDIATE
    \end{nrtc}

\item	PERMITTED ACTIONS
    \begin{nrtc}
    \item	DEPENDS ON THE FILE'S MODE

    \item	READ, WRITE: BASED ON INITIATOR'S PERMISSIONS

    \item	READ ATTRIBUTES: ALWAYS

    \item	CHANGE ATTRIBUTES: ONLY IF OWNER OF FILE

    \item	DELETE: BASED ON WRITABILITY OF PARENT DIRECTORY
    \end{nrtc}

\item	FILESIZE
    \begin{nrtc}
    \item	THE FILE'S SIZE
    \end{nrtc}
\end{nrtc}
\end{bwslide}


\begin{bwslide}
\ctitle	{UNAVAILABLE ATTRIBUTES}

\begin{nrtc}
\item	FUTURE FILESIZE

\item	ACCESS CONTROL

\item	ENCRYPTION NAME

\item	LEGAL QUALIFICATIONS

\item	PRIVATE USE
\end{nrtc}
\end{bwslide}


\begin{bwslide}
\ctitle	{UNSTRUCTURED BINARY FILES}

\begin{nrtc}
\item	SEMANTICS: THE DOCUMENT CONSISTS OF A SINGLE DATA UNIT
    \begin{nrtc}
    \item	THE DATA UNIT CONSISTS OF AN UNBOUNDED SEQUENCE OF DATA
		ELEMENTS

    \item	EACH DATA ELEMENT IS AN OCTET STRING

    \item	TRANSFER SYNTAX IS NOT SELF-DELIMITING
    \end{nrtc}

\item	CONSTRAINT SET: UNSTRUCTURED

\item	THIS IS SIMPLY A REGULAR UNIX FILE, NO STRUCTURE MAPPING IS REQUIRED
\end{nrtc}
\end{bwslide}


\begin{bwslide}
\ctitle	{EFFICIENCY CONSIDERATIONS}

\begin{nrtc}
\item	NEED TO DETERMINE HOW FADUs MAP TO SSDUs

\item	IDEALLY, WISH TO CONFIGURE DATA TRANSFER SUCH THAT EACH PARTIAL
	FADU SENT MAPS TO EXACTLY ONE SSDU!

\item	THIS IS (ARGUABLY) A HORRIBLE VIOLATION OF THE LAYERING PHILOSOPHY
\end{nrtc}
\end{bwslide}


\begin{bwslide}
\ctitle	{UNSTRUCTURED TEXT FILES (VARCRLF)}

\begin{nrtc}
\item	SEMANTICS: THE DOCUMENT CONSISTS OF A SINGLE DATA UNIT
    \begin{nrtc}
    \item	THE DATA UNIT CONSISTS OF AN UNBOUNDED SEQUENCE OF DATA
		ELEMENTS

    \item	EACH DATA ELEMENT IS AN IA5String, TERMINATED BY CR-LF,
		AND NEITHER CR NOR LF MAY APPEAR ANYWHERE ELSE IN THE STRING

    \item	TRANSFER SYNTAX IS NOT SELF-DELIMITING
    \end{nrtc}

\item	CONSTRAINT SET: UNSTRUCTURED

\item	HOW TO DETERMINE IF REGULAR FILE IS BINARY OR TEXT
    \begin{nrtc}
    \item	USE A HEURISTIC (AND ALL THAT IMPLIES)
    \end{nrtc}

\item	OTHERWISE, GIVEN CR-LF VS. NEWLINE HANDLING, STRAIGHT-FORWARD TO
	HANDLE
\end{nrtc}
\end{bwslide}


\begin{bwslide}
\ctitle	{FILE DIRECTORY FILES}

\begin{nrtc}
\item	ALTHOUGH THE VIRTUAL FILESTORE DOESN'T SUPPORT A ``DIRECTORY''
	CONCEPT, WE CAN EMULATE IT

\item	EXPERIENCE WITH OTHER NETWORK FILE SERVICES HAS SHOWN THIS TO BE
	VERY USEFUL
\end{nrtc}
\end{bwslide}


\begin{bwslide}
\ctitle	{FILE DIRECTORY FILES (cont.)}

\begin{nrtc}
\item	SEMANTICS: THE DOCUMENT CONSISTS OF AN UNBOUNDED SEQUENCE OF DATA UNITS
    \begin{nrtc}
    \item	EACH DATA UNIT CONSISTS OF EXACTLY ONE DATA ELEMENT OF
		TYPE FileDirectoryEntry

    \item	THE TRANSFER SYNTAX IS SELF-DELIMITING
    \end{nrtc}

\item	CONSTRAINT SET: SEQUENTIAL FLAT

\item	QUESTION: WHAT FILENAME TO REPORT IN EACH ENTRY?

\item	EFFICIENCY CONCERN: ``BUFFER'' DATA ELEMENTS FOR F-DATA SERVICE
\end{nrtc}
\end{bwslide}


\begin{bwslide}
\ctitle	{EXAMPLE: FILE DIRECTORY FILE}\small

\begin{tgrind}
\let\linebox=\relax
\input figure9\relax
\end{tgrind}
\end{bwslide}


\begin{bwslide}
\ctitle	{COMPARISON OF THE TWO APPROACHES}

\begin{nrtc}
\item	``PURITY'' OF VIRTUAL FILESTORE MAPPINGS
    \begin{nrtc}
    \item	ALTERNATE: ALL MAPPINGS CAN BE PERFORMED

    \item	INTEGRATED: MOST MAPPINGS ARE DIRECT, SOME ARE STRAINED, OTHERS
		ARE IMPOSSIBLE
    \end{nrtc}

\item	INTEGRATED: TIGHT INTERACTION WITH LOCAL ENVIRONMENT YIELDING
	SIMPLICITY OF IMPLEMENTATION
    \begin{nrtc}
    \item	ABOUT 3500 LINES OF C CODE
    \end{nrtc}

\item	EFFICIENCY OF IMPLEMENTATION
    \begin{nrtc}
    \item	ALTERNATE: EFFICIENT FOR VIRTUAL FILESTORE

    \item	INTEGRATED: EFFICIENT USE OF UNIX RESOURCES
    \end{nrtc}
\end{nrtc}
\end{bwslide}


\begin{note}\em
the responder, per se, currently runs on berkeley or at\&t unix
(although only the berkeley version has been extensively tested)

it was inspired, a bit, by the berkeley unix ftp server

the library, of course, just needs a decent C compiler, unix isn't an issue
\end{note}


\begin{bwslide}
\ctitle	{COMPARISON OF THE TWO APPROACHES (cont.)}

\begin{nrtc}
\item	FOR SIMPLE APPLICATIONS, THE ``INTEGRATED'' APPROACH IS PROBABLY
	SUPERIOR

\item	IT IS INSUFFICIENT FOR ADVANCED APPLICATIONS HOWEVER

\item	THIS SUGGESTS THAT IMPLEMENTATIONS IN THE LONG-TERM WILL PROBABLY
	EVOLVE TOWARD A HYBRID APPROACH

\item	PERHAPS THE NEXT GENERATION LOCALSTORE SHOULD BE DESIGNED WITH FTAM
	IN MIND?
\end{nrtc}
\end{bwslide}


\begin{note}\em
not discussed due to time constraints:

managing multiple associations at the file store
(needed: reasonable file-level locking facilities in the realstore)
\end{note}


\begin{bwslide}
\part*	{SUMMARY}\bf

\begin{nrtc}
\item	THE FILE PROTOCOL IS TRIVIAL COMPARED TO IMPLEMENTING A GOOD MAPPING
	OF THE VIRTUAL FILESTORE

\item	THE ``ALTERNATE FILESYSTEM'' APPROACH CAN BE USED TO PROVIDE FULL
	VIRTUAL FILESTORE SERVICES

\item	THE ``INTEGRATED'' APPROACH, WHILE NOT AS CAPABLE, IS ACCEPTABLE FOR
	MOST APPLICATIONS
\end{nrtc}
\end{bwslide}

% run this through SLiTeX with the appropriate wrapper

\begin{bwslide}
\part	{ISSUES IN IMPLEMENTING A CLIENT OF THE VIRTUAL FILESTORE}

\begin{nrtc}\bf
\item	AN INTERACTIVE FTAM INITIATOR FOR UNIX

\item	DIRECTORY HANDLING
\end{nrtc}
\end{bwslide}


\begin{bwslide}
\ctitle	{INITIATORS FOR FTAM}

\begin{nrtc}
\item	IN GENERAL, THREE APPROACHES

\item	THE INTERACTIVE APPROACH: USER (OR PROGRAM) EXPLICITLY INVOKES
	A PROGRAM TO REQUEST FILE SERVICES

\item	THE APPLICATIONS APPROACH: PROGRAM INVOKES OTHER SERVICES
	(e.g., DATABASE ACCESS) WHICH INVOKES FTAM

\item	THE EMBEDDED APPROACH: THE HOST KERNEL USES FTAM AS A PART OF ITS
	FILESYSTEM REPETOIRE
    \begin{nrtc}
    \item	THE USE OF FTAM IS TRANSPARENT TO USERS AND PROGRAMS
    \end{nrtc}

\item	THE EMBEDED APPROACH IS THE MOSE USEFUL, BUT ALSO THE MOST COMPLICATED
\end{nrtc}
\end{bwslide}


\begin{bwslide}
\part*	{AN INTERACTIVE FTAM INITIATOR FOR UNIX}%%%\bf

\begin{nrtc}\small
\item	IMPORTANT TO DISTINGUISH BETWEEN
    \begin{nrtc}
    \item	THE USER INTERFACE: HOW THE PROGRAM INTERACTS WITH THE USER

    \item	THE FTAM INTERFACE: HOW THE PROGRAM ACCESS THE FILE SERVICE
    \end{nrtc}

\item	A ``FRIENDLY'' APPROACH WAS TAKEN FOR THE USER INTERFACE
    \begin{nrtc}
    \item	USER COMMANDS TO THE INITIATOR APPEAR TO BE UNIX COMMANDS
    \end{nrtc}

\item	SUPPORTS FTAM-3, NBS-2, NBS-9
    \begin{nrtc}
    \item	THE SAME DOCUMENT TYPES SUPPORTED BY THE RESPONDER
    \end{nrtc}

\item	ABOUT 6000 LINES OF C CODE AND 150 LINES OF PEPY CODE
\end{nrtc}
\end{bwslide}


\begin{note}\em
the initiator, per se, currently runs on berkeley or at\&t unix
(although only the berkeley version has been extensively tested)

the user interface was heavily inspired by the berkeley unix ftp client
\end{note}


\begin{bwslide}
\ctitle	{COMMAND LOOP}

\begin{nrtc}
\item	ONCE INVOKED, PROGRAM PROMPTS AND ACCEPTS COMMANDS

\item	SOME COMMANDS RESULT IN FILE SERVICE ACTIONS
    \begin{nrtc}
    \item	e.g., ASSOCIATE WITH FILESTORE

    \item	UNEXPECTED DIAGNOSTICS ARE REPORTED TO THE USER
    \end{nrtc}

\item	OTHER COMMANDS SET OPTIONS CONTROLLING USE OF FILE SERVICE

\item	OTHER COMMANDS SET OPTIONS CONTROLLING LOCAL SYSTEM
\end{nrtc}
\end{bwslide}


\begin{bwslide}
\ctitle	{FILESTORE ASSOCIATION COMMANDS}

\begin{nrtc}
\item	OPEN host user [account]
    \begin{nrtc}
    \item	PROMPTS FOR A PASSWORD

    \item	ISSUES AN F-INITIALIZE.REQUEST AND AWAITS THE RESPONSE
    \end{nrtc}

\item	CLOSE
    \begin{nrtc}
    \item	TERMINATES THE ASSOCIATION WITH THE FILESTORE USING
		F-TERMINATE.REQUEST
    \end{nrtc}

\item	ABORT
    \begin{nrtc}
    \item	UPON ENCOUNTERING A FATAL ERROR, F-U-ABORT.REQUEST IS USED
    \end{nrtc}

\item	STATUS
    \begin{nrtc}
    \item	REPORTS SUMMARY OF SERVICE PARAMETERS IN EFFECT OVER FTAM
		REGIME
    \end{nrtc}
\end{nrtc}
\end{bwslide}


\begin{bwslide}
\ctitle	{EXAMPLE: FILESTORE ASSOCIATION COMMANDS}\small

\begin{verbatim}
kr:22-- ftam
ftam> open gr
user (gr:mrose): mrose
password (gr:mrose): 

gr... connected
gr> status
associated with virtual filestore on "gr" as user "mrose"
service level: reliable, service class: transfer-and-management
functional units: 0x3b<READ,WRITE,LIMITED,ENHANCED,GROUPING>
attribute groups: 0x3<STORAGE,SECURITY>
document types:
  1.0.8571.6.3  unstructured binary file
  1.17.3.6.1    unstructured text file
  1.17.3.6.8    file directory file
estimated integral FADU size: 65502
\end{verbatim}
\end{bwslide}


\begin{bwslide}
\ctitle	{FILE TRANSFER COMMANDS}

\begin{nrtc}
\item	GET source destination
    \begin{nrtc}
    \item	RETREIVES A FILE FROM THE FILESTORE

    \item	FILE SERVICES
	\begin{nrtc}
	\item	F-BEGIN-GROUP F-SELECT F-OPEN(READ) F-END-GROUP

	\item	BULK DATA TRANSFER FOR READ

	\item	F-BEGIN-GROUP F-CLOSE F-DESELECT F-END-GROUP
	\end{nrtc}
    \end{nrtc}

\item	PUT source destination
    \begin{nrtc}
    \item	STORES A FILE ON THE FILESTORE

    \item	FILE SERVICES
	\begin{nrtc}
	\item	F-BEGIN-GROUP F-CREATE(OVERWRITE) F-OPEN(WRITE) F-END-GROUP

	\item	BULK DATA TRANSFER FOR WRITE

	\item	F-BEGIN-GROUP F-CLOSE F-DESELECT F-END-GROUP
	\end{nrtc}

    \item	OPTION SETTING DETERMINES OVERWRITE MODE
    \end{nrtc}
\end{nrtc}
\end{bwslide}


\begin{note}\em
need to exercise a fair bit of caution in ordering of local file access for
the get command: 

\begin{quote}
ideally don't want to open file for write until remote open
has succeeded; but if local open fails, still have to do a little bit of bulk
data-transfer before backing out.
\end{quote}
\end{note}


\begin{bwslide}
\ctitle	{FILE TRANSFER COMMANDS (cont.)}

\begin{nrtc}
\item	APPEND source destination
    \begin{nrtc}
    \item	APPENDS TO A FILE IN THE FILESTORE

    \item	FILE SERVICES: AS ABOVE, BUT F-CREATE(EXISTING)
    \end{nrtc}

\item	OPTION SETTING DETERMINES EITHER TEXT OR BINARY
	(BUT RESPONSE FROM SERVER OVERRIDES THIS)

\item	INTERRUPT FROM USER DURING TRANFSER INVOKES F-CANCEL SERVICE
\end{nrtc}
\end{bwslide}


\begin{bwslide}
\ctitle	{EXAMPLE: FILE TRANSFER COMMANDS}\small

\begin{verbatim}
gr> set type text
type       = text       - file transfer mode
gr> get manual.dvi manual.dvi
ftam: unstructured binary file transfer
ftam: 613628 bytes received in 10.33 seconds (58.00 Kbytes/s)
gr> get calendar calendar
ftam: 2051 bytes received in 0.13 seconds (15.06 Kbytes/s)
gr> ls
file: .
.               ..              .Xdefaults      .mh_profile     bin        
calendar        lib             manual.dvi      libisode.a      mhbox
gr> get libisode.a libisode.a
ftam: unstructured binary file transfer
^C
ftam: cancelling transfer
ftam: 196506 bytes received in 5.48 seconds (35.00 Kbytes/s)
\end{verbatim}
\end{bwslide}


\begin{note}\em
this is a good example of document type simplification:

manual.dvi was not available as a text file,
hence the responder in the open request returned a simplification of the
document type requested
\end{note}


\begin{bwslide}
\ctitle	{FILE MANAGEMENT COMMANDS}

\begin{nrtc}
\item	MV source destination
    \begin{nrtc}
    \item	CHANGES THE NAME OF A FILE

    \item	FILE SERVICES
	\begin{nrtc}
	\item	F-BEGIN-GROUP F-SELECT F-CHANGE-ATTRIBUTE F-DESELECT
		F-END-GROUP
	\end{nrtc}
    \end{nrtc}

\item	RM file
    \begin{nrtc}
    \item	DELETES A FILE

    \item	FILE SERVICES
	\begin{nrtc}
	\item	F-BEGIN-GROUP F-SELECT F-DELETE F-END-GROUP
	\end{nrtc}
    \end{nrtc}
\end{nrtc}
\end{bwslide}


\begin{bwslide}
\ctitle	{FILE MANAGEMENT COMMANDS (cont.)}

\begin{nrtc}
\item	LS file
    \begin{nrtc}
    \item	LISTS THE ATTRIBUTES OF A FILE (USE DIR FOR LONG LISTING)

    \item	FILE SERVICES
	\begin{nrtc}
	\item	F-BEGIN-GROUP F-SELECT F-READ-ATTRIBUTE F-DESELECT F-END-GROUP
	\end{nrtc}

    \item	THE (APPROPRIATE) FTAM ATTRIBUTES ARE DISPLAYED AS UNIX
		ATTRIBUTES
    \end{nrtc}
\end{nrtc}
\end{bwslide}


\begin{bwslide}
\ctitle	{EXAMPLE: FILE MANAGEMENT COMMANDS}\small

\begin{verbatim}
gr> mv calendar oldcalendar
gr> rm manual.dvi
gr> ls .
.               ..              .Xdefaults      .mh_profile     bin
lib             libisode.a      mhbox           oldcalendar
gr> mv oldcalendar calendar
gr> ls .
.               ..              .Xdefaults      .mh_profile     bin
calendar        lib             libisode.a      mhbox
\end{verbatim}
\end{bwslide}


\begin{bwslide}
\ctitle	{EXAMPLE: FILE MANAGEMENT COMMANDS (cont.)}\small

\begin{verbatim}
gr> dir .
d mrose    csl          1536 Jun  2 23:25 .
d root     wheel         512 May 29 21:50 ..
t mrose    csl           563 May 27 13:09 .Xdefaults
t mrose    csl          1968 Apr 29 09:49 .mh_profile
d mrose    csl          1536 May 29 21:58 bin
t mrose    csl          2051 Jun  2 18:23 calendar
d mrose    csl           512 May  4 19:13 lib
b mrose    csl        691798 Jun  2 23:23 libisode.a
d mrose    csl          1024 Jun  2 17:43 mhbox
\end{verbatim}
\end{bwslide}


\begin{bwslide}
\ctitle	{WHAT THE RESPONDER PROVIDES}\small

\begin{verbatim}
read F-READ-ATTRIB-response, context 1
{
   {
      action-result success,
      attributes {
         { filename { "." } },
         { contents-type { document-type-name 1.17.3.6.8 } },
         { storage-account "csl" },
         { date-and-time-of-creation { actual-values "19870603062745Z" } },
         { date-and-time-of-last-modification
             { actual-values "19870603062745Z" }
         },
         { date-and-time-of-last-read-access
             { actual-values "19870603062743Z" }
         },
         { date-and-time-of-last-attribute-modification
             { actual-values "19870603062745Z" }
         },
         { identity-of-creator { actual-values "mrose" } },
         { identity-of-last-modifier { actual-values "mrose" } },
\end{verbatim}
\end{bwslide}


\begin{bwslide}\small
\begin{verbatim}
         { identity-of-last-reader { no-value-available "" } },
         { identity-of-last-attribute-modifier
             { no-value-available "mrose" }
         },
         { file-availability { actual-values immediate-availability } },
         {
            permitted-actions {
               actual-values { read, insert, replace, extend, erase,
                               read-attribute, change-attribute } }
         },
         { filesize { actual-values 1536 } },
         { future-filesize { no-value-available "" } },
         { encryption-name { no-value-available "" } },
         { legal-qualification { no-value-available "" } }
      }
   }
}
\end{verbatim}
\end{bwslide}


\begin{bwslide}
\ctitle	{LOCAL ENVIRONMENT COMMANDS (AND MISCELLANY)}

\begin{nrtc}
\item	LCD [file]
    \begin{nrtc}
    \item	CHANGES THE WORKING DIRECTORY ON THE LOCAL SYSTEM
    \end{nrtc}

\item	HELP [command]

\item	SET [variable [value]]
\end{nrtc}
\end{bwslide}


\begin{note}\em
also need protocol exception reporting and tracing

for debugging complete stacks
(and resolving episodes of finger-pointing)

good facilities for reporting exceptions and tracing protocol actions
are invaluable
\end{note}


\begin{bwslide}
\part*	{DIRECTORY HANDLING}\bf

\begin{nrtc}
\item	WHAT ARE WE MISSING?

\item	SOME KIND OF RUDIMENTARY DIRECTORY FACILITIES

\item	WILDCARDING OF FILENAMES (GLOBBING)
\end{nrtc}
\end{bwslide}


\begin{bwslide}
\ctitle	{DIRECTORY AND GLOBBING FUNCTIONS}

\begin{nrtc}
\item	CD [dir]
    \begin{nrtc}
    \item	CHANGES THE WORKING DIRECTORY ON THE VIRTUAL FILESTORE
    \end{nrtc}

\item	MKDIR dir
    \begin{nrtc}
    \item	CREATES A DIRECTORY ON THE VIRTUAL FILESTORE
    \end{nrtc}

\item	PWD
    \begin{nrtc}
    \item	PRINTS THE WORKING DIRECTORY ON THE LOCAL SYSTEM AND THE
		VIRTUAL FILESTORE
    \end{nrtc}

\item	ECHO file $\ldots$
    \begin{nrtc}
    \item	EXPANDS WILDCARD EXPRESSIONS
    \end{nrtc}

\item	PLUS: VIRTUAL FILESTORE FILENAME ARGUMENTS FOR ALL COMMANDS ARE
	GLOBBED, DEPENDING ON OPTION SETTING
\end{nrtc}
\end{bwslide}


\begin{bwslide}
\ctitle	{BUT HOW TO DO THIS?}

\begin{nrtc}
\item	MKDIR IS EASY
    \begin{nrtc}
    \item	F-BEGIN F-CREATE(NBS-9) F-DESELECT F-END
    \end{nrtc}

\item	BUT CD AND GLOBBING AREN'T POSSIBLE:
    \begin{nrtc}
    \item	THE VIRTUAL FILESTORE LACKS A DIRECTORY CONCEPT
    \end{nrtc}

\item	SUPPOSITION
    \begin{nrtc}
    \item	IF WE KNOW THE RULES FOR BUILDING FILENAMES ON THE REMOTE
		SYSTEM,

    \item	THEN WE CAN EMULATE THIS CONCEPT IN THE INITIATOR!

    \item	WE CAN ALSO MAKE THE USER INTERFACE SMARTER AS WELL
		(BY DEFAULTING MISSING ARGUMENTS)
    \end{nrtc}
\end{nrtc}
\end{bwslide}


\begin{bwslide}
\ctitle	{APPROACH}

\begin{nrtc}
\item	LET THE ``REALSTORE'' OPTION DETERMINE HOW FILENAMES ARE BUILT

\item	ON A CD COMMAND
    \begin{nrtc}
    \item	REMEMBER THE ARGUMENT

    \item	NORMALIZE FUTURE ARGUMENTS USING THIS STRING
    \end{nrtc}

\item	FOR GLOBBING
    \begin{nrtc}
    \item	BASED ON THE REALSTORE, SELECT A GLOBBING ROUTINE WHICH
		IS INTEGRATED WITH FTAM

    \item	FOR THE UNIX LOCALSTORE THIS IS TRIVIAL
    \end{nrtc}
\end{nrtc}
\end{bwslide}


\begin{note}\em
the globbing facility in the interactive ftam initiator is
based on routines generously supplied by u.c. berkeley
\end{note}


\begin{bwslide}
\ctitle	{EXAMPLE: DIRECTORY HANDLING COMMANDS}\small

\begin{verbatim}
gr> set realstore unix
realstore  = unix       - type of remote realstore
gr> ls
bin             lib             mhbox
calendar        libisode.a      nrtc
gr> ls lib
MakeUpdEnv      ctype           icons           notes           termcap
bfly            emacs           la-template     sl-template     tex82
gr> ls lib/*tem*
lib/la-template:
Makefile        template.bbl    template.tex    text.tex        version.sh

lib/sl-template:
Makefile        figure.pic      template.tex
gr> cd lib/sl-template
gr> pwd
local directory: /f/iso/ftam2
virtual filestore directory: lib/sl-template
gr> ls
Makefile        figure.pic      template.tex
\end{verbatim}
\end{bwslide}


\begin{bwslide}
\ctitle	{THE REALSTORE CONTROVERSY}

\begin{nrtc}
\item	THIS IS AN $M*N$ (RATHER THAN $M+N\/$) APPROACH
    \begin{nrtc}
    \item	THE INITIATOR ON ALL LOCAL SYSTEMS MUST POTENTIALLY KNOW ABOUT
		ALL LOCAL SYSTEMS HOSTING A VIRTUAL FILESTORE
    \end{nrtc}

\item	NEEDLESS TO SAY THIS IS CONTRARY TO THE OSI PHILOSOPHY
    \begin{nrtc}
    \item	IT'S PROBABLY WORSE THAN THE PRIVATE ATTRIBUTE GROUP IN
		THE FILESTORE
    \end{nrtc}

\item	IT'S ALSO AMAZINGLY USEFUL\\ (THE FOREWARD PROGRESS ARGUMENT)

\item	THE BEST SOLUTION, HOWEVER, WOULD BE FOR FTAM TO BE FIXED
\end{nrtc}
\end{bwslide}


\begin{bwslide}
\part*	{SUMMARY}\bf

\begin{nrtc}
\item	AN INTERACTIVE FTAM CLIENT WITH REASONABLE CAPABILITIES CAN BE
	IMPLEMENTED IN A STRAIGHT-FORWARD FASHION

\item	DIRECTORY HANDLING CAN BE IMPLEMENTED, BUT AT A CONCEPTUAL COST
\end{nrtc}
\end{bwslide}

% run this through SLiTeX with the appropriate wrapper

\begin{bwslide}
\part	{FTAM STATUS\\ AS OF 1 JUNE 1987}

\begin{nrtc}\bf
\item	STATUS OF THE FTAM STANDARD

\item	USER PROFILES

\item	CONFORMANCE TESTING
\end{nrtc}
\end{bwslide}


\begin{note}\em
strong disclaimer: this section of the presentation is a limited snapshot of
the state of the world as of june first

everything will have changed by june second
\end{note}


\begin{bwslide}
\part*	{STATUS OF THE FTAM STANDARD}\bf

\begin{nrtc}
\item	FTAM IS CURRENTLY A DRAFT INTERNATIONAL STANDARD (DIS)
	AS OF OCTOBER, 1986

\item	THE DRAFT OF THE FTAM INTERNATIONAL STANDARD (IS) WILL PROBABLY
	BE AVAILABLE BY THE FOURTH QUARTER, 1987

\item	MOST RELATED STANDARDS IN PLACE PRIOR TO THEN
\end{nrtc}
\end{bwslide}


\begin{bwslide}
\ctitle	{THE FINAL FTAM STANDARD\\ WILL PROBABLY:}

\begin{nrtc}
\item	INCLUDE MORE TUTORIAL INFORMATION!

\item	SIMPLIFY THE RELIABLE AND USER-CORRECTABLE SERVICE LEVELS INTO ONE
	(USER-VISIBLE) SERVICE LEVEL

\item	MAKE THE SERVICE CLASS NEGOTIABLE

\item	SIMPLIFY THE SYNTAX OF FADUs SOMEWHAT (FA, DU and DE MAPPINGS)

\item	CORRECT A NUMBER OF BUGLETS AND TYPOS
\end{nrtc}
\end{bwslide}


\begin{bwslide}
\part*	{USER PROFILES}\bf

\begin{nrtc}
\item	MOST U.S. PROFILES SEEM TO HAVE ALIGNED WITH
	NBS IMPLEMENTATION AGREEMENTS FOR OSI PROTOCOLS
    \begin{nrtc}
    \item	GOSIP

    \item	MAP/TOP

    \item	COS
    \end{nrtc}

\item	EUROPEANS SEEM VERY INTERESTED AS WELL
    \begin{nrtc}
    \item	SPAG

    \item	CEN/CENELEC

    \item	CEPT
    \end{nrtc}

\item	STANDARDIZATION OF FUNCTIONAL PROFILES IS OCCURING BETWEEN THESE
	GROUPS
\end{nrtc}
\end{bwslide}


\begin{bwslide}
\part*	{CONFORMANCE TESTING}\bf

\begin{nrtc}
\item	THE NATIONAL PHYSICAL LABORATORY IN THE UK IS BUILDING AN
	FTAM (AND EMBEDDED SESSION) TESTING SYSTEM

\item	THE NATIONAL COMPUTER CENTRE (UK) WILL RUN THEIR FTAM CONFORMANCE
	TESTING SERVICE BASED ON THIS SYSTEM

\item	THE NCC IS MAKING THIS FTAM TESTING SYSTEM AVAILABLE TO COS AND
	VARIOUS EUROPEAN TESTING SERVICES

\item	HENCE, THERE SHOULD BE HARMONIZATION OF CONFORMANCE TESTING BETWEEN
	THE US AND EUROPE
\end{nrtc}
\end{bwslide}


\begin{note}\em
to summarize:

things may be finalized by mid-1988
\end{note}


\begin{bwslide}
\part*	{ACKNOWLEDGEMENTS}

\begin{quote}\em
this presentation is based on experiences in implementing the
ISO Development Environment (ISODE) at NRTC,
an openly available implementation of the upper-layers of OSI

others have made significant contributions to the content and quality of this
presentation,
notably:
\end{quote}

\begin{nrtc}\em
\item	at UCL: Steve Kille

\item	at BRL: Mike Muuss

\item	at NPL: John Pavel

\item	at NRTC: John L.~Romine

\item	at NMA: Einar A.~Stefferud
\end{nrtc}

also, UNIX is a trademark of at\&t bell laboratories
\end{bwslide}

\end{document}
